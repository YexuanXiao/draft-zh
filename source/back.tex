%!TEX root = std.tex

\renewcommand{\leftmark}{\bibname}

\begin{thebibliography}{99}
% ISO 文档按数字顺序排列。
\bibitem{iso4217}
  ISO 4217:2015,
  \doccite{表示货币的代码}
\bibitem{iso10967-1}
  ISO/IEC 10967-1:2012,
  \doccite{信息技术 --- 语言独立算术 ---
    第 1 部分:整数和浮点数算术}
\bibitem{iso14882:2023}
  ISO/IEC 14882:2023,
  \doccite{编程语言 --- \Cpp{}}
\bibitem{iso14882:2020}
  ISO/IEC 14882:2020,
  \doccite{编程语言 --- \Cpp{}}
\bibitem{iso14882:2017}
  ISO/IEC 14882:2017,
  \doccite{编程语言 --- \Cpp{}}
\bibitem{iso14882:2014}
  ISO/IEC 14882:2014,
  \doccite{信息技术 --- 编程语言 --- \Cpp{}}
\bibitem{iso14882:2011}
  ISO/IEC 14882:2011,
  \doccite{信息技术 --- 编程语言 --- \Cpp{}}
\bibitem{iso14882:2003}
  ISO/IEC 14882:2003,
  \doccite{编程语言 --- \Cpp{}}
\bibitem{iso18661-3}
  ISO/IEC TS 18661-3:2015,
  \doccite{信息技术 ---
    编程语言、其环境和系统软件接口 ---
    C 的浮点数扩展 --- 第 3 部分:交换和扩展类型}
% 其他国际标准。
\bibitem{iana-charset}
  IANA 字符集数据库。
  可访问于:\newline
  \url{https://www.iana.org/assignments/character-sets/}, 2021-04-01
\bibitem{iana-tz}
  IANA 时区数据库。
  可访问于: \url{https://www.iana.org/time-zones}
\bibitem{unicode-charmap}
  Unicode 字符映射标记语言 [在线]。
  由 Mark Davis 和 Markus Scherer 编辑。修订版 5.0.1;2017-05-31
  可访问于: \url{https://www.unicode.org/reports/tr22/tr22-8.html}
% 文献引用。
\bibitem{cpp-r}
  Bjarne Stroustrup,
  \doccite{The \Cpp{} Programming Language, 第二版}, 第 R 章。
  Addison-Wesley Publishing Company, ISBN 0-201-53992-6, 版权 \copyright 1991 AT\&T
\bibitem{kr}
  Brian W.\ Kernighan 和 Dennis M.\ Ritchie,
  \doccite{The C Programming Language}, 附录 A。
  Prentice-Hall, 1978, ISBN 0-13-110163-3, 版权 \copyright 1978 AT\&T
\bibitem{cpp-lib}
  P.\,J.\ Plauger,
  \doccite{The Draft Standard \Cpp{} Library}。
  Prentice-Hall, ISBN 0-13-117003-1, 版权 \copyright 1995 P.\,J.\ Plauger
\bibitem{linalg-stable}
  J.\ Demmel, I.\ Dumitriu, 和 O.\ Holtz,
  \doccite{快速线性代数是稳定的},
  Numerische Mathematik 108 (59--91), 2007。
\bibitem{blas1}
  C.\,L.\ Lawson, R.\,J.\ Hanson, D.\ Kincaid, 和 F.\,T.\ Krogh,
  \doccite{用于 Fortran 的基本线性代数子程序}。
  ACM 数学软件汇刊, 卷 5, 页 308--323, 1979。
\bibitem{blas2}
  Jack J.\ Dongarra, Jeremy Du Croz, Sven Hammarling, 和 Richard J.\ Hanson,
  \doccite{一组扩展的 FORTRAN 基本线性代数子程序}。
  ACM 数学软件汇刊, 卷 14, 第 1 期, 页 1--17, 三月 1988。
\bibitem{blas3}
  Jack J.\ Dongarra, Jeremy Du Croz, Sven Hammarling, 和 Iain Duff,
  \doccite{一组 3 级基本线性代数子程序}。
  ACM 数学软件汇刊, 卷 16, 第 1 期, 页 1--17, 三月 1990。
\bibitem{lapack}
  E.\ Anderson, Z.\ Bai, C.\ Bischof, S.\ Blackford, J.\ Demmel, J.\ Dongarra,
  J.\ Du Croz, A.\ Greenbaum, S.\ Hammarling, A.\ McKenney, 和 D.\ Sorensen,
  \doccite{LAPACK 用户指南, 第三版}。
  SIAM, Philadelphia, PA, USA, 1999。
\bibitem{blas-std}
  L.\ Susan Blackford, James Demmel, Jack Dongarra, Iain Duff, Sven Hammarling,
  Greg Henry, Michael Heroux, Linda Kaufman, Andrew Lumsdaine, Antoine Petitet,
  Roldan Pozo, Karin Remington, 和 R.\ Clint Whaley,
  \doccite{一组更新的基本线性代数子程序 (BLAS)}。
  ACM 数学软件汇刊, 卷 28, 第 2 期, 页 135--151, 2002。
\bibitem{flynn-taxonomy}
  Michael J.\ Flynn,
  \doccite{超高速计算系统}。
  IEEE 会刊, 卷 54, 第 12 期, 页 1901--1909, 1966。
\end{thebibliography}

% FIXME: 由于未知原因,在我们的术语表中,悬挂段落默认不缩进。
\let\realglossitem\glossitem
\renewcommand{\glossitem}[4]{\hangpara{4em}{1}\realglossitem{#1}{#2}{#3}{#4}}

\clearpage
\renewcommand{\glossaryname}{交叉引用}
\renewcommand{\preglossaryhook}{下面按标签字母顺序列出了每个条款和子条款的标签,以及相应的条款或子条款编号和页码。\\}
\twocolglossary
\renewcommand{\leftmark}{\glossaryname}
{
\raggedright
\printglossary[xrefindex]
}

\clearpage
\input{xrefdelta}
\renewcommand{\glossaryname}{来自 ISO \CppXVII{} 的交叉引用}
\renewcommand{\preglossaryhook}{本文件包含了 ISO \CppXVII{} (ISO/IEC 14882:2017, \doccite{编程语言 --- \Cpp{}}) 的所有条款和子条款标签,但以下描述的例外情况除外。\\}
\renewcommand{\leftmark}{\glossaryname}
{
\raggedright
\printglossary[xrefdelta]
}

\clearpage
\renewcommand{\leftmark}{\indexname}
\renewcommand{\preindexhook}{名称以 \exposid{等宽斜体} 出现的结构仅用于说明。\\}
{
\raggedright
\printindex[generalindex]
}

\clearpage
\renewcommand{\preindexhook}{每个条目的第一个粗体页码是定义文法产生式的一般文本中的页码。第二个粗体页码是语法摘要\iref{gram}中对应的页码。其他页码指的是在一般文本中提到该文法产生式的页码。\\}
{
\raggedright
\printindex[grammarindex]
}

\clearpage
\renewcommand{\preindexhook}{每个条目的粗体页码指的是显示该头文件概要的页码。\\}
{
\raggedright
\printindex[headerindex]
}

\clearpage
\renewcommand{\preindexhook}{名称以 \exposid{斜体} 出现的结构仅用于说明。\\}
{
\raggedright
\printindex[libraryindex]
}

\clearpage
\renewcommand{\preindexhook}{每个条目的粗体页码是定义该概念的页码。
其他页码指的是在一般文本中提到该概念的页码。
名称以 \exposid{斜体} 出现的概念仅用于说明。\\}
{
\raggedright
\printindex[conceptindex]
}

\clearpage
\renewcommand{\preindexhook}{本索引中的条目是粗略描述;精确规范请参见一般文本中指示的页码。\\}
{
\raggedright
\printindex[impldefindex]
}