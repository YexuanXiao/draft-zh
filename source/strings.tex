%!TEX root = std.tex
\rSec0[strings]{字符串库}

\rSec1[strings.general]{概述}

\pnum
本条款描述了用于操作任何非数组的可平凡复制的标准布局\iref{term.standard.layout.type}类型 \tcode{T} 的序列的组件,
其中 \tcode{is_trivially_default_constructible_v<T>} 为 \tcode{true}。
此类类型称为\defnx{类字符类型}{char-like type},
而类字符类型的对象称为\defnx{类字符对象}{char-like object},或简称为\term{字符}。

\pnum
以下子条款描述了字符特性类、字符串类以及空终止序列工具,
其概要如\tref{strings.summary}所示。

\begin{libsumtab}[x{2.1in}]{字符串库摘要}{strings.summary}
\ref{char.traits}     & 字符特性                    & \tcode{<string>}  \\
\ref{string.view}     & 字符串视图类                & \tcode{<string_view>} \\ \rowsep
\ref{string.classes}  & 字符串类                    & \tcode{<string>}  \\ \rowsep
\ref{c.strings}       & 空终止序列工具              & \tcode{<cstring>} \\
\end{libsumtab}

\rSec1[char.traits]{字符特性}

\rSec2[char.traits.general]{概述}

\pnum
子章节 \ref{char.traits} 定义了用于表示
\term{字符特性}
的类的需求,
并定义了类模板
\tcode{char_traits<charT>},
以及五个特化:
\tcode{char_traits<char>}、
\tcode{char_traits<char8_t>}、
\tcode{char_traits<char16_t>}、
\tcode{char_traits<char32_t>}
和
\tcode{char_traits<wchar_t>},
它们满足这些需求。

\pnum
\ref{string.classes}、\ref{string.view}
和 \ref{input.output} 中规定的大多数类需要一组相关的类型和函数来完善
其语义的定义。这些类型和函数通过模板参数 \tcode{traits}
中的成员 \grammarterm{typedef-name}{s} 和函数集合提供,
每个此类模板都使用该参数。
子章节 \ref{char.traits} 定义了这些成员的语义。

\pnum
为了特化那些模板以生成一个字符串、字符串视图或
输入输出流类来处理特定的字符容器类型\iref{defns.character.container}
\tcode{C},
该类型及其相关的字符特性类
\tcode{X}
将作为一对参数传递给字符串、字符串视图或输入输出流模板,参数名为
\tcode{charT}
和
\tcode{traits}。
如果
\tcode{X::char_type}
与
\tcode{C}
不是同一类型,则程序为非良构。

\rSec2[char.traits.require]{字符特性要求}
\pnum
在\tref{char.traits.req}中,
\tcode{X}
表示一个定义字符容器类型\tcode{C}的类型与函数的特性类;
\tcode{c}
与
\tcode{d}
表示\tcode{C}类型的值;
\tcode{p}
与
\tcode{q}
表示\tcode{const C*}类型的值;
\tcode{s}
表示\tcode{C*}类型的值;
\tcode{n}、
\tcode{i}
与
\tcode{j}
表示\tcode{size_t}类型的值;
\tcode{e}
与
\tcode{f}
表示\tcode{X::int_type}类型的值;
\tcode{pos}
表示\tcode{X::pos_type}类型的值;
而
\tcode{r}
表示\tcode{C}类型的左值。
在\ref{char.traits.require}中指定的字符特性要求的一部分的表达式均不应通过异常退出。

\begin{libreqtab4d}
{字符特性要求}
{char.traits.req}
\\ \topline
\lhdr{表达式}       &   \chdr{返回类型}  &   \chdr{断言/注释}   &   \rhdr{复杂度}\\
                    &                       &   \chdr{前/后条件}   &               \\ \capsep
\endfirsthead
\continuedcaption\\
\topline
\lhdr{表达式}       &   \chdr{返回类型}  &   \chdr{断言/注释}   &   \rhdr{复杂度}\\
                    &                       &   \chdr{前/后条件}   &               \\ \capsep
\endhead
\tcode{X::char_type}    &   \tcode{C}       &
   &    \\ \rowsep
\tcode{X::int_type} &                       &
(描述于~\ref{char.traits.typedefs})   &    \\ \rowsep
\tcode{X::off_type} &                       &
(描述于~\ref{iostreams.limits.pos}与\ref{iostream.forward})   &    \\ \rowsep
\tcode{X::pos_type} &                       &
(描述于~\ref{iostreams.limits.pos}与\ref{iostream.forward})   &    \\ \rowsep
\tcode{X::state_type}   &                       &
(描述于~\ref{char.traits.typedefs})   &    \\ \rowsep
\tcode{X::eq(c,d)}      &   \tcode{bool}        &
 \returns
\tcode{c}是否应当被视为与\tcode{d}相等。   &   常量    \\ \rowsep
\tcode{X::lt(c,d)}      &   \tcode{bool}        &
 \returns
\tcode{c}是否应当被视为小于\tcode{d}。  &   常量    \\ \rowsep
\tcode{X::compare(p,q,n)}   &   \tcode{int}     &
 \returns
若对于\range{0}{n}中的每个\tcode{i},\tcode{X::eq(p[i],q[i])}均为\tcode{true},则返回\tcode{0};否则,若对于\range{0}{n}中的某个\tcode{j},\tcode{X::lt(p[j],q[j])}为\tcode{true}且对于\range{0}{j}中的每个\tcode{i},\tcode{X::eq(p[i],q[i])}为\tcode{true},则返回负值;否则返回正值。            &   线性      \\ \rowsep
\tcode{X::length(p)}    &   \tcode{size_t}     &
 \returns
使得\tcode{X::eq(p[i],charT())}为\tcode{true}的最小\tcode{i}。  &   线性  \\ \rowsep
\tcode{X::find(p,n,c)}  &   \tcode{const X::char_type*} &
 \returns
\range{p}{p+n}中使得\tcode{X::eq(*q,c)}为\tcode{true}的最小\tcode{q},否则返回\tcode{nullptr}。                &   线性      \\ \rowsep
\tcode{X::move(s,p,n)}  &   \tcode{X::char_type*}   &
对于\range{0}{n}中的每个\tcode{i},执行\tcode{X::assign(s[i],p[i])}。
即使在\range{p}{p+n}与\range{s}{s+n}范围重叠时也能正确复制。\br \returns \tcode{s}。    &   线性  \\ \rowsep
\tcode{X::copy(s,p,n)}  &   \tcode{X::char_type*}   &
 \expects
范围\range{p}{p+n}与\range{s}{s+n}不重叠。\par
 \returns
\tcode{s}。\br
对于\range{0}{n}中的每个\tcode{i},执行\tcode{X::assign(s[i],p[i])}。               &   线性      \\ \rowsep
\tcode{X::assign(r,d)}  &   (未使用)          &
将\tcode{r=d}赋值。                            &   常量        \\ \rowsep
\tcode{X::assign\- (s,n,c)}  &   \tcode{X::char_type*}   &
对于\range{0}{n}中的每个\tcode{i},执行\tcode{X::assign(s[i],c)}。\br
 \returns
\tcode{s}。                       &   线性      \\ \rowsep
\tcode{X::not_eof(e)}   &   \tcode{int_type}        &
 \returns
若\tcode{X::eq_int_type(e,X::eof())}为\tcode{false},则返回\tcode{e},
否则返回一个值\tcode{f}使得\tcode{X::eq_int_type(f,X::eof())}为\tcode{false}。                       &   常量    \\ \rowsep
\tcode{X::to_char_type\- (e)}    &   \tcode{X::char_type}    &
 \returns
若对于某个\tcode{c},\tcode{X::eq_int_type(e,X::to_int_type(c))}
为\tcode{true},则返回\tcode{c};否则返回某个未指定的值。                    &   常量    \\ \rowsep
\tcode{X::to_int_type\- (c)} &   \tcode{X::int_type} &
 \returns
某个值\tcode{e},受\tcode{to_char_type}与\tcode{eq_int_type}的定义约束。                  &   常量    \\ \rowsep
\tcode{X::eq_int_type\- (e,f)}   &   \tcode{bool}            &
 \returns
对于所有\tcode{c}与\tcode{d},\tcode{X::eq(c,d)}等于
\tcode{X::eq_int_type(X::to_int_type(c), X::to_int_type(d))};否则,若\tcode{e}与\tcode{f}均为\tcode{X::eof()}的副本,则产生\tcode{true};否则,若\tcode{e}与\tcode{f}其中之一是\tcode{X::eof()}的副本而另一个不是,则产生\tcode{false};否则
该值未指定。                                           &   常量    \\ \rowsep
\tcode{X::eof()}                &   \tcode{X::int_type} &
 \returns
一个值\tcode{e}使得对于所有值\tcode{c},\tcode{X::eq_int_type(e,X::to_int_type(c))}
均为\tcode{false}。                                  &   常量    \\
\end{libreqtab4d}

\pnum
在头文件 \libheaderref{string} 中提供的类模板
\indexlibraryglobal{char_traits}%
\begin{codeblock}
template<class charT> struct char_traits;
\end{codeblock}
是作为显式特化的基础。

\rSec2[char.traits.typedefs]{特征 typedef}

\indexlibrarymember{char_type}{char_traits}%
\indexlibrarymember{int_type}{char_traits}%
\begin{itemdecl}
using int_type = @\seebelow@;
\end{itemdecl}

\begin{itemdescr}
\pnum
\expects
\tcode{int_type}
应能表示所有由相应
\tcode{char_type}
值转换而来的有效字符,以及一个文件结束值
\tcode{eof()}。
\begin{footnote}
如果
\tcode{eof()}
能存放于
\tcode{char_type}
中,则某些 iostream 操作可能会产生令人惊讶的结果。
\end{footnote}
\end{itemdescr}

\indexlibrarymember{state_type}{char_traits}%
\begin{itemdecl}
using state_type = @\seebelow@;
\end{itemdecl}

\begin{itemdescr}
\pnum
\expects
\tcode{state_type} 满足
\oldconcept{Destructible}(\tref{cpp17.destructible})、
\oldconcept{CopyAssignable}(\tref{cpp17.copyassignable})、
\oldconcept{CopyConstructible}(\tref{cpp17.copyconstructible})以及
\oldconcept{DefaultConstructible}(\tref{cpp17.defaultconstructible})要求。
\end{itemdescr}

\rSec2[char.traits.specializations]{\tcode{char_traits} 特化}

\rSec3[char.traits.specializations.general]{概述}

\indexlibraryglobal{char_traits}%
\begin{codeblock}
namespace std {
  template<> struct char_traits<char>;
  template<> struct char_traits<char8_t>;
  template<> struct char_traits<char16_t>;
  template<> struct char_traits<char32_t>;
  template<> struct char_traits<wchar_t>;
}
\end{codeblock}

\pnum
头文件 \libheader{string}
定义了类模板 \tcode{char_traits} 的五个特化:
\tcode{char_traits<\brk{}char>}、
\tcode{char_traits<char8_t>}、
\tcode{char_traits<char16_t>}、
\tcode{char_traits<char32_t>}
以及
\tcode{char_traits<wchar_t>}。

\rSec3[char.traits.specializations.char]{\tcode{struct char_traits<char>}}

\indexlibraryglobal{char_traits}%
\begin{codeblock}
namespace std {
  template<> struct char_traits<char> {
    using char_type  = char;
    using int_type   = int;
    using off_type   = streamoff;
    using pos_type   = streampos;
    using state_type = mbstate_t;
    using comparison_category = strong_ordering;

    static constexpr void assign(char_type& c1, const char_type& c2) noexcept;
    static constexpr bool eq(char_type c1, char_type c2) noexcept;
    static constexpr bool lt(char_type c1, char_type c2) noexcept;

    static constexpr int compare(const char_type* s1, const char_type* s2, size_t n);
    static constexpr size_t length(const char_type* s);
    static constexpr const char_type* find(const char_type* s, size_t n,
                                           const char_type& a);
    static constexpr char_type* move(char_type* s1, const char_type* s2, size_t n);
    static constexpr char_type* copy(char_type* s1, const char_type* s2, size_t n);
    static constexpr char_type* assign(char_type* s, size_t n, char_type a);

    static constexpr int_type not_eof(int_type c) noexcept;
    static constexpr char_type to_char_type(int_type c) noexcept;
    static constexpr int_type to_int_type(char_type c) noexcept;
    static constexpr bool eq_int_type(int_type c1, int_type c2) noexcept;
    static constexpr int_type eof() noexcept;
  };
}
\end{codeblock}

\pnum
类型 \tcode{mbstate_t} 在 \libheaderref{cwchar} 中定义,
且可以表示在任何\impldef{受支持的多字节字符编码规则集}的受支持多字节字符编码规则集中可能出现的转换状态。

\pnum
双参数成员 \tcode{assign} 的定义与内置运算符 \tcode{=} 相同。双参数成员 \tcode{eq} 和 \tcode{lt} 的定义与针对 \tcode{unsigned char} 类型的内置运算符 \tcode{==} 和 \tcode{<} 相同。

\pnum
成员 \tcode{eof()} 返回 \tcode{EOF}。

\rSec3[char.traits.specializations.char8.t]{\tcode{struct char_traits<char8_t>}}

\indexlibraryglobal{char_traits}%
\begin{codeblock}
namespace std {
  template<> struct char_traits<char8_t> {
    using char_type  = char8_t;
    using int_type   = unsigned int;
    using off_type   = streamoff;
    using pos_type   = u8streampos;
    using state_type = mbstate_t;
    using comparison_category = strong_ordering;

    static constexpr void assign(char_type& c1, const char_type& c2) noexcept;
    static constexpr bool eq(char_type c1, char_type c2) noexcept;
    static constexpr bool lt(char_type c1, char_type c2) noexcept;

    static constexpr int compare(const char_type* s1, const char_type* s2, size_t n);
    static constexpr size_t length(const char_type* s);
    static constexpr const char_type* find(const char_type* s, size_t n,
                                           const char_type& a);
    static constexpr char_type* move(char_type* s1, const char_type* s2, size_t n);
    static constexpr char_type* copy(char_type* s1, const char_type* s2, size_t n);
    static constexpr char_type* assign(char_type* s, size_t n, char_type a);
    static constexpr int_type not_eof(int_type c) noexcept;
    static constexpr char_type to_char_type(int_type c) noexcept;
    static constexpr int_type to_int_type(char_type c) noexcept;
    static constexpr bool eq_int_type(int_type c1, int_type c2) noexcept;
    static constexpr int_type eof() noexcept;
  };
}
\end{codeblock}

\pnum
双参数成员 \tcode{assign}、\tcode{eq} 和 \tcode{lt}
被定义为分别与
内置运算符 \tcode{=}、\tcode{==} 和 \tcode{<} 完全相同。

\pnum
\indextext{UTF-8}%
成员 \tcode{eof()} 返回一个
\impldef{return value of \tcode{char_traits<char8_t>::eof}} 常量,
该常量不能作为有效的 UTF-8 代码单元。

\rSec3[char.traits.specializations.char16.t]{\tcode{struct char_traits<char16_t>}}

\indexlibraryglobal{char_traits}%
\begin{codeblock}
namespace std {
  template<> struct char_traits<char16_t> {
    using char_type  = char16_t;
    using int_type   = uint_least16_t;
    using off_type   = streamoff;
    using pos_type   = u16streampos;
    using state_type = mbstate_t;
    using comparison_category = strong_ordering;

    static constexpr void assign(char_type& c1, const char_type& c2) noexcept;
    static constexpr bool eq(char_type c1, char_type c2) noexcept;
    static constexpr bool lt(char_type c1, char_type c2) noexcept;

    static constexpr int compare(const char_type* s1, const char_type* s2, size_t n);
    static constexpr size_t length(const char_type* s);
    static constexpr const char_type* find(const char_type* s, size_t n,
                                           const char_type& a);
    static constexpr char_type* move(char_type* s1, const char_type* s2, size_t n);
    static constexpr char_type* copy(char_type* s1, const char_type* s2, size_t n);
    static constexpr char_type* assign(char_type* s, size_t n, char_type a);

    static constexpr int_type not_eof(int_type c) noexcept;
    static constexpr char_type to_char_type(int_type c) noexcept;
    static constexpr int_type to_int_type(char_type c) noexcept;
    static constexpr bool eq_int_type(int_type c1, int_type c2) noexcept;
    static constexpr int_type eof() noexcept;
  };
}
\end{codeblock}


\pnum
双参数成员\tcode{assign}、\tcode{eq}和\tcode{lt}的定义分别与内置运算符\tcode{=}、\tcode{==}和\tcode{<}完全相同。

\pnum
成员\tcode{eof()}返回一个\impldef{char_traits<char16_t>::eof 的返回值}常量,该常量不能作为有效的UTF-16码元出现。

\rSec3[char.traits.specializations.char32.t]{\tcode{struct char_traits<char32_t>}}

\indexlibraryglobal{char_traits}%
\begin{codeblock}
namespace std {
  template<> struct char_traits<char32_t> {
    using char_type  = char32_t;
    using int_type   = uint_least32_t;
    using off_type   = streamoff;
    using pos_type   = u32streampos;
    using state_type = mbstate_t;
    using comparison_category = strong_ordering;

    static constexpr void assign(char_type& c1, const char_type& c2) noexcept;
    static constexpr bool eq(char_type c1, char_type c2) noexcept;
    static constexpr bool lt(char_type c1, char_type c2) noexcept;

    static constexpr int compare(const char_type* s1, const char_type* s2, size_t n);
    static constexpr size_t length(const char_type* s);
    static constexpr const char_type* find(const char_type* s, size_t n,
                                           const char_type& a);
    static constexpr char_type* move(char_type* s1, const char_type* s2, size_t n);
    static constexpr char_type* copy(char_type* s1, const char_type* s2, size_t n);
    static constexpr char_type* assign(char_type* s, size_t n, char_type a);

    static constexpr int_type not_eof(int_type c) noexcept;
    static constexpr char_type to_char_type(int_type c) noexcept;
    static constexpr int_type to_int_type(char_type c) noexcept;
    static constexpr bool eq_int_type(int_type c1, int_type c2) noexcept;
    static constexpr int_type eof() noexcept;
  };
}
\end{codeblock}


\pnum
双参数的成员 \tcode{assign}、\tcode{eq} 和 \tcode{lt} 的定义分别与内置运算符 \tcode{=}、\tcode{==} 和 \tcode{<} 相同。

\pnum
成员 \tcode{eof()} 返回一个 \impldef{\tcode{char_traits<char32_t>::eof} 的返回值} 常量,该常量不能作为 Unicode 码位出现。

\rSec3[char.traits.specializations.wchar.t]{\tcode{struct char_traits<wchar_t>}}

\indexlibraryglobal{char_traits}%
\begin{codeblock}
namespace std {
  template<> struct char_traits<wchar_t> {
    using char_type  = wchar_t;
    using int_type   = wint_t;
    using off_type   = streamoff;
    using pos_type   = wstreampos;
    using state_type = mbstate_t;
    using comparison_category = strong_ordering;

    static constexpr void assign(char_type& c1, const char_type& c2) noexcept;
    static constexpr bool eq(char_type c1, char_type c2) noexcept;
    static constexpr bool lt(char_type c1, char_type c2) noexcept;

    static constexpr int compare(const char_type* s1, const char_type* s2, size_t n);
    static constexpr size_t length(const char_type* s);
    static constexpr const char_type* find(const char_type* s, size_t n,
                                           const char_type& a);
    static constexpr char_type* move(char_type* s1, const char_type* s2, size_t n);
    static constexpr char_type* copy(char_type* s1, const char_type* s2, size_t n);
    static constexpr char_type* assign(char_type* s, size_t n, char_type a);

    static constexpr int_type not_eof(int_type c) noexcept;
    static constexpr char_type to_char_type(int_type c) noexcept;
    static constexpr int_type to_int_type(char_type c) noexcept;
    static constexpr bool eq_int_type(int_type c1, int_type c2) noexcept;
    static constexpr int_type eof() noexcept;
  };
}
\end{codeblock}


\pnum
双参数成员
\tcode{assign}、
\tcode{eq}、
与
\tcode{lt}
的定义方式分别等同于内置运算符
\tcode{=}、
\tcode{==}、
与
\tcode{<}。

\pnum
成员
\tcode{eof()}
返回
\tcode{WEOF}。

\rSec1[string.view]{字符串视图类}

\rSec2[string.view.general]{概述}

\pnum
类模板 \tcode{basic_string_view} 描述一个对象,它可以引用一个常量连续序列的类字符\iref{strings.general}对象,该序列的第一个元素位于位置零。
在\ref{string.view}的后续部分中,存储在 \tcode{basic_string_view} 对象中的类字符对象的类型由 \tcode{charT} 指代。

\pnum
\begin{note}
库提供了从 \tcode{const charT*} 和 \tcode{std::basic_string<charT, ...>} 到 \tcode{std::basic_string_view<charT, ...>} 的隐式转换,因此用户代码在任何需要字符序列的地方,可以只接受 \tcode{std::basic_string_view<charT>} 作为非模板化参数。
用户定义的类型可以定义自己的到 \tcode{std::basic_string_view<charT>} 的隐式转换,以便与这些函数互操作。
\end{note}

\rSec2[string.view.synop]{头文件 \tcode{<string_view>} 概要}

\indexheader{string_view}%
\begin{codeblock}
// 主要独立
#include <compare>              // 见 \ref{compare.syn}

namespace std {
  // \ref{string.view.template},类模板 \tcode{basic_string_view}
  template<class charT, class traits = char_traits<charT>>
  class basic_string_view;                                              // 部分独立

  template<class charT, class traits>
    constexpr bool ranges::@\libspec{enable_view}{basic_string_view}@<basic_string_view<charT, traits>> = true;
  template<class charT, class traits>
    constexpr bool ranges::@\libspec{enable_borrowed_range}{basic_string_view}@<basic_string_view<charT, traits>> = true;

  // \ref{string.view.comparison},非成员比较函数
  template<class charT, class traits>
    constexpr bool operator==(basic_string_view<charT, traits> x,
                              type_identity_t<basic_string_view<charT, traits>> y) noexcept;
  template<class charT, class traits>
    constexpr @\seebelow@ operator<=>(basic_string_view<charT, traits> x,
              @\itcorr@                      type_identity_t<basic_string_view<charT,
              @\itcorr@                                      traits>> y) noexcept;

  // \ref{string.view.io},插入器与提取器
  template<class charT, class traits>
    basic_ostream<charT, traits>&
      operator<<(basic_ostream<charT, traits>& os,
                 basic_string_view<charT, traits> str);                 // 托管

  // \tcode{basic_string_view} 的 \grammarterm{typedef-name}
  using @\libglobal{string_view}@    = basic_string_view<char>;
  using @\libglobal{u8string_view}@  = basic_string_view<char8_t>;
  using @\libglobal{u16string_view}@ = basic_string_view<char16_t>;
  using @\libglobal{u32string_view}@ = basic_string_view<char32_t>;
  using @\libglobal{wstring_view}@   = basic_string_view<wchar_t>;

  // \ref{string.view.hash},哈希支持
  template<class T> struct hash;
  template<> struct hash<string_view>;
  template<> struct hash<u8string_view>;
  template<> struct hash<u16string_view>;
  template<> struct hash<u32string_view>;
  template<> struct hash<wstring_view>;

  inline namespace literals {
    inline namespace string_view_literals {
      // \ref{string.view.literals},\tcode{basic_string_view} 字面量后缀
      constexpr string_view    operator""sv(const char* str, size_t len) noexcept;
      constexpr u8string_view  operator""sv(const char8_t* str, size_t len) noexcept;
      constexpr u16string_view operator""sv(const char16_t* str, size_t len) noexcept;
      constexpr u32string_view operator""sv(const char32_t* str, size_t len) noexcept;
      constexpr wstring_view   operator""sv(const wchar_t* str, size_t len) noexcept;
    }
  }
}
\end{codeblock}

\pnum
当包含 \tcode{<string_view>} 时,\ref{utility.swap} 和 \ref{iterator.range} 中定义的函数模板可用。

\rSec2[string.view.template]{类模板 \tcode{basic_string_view}}

\rSec3[string.view.template.general]{概述}

\indexlibraryglobal{basic_string_view}%
\indexlibrarymember{traits_type}{basic_string_view}%
\indexlibrarymember{value_type}{basic_string_view}%
\indexlibrarymember{pointer}{basic_string_view}%
\indexlibrarymember{const_pointer}{basic_string_view}%
\indexlibrarymember{reference}{basic_string_view}%
\indexlibrarymember{const_reference}{basic_string_view}%
\indexlibrarymember{const_iterator}{basic_string_view}%
\indexlibrarymember{iterator}{basic_string_view}%
\indexlibrarymember{const_reverse_iterator}{basic_string_view}%
\indexlibrarymember{reverse_iterator}{basic_string_view}%
\indexlibrarymember{size_type}{basic_string_view}%
\indexlibrarymember{difference_type}{basic_string_view}%
\begin{codeblock}
namespace std {
  template<class charT, class traits = char_traits<charT>>
  class basic_string_view {
  public:
    // 类型
    using traits_type            = traits;
    using value_type             = charT;
    using pointer                = value_type*;
    using const_pointer          = const value_type*;
    using reference              = value_type&;
    using const_reference        = const value_type&;
    using const_iterator         = @\impdefx{类型 \tcode{basic_string_view::const_iterator}}@; // 见 \ref{string.view.iterators}
    using iterator               = const_iterator;@
\begin{footnote}
因为 \tcode{basic_string_view} 引用的是常量序列,\tcode{iterator} 与 \tcode{const_iterator} 是同一类型。
\end{footnote}@
    using const_reverse_iterator = reverse_iterator<const_iterator>;
    using reverse_iterator       = const_reverse_iterator;
    using size_type              = size_t;
    using difference_type        = ptrdiff_t;
    static constexpr size_type npos = size_type(-1);

    // \ref{string.view.cons},构造与赋值
    constexpr basic_string_view() noexcept;
    constexpr basic_string_view(const basic_string_view&) noexcept = default;
    constexpr basic_string_view& operator=(const basic_string_view&) noexcept = default;
    constexpr basic_string_view(const charT* str);
    basic_string_view(nullptr_t) = delete;
    constexpr basic_string_view(const charT* str, size_type len);
    template<class It, class End>
      constexpr basic_string_view(It begin, End end);
    template<class R>
      constexpr explicit basic_string_view(R&& r);

    // \ref{string.view.iterators},迭代器支持
    constexpr const_iterator begin() const noexcept;
    constexpr const_iterator end() const noexcept;
    constexpr const_iterator cbegin() const noexcept;
    constexpr const_iterator cend() const noexcept;
    constexpr const_reverse_iterator rbegin() const noexcept;
    constexpr const_reverse_iterator rend() const noexcept;
    constexpr const_reverse_iterator crbegin() const noexcept;
    constexpr const_reverse_iterator crend() const noexcept;

    // \ref{string.view.capacity},容量
    constexpr size_type size() const noexcept;
    constexpr size_type length() const noexcept;
    constexpr size_type max_size() const noexcept;
    constexpr bool empty() const noexcept;

    // \ref{string.view.access},元素访问
    constexpr const_reference operator[](size_type pos) const;
    constexpr const_reference at(size_type pos) const;                  // freestanding-deleted
    constexpr const_reference front() const;
    constexpr const_reference back() const;
    constexpr const_pointer data() const noexcept;

    // \ref{string.view.modifiers},修改器
    constexpr void remove_prefix(size_type n);
    constexpr void remove_suffix(size_type n);
    constexpr void swap(basic_string_view& s) noexcept;

    // \ref{string.view.ops},字符串操作
    constexpr size_type copy(charT* s, size_type n,
                             size_type pos = 0) const;                  // freestanding-deleted

    constexpr basic_string_view substr(size_type pos = 0,
                                       size_type n = npos) const;       // freestanding-deleted
    constexpr basic_string_view subview(size_type pos = 0,
                                        size_type n = npos) const;      // freestanding-deleted

    constexpr int compare(basic_string_view s) const noexcept;
    constexpr int compare(size_type pos1, size_type n1,
                          basic_string_view s) const;                   // freestanding-deleted
    constexpr int compare(size_type pos1, size_type n1, basic_string_view s,
                          size_type pos2, size_type n2) const;          // freestanding-deleted
    constexpr int compare(const charT* s) const;
    constexpr int compare(size_type pos1, size_type n1,
                          const charT* s) const;                        // freestanding-deleted
    constexpr int compare(size_type pos1, size_type n1, const charT* s,
                          size_type n2) const;                          // freestanding-deleted

    constexpr bool starts_with(basic_string_view x) const noexcept;
    constexpr bool starts_with(charT x) const noexcept;
    constexpr bool starts_with(const charT* x) const;
    constexpr bool ends_with(basic_string_view x) const noexcept;
    constexpr bool ends_with(charT x) const noexcept;
    constexpr bool ends_with(const charT* x) const;

    constexpr bool contains(basic_string_view x) const noexcept;
    constexpr bool contains(charT x) const noexcept;
    constexpr bool contains(const charT* x) const;

    // \ref{string.view.find},查找
    constexpr size_type find(basic_string_view s, size_type pos = 0) const noexcept;
    constexpr size_type find(charT c, size_type pos = 0) const noexcept;
    constexpr size_type find(const charT* s, size_type pos, size_type n) const;
    constexpr size_type find(const charT* s, size_type pos = 0) const;
    constexpr size_type rfind(basic_string_view s, size_type pos = npos) const noexcept;
    constexpr size_type rfind(charT c, size_type pos = npos) const noexcept;
    constexpr size_type rfind(const charT* s, size_type pos, size_type n) const;
    constexpr size_type rfind(const charT* s, size_type pos = npos) const;

    constexpr size_type find_first_of(basic_string_view s, size_type pos = 0) const noexcept;
    constexpr size_type find_first_of(charT c, size_type pos = 0) const noexcept;
    constexpr size_type find_first_of(const charT* s, size_type pos, size_type n) const;
    constexpr size_type find_first_of(const charT* s, size_type pos = 0) const;
    constexpr size_type find_last_of(basic_string_view s, size_type pos = npos) const noexcept;
    constexpr size_type find_last_of(charT c, size_type pos = npos) const noexcept;
    constexpr size_type find_last_of(const charT* s, size_type pos, size_type n) const;
    constexpr size_type find_last_of(const charT* s, size_type pos = npos) const;
    constexpr size_type find_first_not_of(basic_string_view s, size_type pos = 0) const noexcept;
    constexpr size_type find_first_not_of(charT c, size_type pos = 0) const noexcept;
    constexpr size_type find_first_not_of(const charT* s, size_type pos,
                                          size_type n) const;
    constexpr size_type find_first_not_of(const charT* s, size_type pos = 0) const;
    constexpr size_type find_last_not_of(basic_string_view s,
                                         size_type pos = npos) const noexcept;
    constexpr size_type find_last_not_of(charT c, size_type pos = npos) const noexcept;
    constexpr size_type find_last_not_of(const charT* s, size_type pos,
                                         size_type n) const;
    constexpr size_type find_last_not_of(const charT* s, size_type pos = npos) const;

  private:
    const_pointer @\exposid{data_}@;        // \expos
    size_type @\exposid{size_}@;            // \expos
  };

  // \ref{string.view.deduct},推导指引
  template<class It, class End>
    basic_string_view(It, End) -> basic_string_view<iter_value_t<It>>;
  template<class R>
    basic_string_view(R&&) -> basic_string_view<ranges::range_value_t<R>>;
}
\end{codeblock}

\pnum
在每个特化 \tcode{basic_string_view<charT, traits>} 中,类型 \tcode{traits} 应当满足字符特征要求\iref{char.traits}。
\begin{note}
若 \tcode{traits::char_type} 与 \tcode{charT} 不是同一类型,则程序非良构。
\end{note}

\pnum
对于 \tcode{basic_string_view str},
任何使范围
\begin{codeblock}
@\range{str.data()}{str.data() + str.size()}@
\end{codeblock}
中指针失效的操作,
亦会使指向 \tcode{str} 元素的指针、迭代器及引用失效。

\pnum
\tcode{basic_string_view} 成员函数的复杂度是 \bigoh{1},除非另有规定。
\pnum
\tcode{basic_string_view<charT, traits>} 是一个可平凡复制类型\iref{term.trivially.copyable.type}。

\rSec3[string.view.cons]{构造与赋值}

\indexlibraryctor{basic_string_view}%
\begin{itemdecl}
constexpr basic_string_view() noexcept;
\end{itemdecl}

\begin{itemdescr}
\pnum
\ensures
\tcode{\exposid{size_} == 0} 且 \tcode{\exposid{data_} == nullptr}。
\end{itemdescr}

\indexlibraryctor{basic_string_view}%
\begin{itemdecl}
constexpr basic_string_view(const charT* str);
\end{itemdecl}

\begin{itemdescr}
\pnum
\expects
\range{str}{str + traits::length(str)} 是一个有效范围。

\pnum
\effects
构造一个 \tcode{basic_string_view},用 \tcode{str} 初始化 \exposid{data_},
并用 \tcode{traits::length(str)} 初始化 \exposid{size_}。

\pnum
\complexity
\bigoh{\tcode{traits::length(str)}}。
\end{itemdescr}

\indexlibraryctor{basic_string_view}%
\begin{itemdecl}
constexpr basic_string_view(const charT* str, size_type len);
\end{itemdecl}

\begin{itemdescr}
\pnum
\expects
\range{str}{str + len} 是一个有效范围。

\pnum
\effects
构造一个 \tcode{basic_string_view},用 \tcode{str} 初始化 \exposid{data_},
并用 \tcode{len} 初始化 \exposid{size_}。
\end{itemdescr}

\indexlibraryctor{basic_string_view}%
\begin{itemdecl}
template<class It, class End>
  constexpr basic_string_view(It begin, End end);
\end{itemdecl}

\begin{itemdescr}
\pnum
\constraints
\begin{itemize}
\item \tcode{It} 满足 \libconcept{contiguous_iterator}。
\item \tcode{End} 满足 \tcode{\libconcept{sized_sentinel_for}<It>}。
\item \tcode{is_same_v<iter_value_t<It>, charT>} 为 \tcode{true}。
\item \tcode{is_convertible_v<End, size_type>} 为 \tcode{false}。
\end{itemize}

\pnum
\expects
\begin{itemize}
\item \range{begin}{end} 是有效的范围。
\item \tcode{It} 满足 \libconcept{contiguous_iterator}。
\item \tcode{End} 满足 \tcode{\libconcept{sized_sentinel_for}<It>}。
\end{itemize}

\pnum
\effects
以 \tcode{to_address(begin)} 初始化 \exposid{data_},
并以 \tcode{end - begin} 初始化 \exposid{size_}。

\pnum
\throws
当 \tcode{end - begin} 抛出时所抛出的内容。
\end{itemdescr}

\indexlibraryctor{basic_string_view}%
\begin{itemdecl}
template<class R>
  constexpr explicit basic_string_view(R&& r);
\end{itemdecl}

\begin{itemdescr}
\pnum
令 \tcode{d} 为类型 \tcode{remove_cvref_t<R>} 的左值。

\pnum
\constraints
\begin{itemize}
\item
\tcode{remove_cvref_t<R>} 与 \tcode{basic_string_view} 不是同一类型,
\item
\tcode{R} 满足
\tcode{ranges::\libconcept{contiguous_range}} 与 \tcode{ranges::\libconcept{sized_range}},
\item
\tcode{is_same_v<ranges::range_value_t<R>, charT>} 为 \tcode{true},
\item
\tcode{is_convertible_v<R, const charT*>} 为 \tcode{false},且
\item
\tcode{d.operator ::std::basic_string_view<charT, traits>()}
不是合法的表达式。
\end{itemize}

\pnum
\effects
以 \tcode{ranges::data(r)} 初始化 \exposid{data_} 并以
\tcode{ranges::size(r)} 初始化 \exposid{size_}。

\pnum
\throws
\tcode{ranges::data(r)} 与 \tcode{ranges::size(r)} 所抛出的任何异常。
\end{itemdescr}

\rSec3[string.view.deduct]{推导指引}

\begin{itemdecl}
template<class It, class End>
  basic_string_view(It, End) -> basic_string_view<iter_value_t<It>>;
\end{itemdecl}

\begin{itemdescr}
\pnum
\constraints
\begin{itemize}
\item \tcode{It} 满足 \libconcept{contiguous_iterator}。
\item \tcode{End} 满足 \tcode{\libconcept{sized_sentinel_for}<It>}。
\end{itemize}
\end{itemdescr}

\begin{itemdecl}
template<class R>
  basic_string_view(R&&) -> basic_string_view<ranges::range_value_t<R>>;
\end{itemdecl}

\begin{itemdescr}
\pnum
\constraints
\tcode{R} 满足 \tcode{ranges::\libconcept{contiguous_range}}。
\end{itemdescr}

\rSec3[string.view.iterators]{迭代器支持}

\indexlibrarymember{const_iterator}{basic_string_view}%
\begin{itemdecl}
using const_iterator = @\impdefx{basic_string_view::const_iterator 的类型}@;
\end{itemdecl}

\begin{itemdescr}
\pnum
一种类型,它满足常量
\oldconcept{RandomAccessIterator}\iref{random.access.iterators} 的要求,
符合 \libconcept{contiguous_iterator}\iref{iterator.concept.contiguous} 概念,
并且满足 constexpr 迭代器要求\iref{iterator.requirements.general},
其 \tcode{value_type} 是模板参数 \tcode{charT}。

\pnum
所有对容器的迭代器要求\iref{container.requirements} 同样适用于 \tcode{basic_string_view::const_iterator}。
\end{itemdescr}

\indexlibrarymember{begin}{basic_string_view}%
\indexlibrarymember{cbegin}{basic_string_view}%
\begin{itemdecl}
constexpr const_iterator begin() const noexcept;
constexpr const_iterator cbegin() const noexcept;
\end{itemdecl}

\begin{itemdescr}
\pnum
\returns
一个迭代器,满足
\begin{itemize}
\item 若 \tcode{!empty()},则 \tcode{addressof(*begin()) == \exposid{data_}},
\item 否则,一个未指定的值,使得 \range{begin()}{end()} 是一个有效的范围。
\end{itemize}
\end{itemdescr}

\indexlibrarymember{end}{basic_string_view}%
\indexlibrarymember{cend}{basic_string_view}%
\begin{itemdecl}
constexpr const_iterator end() const noexcept;
constexpr const_iterator cend() const noexcept;
\end{itemdecl}

\begin{itemdescr}
\pnum
\returns
\tcode{begin() + size()}.
\end{itemdescr}

\indexlibrarymember{rbegin}{basic_string_view}%
\indexlibrarymember{crbegin}{basic_string_view}%
\begin{itemdecl}
constexpr const_reverse_iterator rbegin() const noexcept;
constexpr const_reverse_iterator crbegin() const noexcept;
\end{itemdecl}

\begin{itemdescr}
\pnum
\returns
\tcode{const_reverse_iterator(end())}.
\end{itemdescr}

\indexlibrarymember{rend}{basic_string_view}%
\indexlibrarymember{crend}{basic_string_view}%
\begin{itemdecl}
constexpr const_reverse_iterator rend() const noexcept;
constexpr const_reverse_iterator crend() const noexcept;
\end{itemdecl}

\begin{itemdescr}
\pnum
\returns
\tcode{const_reverse_iterator(begin())}。
\end{itemdescr}

\rSec3[string.view.capacity]{容量}

\indexlibrarymember{size}{basic_string_view}%
\indexlibrarymember{length}{basic_string_view}%
\begin{itemdecl}
constexpr size_type size() const noexcept;
constexpr size_type length() const noexcept;
\end{itemdecl}

\begin{itemdescr}
\pnum
\returns
\exposid{size_}。
\end{itemdescr}


\indexlibrarymember{max_size}{basic_string_view}%
\begin{itemdecl}
constexpr size_type max_size() const noexcept;
\end{itemdecl}

\begin{itemdescr}
\pnum
\returns
\tcode{basic_string_view} 所能引用的字符类对象的最大可能数目。
\end{itemdescr}

\indexlibrarymember{empty}{basic_string_view}%
\begin{itemdecl}
constexpr bool empty() const noexcept;
\end{itemdecl}

\begin{itemdescr}
\pnum
\returns
\tcode{\exposid{size_} == 0}。
\end{itemdescr}

\rSec3[string.view.access]{元素访问}

\indexlibrarymember{operator[]}{basic_string_view}%
\begin{itemdecl}
constexpr const_reference operator[](size_type pos) const;
\end{itemdecl}

\begin{itemdescr}
\pnum
\hardexpects
\tcode{pos < size()} 为 \tcode{true}。
\begin{note}
此先决条件强于 \tcode{basic_string::operator[]} 上的那个。
\end{note}

\pnum
\returns
\tcode{\exposid{data_}[pos]}。

\pnum
\throws
无。
\end{itemdescr}

\indexlibrarymember{at}{basic_string_view}%
\begin{itemdecl}
constexpr const_reference at(size_type pos) const;
\end{itemdecl}

\begin{itemdescr}
\pnum
\returns
\tcode{\exposid{data_}[pos]}。

\pnum
\throws
若 \tcode{pos >= size()} 则抛出 \tcode{out_of_range}。
\end{itemdescr}

\indexlibrarymember{front}{basic_string_view}%
\begin{itemdecl}
constexpr const_reference front() const;
\end{itemdecl}

\begin{itemdescr}
\pnum
\hardexpects
\tcode{empty()} 为 \tcode{false}。

\pnum
\returns
\tcode{\exposid{data_}[0]}。

\pnum
\throws
无。
\end{itemdescr}

\indexlibrarymember{back}{basic_string_view}%
\begin{itemdecl}
constexpr const_reference back() const;
\end{itemdecl}

\begin{itemdescr}
\pnum
\hardexpects
\tcode{empty()} 为 \tcode{false}。

\pnum
\returns
\tcode{\exposid{data_}[size() - 1]}。

\pnum
\throws
无。
\end{itemdescr}

\indexlibrarymember{data}{basic_string_view}%
\begin{itemdecl}
constexpr const_pointer data() const noexcept;
\end{itemdecl}

\begin{itemdescr}
\pnum
\returns
\exposid{data_}。

\pnum
\begin{note}
与 \tcode{basic_string::data()} 和 \grammarterm{string-literal} 不同,
\tcode{data()} 可以返回指向非空终止的缓冲区的指针。
因此,将 \tcode{data()} 传递给仅接受 \tcode{const charT*} 并期待空终止字符串的函数通常是错误的。
\end{note}
\end{itemdescr}

\rSec3[string.view.modifiers]{修饰器}

\indexlibrarymember{remove_prefix}{basic_string_view}%
\begin{itemdecl}
constexpr void remove_prefix(size_type n);
\end{itemdecl}

\begin{itemdescr}
\pnum
\hardexpects
\tcode{n <= size()} 为 \tcode{true}。

\pnum
\effects
等价于:\tcode{\exposid{data_} += n; \exposid{size_} -= n;}
\end{itemdescr}

\indexlibrarymember{remove_suffix}{basic_string_view}%
\begin{itemdecl}
constexpr void remove_suffix(size_type n);
\end{itemdecl}

\begin{itemdescr}
\pnum
\hardexpects
\tcode{n <= size()} 为 \tcode{true}。

\pnum
\effects
等价于:\tcode{\exposid{size_} -= n;}
\end{itemdescr}

\indexlibrarymember{swap}{basic_string_view}%
\begin{itemdecl}
constexpr void swap(basic_string_view& s) noexcept;
\end{itemdecl}

\begin{itemdescr}
\pnum
\effects
交换 \tcode{*this} 与 \tcode{s} 的值。
\end{itemdescr}

\rSec3[string.view.ops]{字符串操作}

\indexlibrarymember{copy}{basic_string_view}%
\begin{itemdecl}
constexpr size_type copy(charT* s, size_type n, size_type pos = 0) const;
\end{itemdecl}

\begin{itemdescr}
\pnum
令 \tcode{rlen} 为 \tcode{n} 与 \tcode{size() - pos} 中的较小者。

\pnum
\expects
\range{s}{s + rlen} 是一个有效范围。

\pnum
\effects
等价于 \tcode{traits::copy(s, data() + pos, rlen)}。

\pnum
\returns
\tcode{rlen}。

\pnum
\throws
若 \tcode{pos > size()} 则抛出 \tcode{out_of_range}。

\pnum
\complexity
\bigoh{\tcode{rlen}}。
\end{itemdescr}

\indexlibrarymember{substr}{basic_string_view}%
\indexlibrarymember{subview}{basic_string_view}%
\begin{itemdecl}
constexpr basic_string_view substr(size_type pos = 0, size_type n = npos) const;
constexpr basic_string_view subview(size_type pos = 0, size_type n = npos) const;
\end{itemdecl}

\begin{itemdescr}
\pnum
令 \tcode{rlen} 为 \tcode{n} 与 \tcode{size() - pos} 中的较小者。

\pnum
\effects
确定 \tcode{rlen},即待引用的字符串的有效长度。

\pnum
\returns
\tcode{basic_string_view(data() + pos, rlen)}。

\pnum
\throws
若 \tcode{pos > size()} 则抛出 \tcode{out_of_range}。
\end{itemdescr}

\indexlibrarymember{compare}{basic_string_view}%
\begin{itemdecl}
constexpr int compare(basic_string_view str) const noexcept;
\end{itemdecl}

\begin{itemdescr}
\pnum
令 \tcode{rlen} 为 \tcode{size()} 和 \tcode{str.size()} 的较小者。

\pnum
\effects
确定 \tcode{rlen} ,即待比较字符串的有效长度。
随后通过调用 \tcode{traits::compare(data(), str.data(), rlen)} 比较两个字符串。

\pnum
\returns
若比较结果非零,则返回该非零结果。
否则,返回 \tref{string.view.compare} 所示之值。
\begin{libtab2}{\tcode{compare()} 结果}{string.view.compare}{cc}{条件}{返回值}
\tcode{size() < str.size()}  & \tcode{< 0}\\
\tcode{size() == str.size()} & \tcode{0}\\
\tcode{size() > str.size()} & \tcode{> 0}\\
\end{libtab2}

\pnum
\complexity
\bigoh{\tcode{rlen}}。
\end{itemdescr}

\indexlibrarymember{compare}{basic_string_view}%
\begin{itemdecl}
constexpr int compare(size_type pos1, size_type n1, basic_string_view str) const;
\end{itemdecl}

\begin{itemdescr}
\pnum
\effects
等效于:\tcode{return substr(pos1, n1).compare(str);}
\end{itemdescr}

\indexlibrarymember{compare}{basic_string_view}%
\begin{itemdecl}
constexpr int compare(size_type pos1, size_type n1, basic_string_view str,
                      size_type pos2, size_type n2) const;
\end{itemdecl}

\begin{itemdescr}
\pnum
\effects
等效于:\tcode{return substr(pos1, n1).compare(str.substr(pos2, n2));}
\end{itemdescr}

\indexlibrarymember{compare}{basic_string_view}%
\begin{itemdecl}
constexpr int compare(const charT* s) const;
\end{itemdecl}

\begin{itemdescr}
\pnum
\effects
等价于: \tcode{return compare(basic_string_view(s));}
\end{itemdescr}

\indexlibrarymember{compare}{basic_string_view}%
\begin{itemdecl}
constexpr int compare(size_type pos1, size_type n1, const charT* s) const;
\end{itemdecl}

\begin{itemdescr}
\pnum
\effects
等价于: \tcode{return substr(pos1, n1).compare(basic_string_view(s));}
\end{itemdescr}

\indexlibrarymember{compare}{basic_string_view}%
\begin{itemdecl}
constexpr int compare(size_type pos1, size_type n1, const charT* s, size_type n2) const;
\end{itemdecl}

\begin{itemdescr}
\pnum
\effects
等价于: \tcode{return substr(pos1, n1).compare(basic_string_view(s, n2));}
\end{itemdescr}

\indexlibrarymember{starts_with}{basic_string_view}%
\begin{itemdecl}
constexpr bool starts_with(basic_string_view x) const noexcept;
\end{itemdecl}

\begin{itemdescr}
\pnum
令 \tcode{rlen} 为 \tcode{size()} 和 \tcode{x.size()} 的较小者。

\pnum
\effects
等价于: \tcode{return basic_string_view(data(), rlen) == x;}
\end{itemdescr}

\indexlibrarymember{starts_with}{basic_string_view}%
\begin{itemdecl}
constexpr bool starts_with(charT x) const noexcept;
\end{itemdecl}

\begin{itemdescr}
\pnum
\effects
等价于: \tcode{return !empty() \&\& traits::eq(front(), x);}
\end{itemdescr}

\indexlibrarymember{starts_with}{basic_string_view}%
\begin{itemdecl}
constexpr bool starts_with(const charT* x) const;
\end{itemdecl}

\begin{itemdescr}
\pnum
\effects
等价于: \tcode{return starts_with(basic_string_view(x));}
\end{itemdescr}

\indexlibrarymember{ends_with}{basic_string_view}%
\begin{itemdecl}
constexpr bool ends_with(basic_string_view x) const noexcept;
\end{itemdecl}

\begin{itemdescr}
\pnum
令 \tcode{rlen} 为 \tcode{size()} 和 \tcode{x.size()} 的较小者。

\pnum
\effects
效果等同于:
\begin{codeblock}
return basic_string_view(data() + (size() - rlen), rlen) == x;
\end{codeblock}
\end{itemdescr}

\indexlibrarymember{ends_with}{basic_string_view}%
\begin{itemdecl}
constexpr bool ends_with(charT x) const noexcept;
\end{itemdecl}

\begin{itemdescr}
\pnum
\effects
效果等同于:\tcode{return !empty() \&\& traits::eq(back(), x);}
\end{itemdescr}

\indexlibrarymember{ends_with}{basic_string_view}%
\begin{itemdecl}
constexpr bool ends_with(const charT* x) const;
\end{itemdecl}

\begin{itemdescr}
\pnum
\effects
效果等同于:\tcode{return ends_with(basic_string_view(x));}
\end{itemdescr}

\begin{itemdescr}
\pnum
\effects
等效于:\tcode{return ends_with(basic_string_view(x));}
\end{itemdescr}

\indexlibrarymember{contains}{basic_string_view}%
\begin{itemdecl}
constexpr bool contains(basic_string_view x) const noexcept;
constexpr bool contains(charT x) const noexcept;
constexpr bool contains(const charT* x) const;
\end{itemdecl}

\begin{itemdescr}
\pnum
\effects
等效于:\tcode{return find(x) != npos;}
\end{itemdescr}

\rSec3[string.view.find]{查找}
\pnum
本子条款中的成员函数最差情况下具有 \bigoh{\tcode{size() * str.size()}} 的复杂度,
但实现应力求更优。

\pnum
令 \placeholder{F} 为下列之一:
\tcode{find}、
\tcode{rfind}、
\tcode{find_first_of}、
\tcode{find_last_of}、
\tcode{find_first_not_of}、
与
\tcode{find_last_not_of}。
\begin{itemize}
\item
形式如下的每个成员函数
\begin{codeblock}
constexpr @\placeholder{return-type}@ @\placeholder{F}@(const charT* s, size_type pos) const;
\end{codeblock}
效果等价于:\tcode{return \placeholder{F}(basic_string_view(s), pos);}

\item
形式如下的每个成员函数
\begin{codeblock}
constexpr @\placeholder{return-type}@ @\placeholder{F}@(const charT* s, size_type pos, size_type n) const;
\end{codeblock}
效果等价于:\tcode{return \placeholder{F}(basic_string_view(s, n), pos);}

\item
形式如下的每个成员函数
\begin{codeblock}
constexpr @\placeholder{return-type}@ @\placeholder{F}@(charT c, size_type pos) const noexcept;
\end{codeblock}
效果等价于:\tcode{return \placeholder{F}(basic_string_view(addressof(c), 1), pos);}
\end{itemize}

\indexlibrarymember{find}{basic_string_view}%
\begin{itemdecl}
constexpr size_type find(basic_string_view str, size_type pos = 0) const noexcept;
\end{itemdecl}

\begin{itemdescr}
\pnum
令 \tcode{xpos} 为满足以下条件的最低位置(若可能):
\begin{itemize}
\item
\tcode{pos <= xpos}
\item
\tcode{xpos + str.size() <= size()}
\item
对于 \tcode{str} 所引用的字符串的所有元素 \tcode{I},\tcode{traits::eq(\exposid{data_}[xpos + I], str[I])} 成立。
\end{itemize}

\pnum
\effects
确定 \tcode{xpos}。

\pnum
\returns
若函数能为 \tcode{xpos} 确定这样的值,则返回 \tcode{xpos}。
否则,返回 \tcode{npos}。
\end{itemdescr}

\indexlibrarymember{rfind}{basic_string_view}%
\begin{itemdecl}
constexpr size_type rfind(basic_string_view str, size_type pos = npos) const noexcept;
\end{itemdecl}

\begin{itemdescr}
\pnum
令 \tcode{xpos} 为可能的最大位置,使得以下条件成立:
\begin{itemize}
\item
\tcode{xpos <= pos}
\item
\tcode{xpos + str.size() <= size()}
\item
对于 \tcode{str} 引用的字符串的所有元素 \tcode{I},\tcode{traits::eq(\exposid{data_}[xpos + I], str[I])}。
\end{itemize}

\pnum
\effects
确定 \tcode{xpos}。

\pnum
\returns
若该函数能为 \tcode{xpos} 确定这样一个值,则返回 \tcode{xpos}。
否则,返回 \tcode{npos}。
\end{itemdescr}

\indexlibrarymember{find_first_of}{basic_string_view}%
\begin{itemdecl}
constexpr size_type find_first_of(basic_string_view str, size_type pos = 0) const noexcept;
\end{itemdecl}

\begin{itemdescr}
\pnum
令 \tcode{xpos} 为尽可能低的位置,满足以下条件:
\begin{itemize}
\item
\tcode{pos <= xpos}
\item
\tcode{xpos < size()}
\item
\tcode{traits::eq(\exposid{data_}[xpos], str[I])} 对 \tcode{str} 所引用字符串的某个元素 \tcode{I} 成立。
\end{itemize}

\pnum
\effects
确定 \tcode{xpos}。

\pnum
\returns
若函数能确定这样的 \tcode{xpos} 值,则返回 \tcode{xpos}。
否则,返回 \tcode{npos}。
\end{itemdescr}

\indexlibrarymember{find_last_of}{basic_string_view}%
\begin{itemdecl}
constexpr size_type find_last_of(basic_string_view str, size_type pos = npos) const noexcept;
\end{itemdecl}

\begin{itemdescr}
\pnum
令 \tcode{xpos} 为可能情况下满足以下条件的最高位置:
\begin{itemize}
\item
\tcode{xpos <= pos}
\item
\tcode{xpos < size()}
\item
对 \tcode{str} 所引用字符串中的某个元素 \tcode{I},有 \tcode{traits::eq(\exposid{data_}[xpos], str[I])}。
\end{itemize}

\pnum
\effects
确定 \tcode{xpos}。

\pnum
\returns
若函数能确定这样的 \tcode{xpos},则返回 \tcode{xpos}。
否则,返回 \tcode{npos}。
\end{itemdescr}

\indexlibrarymember{find_first_not_of}{basic_string_view}%
\begin{itemdecl}
constexpr size_type find_first_not_of(basic_string_view str, size_type pos = 0) const noexcept;
\end{itemdecl}

\begin{itemdescr}
\pnum
令 \tcode{xpos} 为尽可能小的位置,若存在,使得以下条件均成立:
\begin{itemize}
\item
\tcode{pos <= xpos}
\item
\tcode{xpos < size()}
\item
对于 \tcode{str} 所引用的字符串中的任何元素 \tcode{I},\tcode{traits::eq(\exposid{data_}[xpos], str[I])} 均不成立。
\end{itemize}

\pnum
\effects
确定 \tcode{xpos}。

\pnum
\returns
若函数能为此 \tcode{xpos} 确定这样的值,则返回 \tcode{xpos}。否则返回 \tcode{npos}。
\end{itemdescr}

\indexlibrarymember{find_last_not_of}{basic_string_view}%
\begin{itemdecl}
constexpr size_type find_last_not_of(basic_string_view str, size_type pos = npos) const noexcept;
\end{itemdecl}

\begin{itemdescr}
\pnum
令 \tcode{xpos}(若可能)为满足下列条件的最高位置:
\begin{itemize}
\item
\tcode{xpos <= pos}
\item
\tcode{xpos < size()}
\item
对于 \tcode{str} 所引用的字符串的任意元素 \tcode{I},\tcode{traits::eq(\exposid{data_}[xpos], str[I])} 均不成立。
\end{itemize}

\pnum
\effects
确定 \tcode{xpos}。

\pnum
\returns
若函数能为 \tcode{xpos} 确定这样的值,则为 \tcode{xpos};否则为 \tcode{npos}。
\end{itemdescr}

\rSec2[string.view.comparison]{非成员比较函数}

\indexlibrarymember{operator==}{basic_string_view}%
\begin{itemdecl}
template<class charT, class traits>
  constexpr bool operator==(basic_string_view<charT, traits> lhs,
                            type_identity_t<basic_string_view<charT, traits>> rhs) noexcept;
\end{itemdecl}

\begin{itemdescr}
\pnum
\returns
\tcode{lhs.compare(rhs) == 0}。
\end{itemdescr}

\indexlibrarymember{operator<=>}{basic_string_view}%
\begin{itemdecl}
template<class charT, class traits>
  constexpr @\seebelow@ operator<=>(basic_string_view<charT, traits> lhs,
            @\itcorr@                      type_identity_t<basic_string_view<charT, traits>> rhs) noexcept;
\end{itemdecl}

\begin{itemdescr}
\pnum
令 \tcode{R} 表示类型 \tcode{traits::comparison_category},若该 \grammarterm{qualified-id} 合法并表示一个类型(见\iref{temp.deduct});否则 \tcode{R} 为 \tcode{weak_ordering}。

\pnum
\mandates
\tcode{R} 表示一个比较类别类型(见\iref{cmp.categories})。

\pnum
\returns
\tcode{static_cast<R>(lhs.compare(rhs) <=> 0)}。

\pnum
\begin{note}
将 \tcode{type_identity_t} 用作参数确保了类型为 \tcode{basic_string_view<charT, traits>} 的对象始终可以与能够隐式转换为 \tcode{basic_string_view<charT, traits>} 的类型 \tcode{T} 的对象进行比较,反之亦然,根据 \ref{over.match.oper} 的规定。
\end{note}
\end{itemdescr}

\rSec2[string.view.io]{插入器与提取器}

\indexlibrarymember{operator<<}{basic_string_view}%
\begin{itemdecl}
template<class charT, class traits>
  basic_ostream<charT, traits>&
    operator<<(basic_ostream<charT, traits>& os, basic_string_view<charT, traits> str);
\end{itemdecl}

\begin{itemdescr}
\pnum
\effects
如 \tcode{os} 的带格式输出函数\iref{ostream.formatted.reqmts} 般运作。
构成一个字符序列 \tcode{seq},初始时由范围
\range{str.begin()}{str.end()} 所定义的元素组成。
如~\ref{ostream.formatted.reqmts} 所述决定 \tcode{seq} 的填充。
然后如同通过调用 \tcode{os.rdbuf()->sputn(\brk{}seq, n)} 来插入 \tcode{seq},
其中 \tcode{n} 是 \tcode{os.width()} 与 \tcode{str.size()} 的较大者;
再调用 \tcode{os.\brk{}width(0)}。

\pnum
\returns
\tcode{os}。
\end{itemdescr}

\rSec2[string.view.hash]{哈希支持}

\indexlibrarymember{hash}{string_view}%
\indexlibrarymember{hash}{u8string_view}%
\indexlibrarymember{hash}{u16string_view}%
\indexlibrarymember{hash}{u32string_view}%
\indexlibrarymember{hash}{wstring_view}%
\begin{itemdecl}
template<> struct hash<string_view>;
template<> struct hash<u8string_view>;
template<> struct hash<u16string_view>;
template<> struct hash<u32string_view>;
template<> struct hash<wstring_view>;
\end{itemdecl}

\begin{itemdescr}
\pnum
该特化被启用\iref{unord.hash}。
\begin{note}
字符串视图对象的哈希值等于对应的字符串对象的哈希值\iref{basic.string.hash}。
\end{note}
\end{itemdescr}

\rSec2[string.view.literals]{\tcode{basic_string_view} 字面量的后缀}

\indexlibrarymember{operator""sv}{string_view}%
\begin{itemdecl}
constexpr string_view operator""sv(const char* str, size_t len) noexcept;
\end{itemdecl}

\begin{itemdescr}
\pnum
\returns
\tcode{string_view\{str, len\}}.
\end{itemdescr}

\indexlibrarymember{operator""sv}{u8string_view}%
\begin{itemdecl}
constexpr u8string_view operator""sv(const char8_t* str, size_t len) noexcept;
\end{itemdecl}
\begin{itemdescr}
\pnum
\returns
\tcode{u8string_view\{str, len\}}.
\end{itemdescr}

\indexlibrarymember{operator""sv}{u16string_view}%
\begin{itemdecl}
constexpr u16string_view operator""sv(const char16_t* str, size_t len) noexcept;
\end{itemdecl}
\begin{itemdescr}
\pnum
\returns
\tcode{u16string_view\{str, len\}}.
\end{itemdescr}

\indexlibrarymember{operator""sv}{u32string_view}%
\begin{itemdecl}
constexpr u32string_view operator""sv(const char32_t* str, size_t len) noexcept;
\end{itemdecl}
\begin{itemdescr}
\pnum
\returns
\tcode{u32string_view\{str, len\}}.
\end{itemdescr}

\indexlibrarymember{operator""sv}{wstring_view}%
\begin{itemdecl}
constexpr wstring_view operator""sv(const wchar_t* str, size_t len) noexcept;
\end{itemdecl}
\begin{itemdescr}
\pnum
\returns
\tcode{wstring_view\{str, len\}}.
\end{itemdescr}

\rSec1[string.classes]{字符串类}

\rSec2[string.classes.general]{概览}

\pnum
头文件 \tcode{<string>} 定义了
用于操纵可变长度的类字符对象序列的类模板 \tcode{basic_string}
以及五个\grammarterm{typedef-name}{s}:\tcode{string}、
\tcode{u8string}、
\tcode{u16string}、
\tcode{u32string} 和
\tcode{wstring},它们分别指名了特化
\tcode{basic_string<char>}、
\tcode{basic_string<char8_t>}、
\tcode{basic_string<char16_t>}、
\tcode{basic_string<char32_t>} 以及
\tcode{basic_string<\brk{}wchar_t>}。

\rSec2[string.syn]{头文件 \tcode{<string>} 概要}
\indexheader{string}%

\begin{codeblock}
#include <compare>              // 参见 \ref{compare.syn}
#include <initializer_list>     // 参见 \ref{initializer.list.syn}

namespace std {
  // \ref{char.traits},字符特征
  template<class charT> struct char_traits;                             // 独立实现
  template<> struct char_traits<char>;                                  // 独立实现
  template<> struct char_traits<char8_t>;                               // 独立实现
  template<> struct char_traits<char16_t>;                              // 独立实现
  template<> struct char_traits<char32_t>;                              // 独立实现
  template<> struct char_traits<wchar_t>;                               // 独立实现

  // \ref{basic.string},\tcode{basic_string}
  template<class charT, class traits = char_traits<charT>, class Allocator = allocator<charT>>
    class basic_string;

  template<class charT, class traits, class Allocator>
    constexpr basic_string<charT, traits, Allocator>
      operator+(const basic_string<charT, traits, Allocator>& lhs,
                const basic_string<charT, traits, Allocator>& rhs);
  template<class charT, class traits, class Allocator>
    constexpr basic_string<charT, traits, Allocator>
      operator+(basic_string<charT, traits, Allocator>&& lhs,
                const basic_string<charT, traits, Allocator>& rhs);
  template<class charT, class traits, class Allocator>
    constexpr basic_string<charT, traits, Allocator>
      operator+(const basic_string<charT, traits, Allocator>& lhs,
                basic_string<charT, traits, Allocator>&& rhs);
  template<class charT, class traits, class Allocator>
    constexpr basic_string<charT, traits, Allocator>
      operator+(basic_string<charT, traits, Allocator>&& lhs,
                basic_string<charT, traits, Allocator>&& rhs);
  template<class charT, class traits, class Allocator>
    constexpr basic_string<charT, traits, Allocator>
      operator+(const charT* lhs,
                const basic_string<charT, traits, Allocator>& rhs);
  template<class charT, class traits, class Allocator>
    constexpr basic_string<charT, traits, Allocator>
      operator+(const charT* lhs,
                basic_string<charT, traits, Allocator>&& rhs);
  template<class charT, class traits, class Allocator>
    constexpr basic_string<charT, traits, Allocator>
      operator+(charT lhs,
                const basic_string<charT, traits, Allocator>& rhs);
  template<class charT, class traits, class Allocator>
    constexpr basic_string<charT, traits, Allocator>
      operator+(charT lhs,
                basic_string<charT, traits, Allocator>&& rhs);
  template<class charT, class traits, class Allocator>
    constexpr basic_string<charT, traits, Allocator>
      operator+(const basic_string<charT, traits, Allocator>& lhs,
                const charT* rhs);
  template<class charT, class traits, class Allocator>
    constexpr basic_string<charT, traits, Allocator>
      operator+(basic_string<charT, traits, Allocator>&& lhs,
                const charT* rhs);
  template<class charT, class traits, class Allocator>
    constexpr basic_string<charT, traits, Allocator>
      operator+(const basic_string<charT, traits, Allocator>& lhs,
                charT rhs);
  template<class charT, class traits, class Allocator>
    constexpr basic_string<charT, traits, Allocator>
      operator+(basic_string<charT, traits, Allocator>&& lhs,
                charT rhs);
  template<class charT, class traits, class Allocator>
    constexpr basic_string<charT, traits, Allocator>
      operator+(const basic_string<charT, traits, Allocator>& lhs,
                type_identity_t<basic_string_view<charT, traits>> rhs);
  template<class charT, class traits, class Allocator>
    constexpr basic_string<charT, traits, Allocator>
      operator+(basic_string<charT, traits, Allocator>&& lhs,
                type_identity_t<basic_string_view<charT, traits>> rhs);
  template<class charT, class traits, class Allocator>
    constexpr basic_string<charT, traits, Allocator>
      operator+(type_identity_t<basic_string_view<charT, traits>> lhs,
                const basic_string<charT, traits, Allocator>& rhs);
  template<class charT, class traits, class Allocator>
    constexpr basic_string<charT, traits, Allocator>
      operator+(type_identity_t<basic_string_view<charT, traits>> lhs,
                basic_string<charT, traits, Allocator>&& rhs);

  template<class charT, class traits, class Allocator>
    constexpr bool
      operator==(const basic_string<charT, traits, Allocator>& lhs,
                 const basic_string<charT, traits, Allocator>& rhs) noexcept;
  template<class charT, class traits, class Allocator>
    constexpr bool operator==(const basic_string<charT, traits, Allocator>& lhs,
                              const charT* rhs);

  template<class charT, class traits, class Allocator>
    constexpr @\seebelow@ operator<=>(const basic_string<charT, traits, Allocator>& lhs,
              @\itcorr@                      const basic_string<charT, traits, Allocator>& rhs) noexcept;
  template<class charT, class traits, class Allocator>
    constexpr @\seebelow@ operator<=>(const basic_string<charT, traits, Allocator>& lhs,
              @\itcorr@                      const charT* rhs);

  // \ref{string.special},交换
  template<class charT, class traits, class Allocator>
    constexpr void
      swap(basic_string<charT, traits, Allocator>& lhs,
           basic_string<charT, traits, Allocator>& rhs)
        noexcept(noexcept(lhs.swap(rhs)));

  // \ref{string.io},插入器与提取器
  template<class charT, class traits, class Allocator>
    basic_istream<charT, traits>&
      operator>>(basic_istream<charT, traits>& is,
                 basic_string<charT, traits, Allocator>& str);
  template<class charT, class traits, class Allocator>
    basic_ostream<charT, traits>&
      operator<<(basic_ostream<charT, traits>& os,
                 const basic_string<charT, traits, Allocator>& str);
  template<class charT, class traits, class Allocator>
    basic_istream<charT, traits>&
      getline(basic_istream<charT, traits>& is,
              basic_string<charT, traits, Allocator>& str,
              charT delim);
  template<class charT, class traits, class Allocator>
    basic_istream<charT, traits>&
      getline(basic_istream<charT, traits>&& is,
              basic_string<charT, traits, Allocator>& str,
              charT delim);
  template<class charT, class traits, class Allocator>
    basic_istream<charT, traits>&
      getline(basic_istream<charT, traits>& is,
              basic_string<charT, traits, Allocator>& str);
  template<class charT, class traits, class Allocator>
    basic_istream<charT, traits>&
      getline(basic_istream<charT, traits>&& is,
              basic_string<charT, traits, Allocator>& str);

  // \ref{string.erasure},擦除
  template<class charT, class traits, class Allocator, class U = charT>
    constexpr typename basic_string<charT, traits, Allocator>::size_type
      erase(basic_string<charT, traits, Allocator>& c, const U& value);
  template<class charT, class traits, class Allocator, class Predicate>
    constexpr typename basic_string<charT, traits, Allocator>::size_type
      erase_if(basic_string<charT, traits, Allocator>& c, Predicate pred);

  // \tcode{basic_string} \grammarterm{typedef-name}s
  using @\libglobal{string}@    = basic_string<char>;
  using @\libglobal{u8string}@  = basic_string<char8_t>;
  using @\libglobal{u16string}@ = basic_string<char16_t>;
  using @\libglobal{u32string}@ = basic_string<char32_t>;
  using @\libglobal{wstring}@   = basic_string<wchar_t>;

  // \ref{string.conversions},数值转换
  int stoi(const string& str, size_t* idx = nullptr, int base = 10);
  long stol(const string& str, size_t* idx = nullptr, int base = 10);
  unsigned long stoul(const string& str, size_t* idx = nullptr, int base = 10);
  long long stoll(const string& str, size_t* idx = nullptr, int base = 10);
  unsigned long long stoull(const string& str, size_t* idx = nullptr, int base = 10);
  float stof(const string& str, size_t* idx = nullptr);
  double stod(const string& str, size_t* idx = nullptr);
  long double stold(const string& str, size_t* idx = nullptr);
  string to_string(int val);
  string to_string(unsigned val);
  string to_string(long val);
  string to_string(unsigned long val);
  string to_string(long long val);
  string to_string(unsigned long long val);
  string to_string(float val);
  string to_string(double val);
  string to_string(long double val);

  int stoi(const wstring& str, size_t* idx = nullptr, int base = 10);
  long stol(const wstring& str, size_t* idx = nullptr, int base = 10);
  unsigned long stoul(const wstring& str, size_t* idx = nullptr, int base = 10);
  long long stoll(const wstring& str, size_t* idx = nullptr, int base = 10);
  unsigned long long stoull(const wstring& str, size_t* idx = nullptr, int base = 10);
  float stof(const wstring& str, size_t* idx = nullptr);
  double stod(const wstring& str, size_t* idx = nullptr);
  long double stold(const wstring& str, size_t* idx = nullptr);
  wstring to_wstring(int val);
  wstring to_wstring(unsigned val);
  wstring to_wstring(long val);
  wstring to_wstring(unsigned long val);
  wstring to_wstring(long long val);
  wstring to_wstring(unsigned long long val);
  wstring to_wstring(float val);
  wstring to_wstring(double val);
  wstring to_wstring(long double val);

  namespace pmr {
    template<class charT, class traits = char_traits<charT>>
      using basic_string = std::basic_string<charT, traits, polymorphic_allocator<charT>>;

    using string    = basic_string<char>;
    using u8string  = basic_string<char8_t>;
    using u16string = basic_string<char16_t>;
    using u32string = basic_string<char32_t>;
    using wstring   = basic_string<wchar_t>;
  }

  // \ref{basic.string.hash},散列支持
  template<class T> struct hash;
  template<class A> struct hash<basic_string<char, char_traits<char>, A>>;
  template<class A> struct hash<basic_string<char8_t, char_traits<char8_t>, A>>;
  template<class A> struct hash<basic_string<char16_t, char_traits<char16_t>, A>>;
  template<class A> struct hash<basic_string<char32_t, char_traits<char32_t>, A>>;
  template<class A> struct hash<basic_string<wchar_t, char_traits<wchar_t>, A>>;

  inline namespace literals {
    inline namespace string_literals {
      // \ref{basic.string.literals},\tcode{basic_string} 字面量后缀
      constexpr string    operator""s(const char* str, size_t len);
      constexpr u8string  operator""s(const char8_t* str, size_t len);
      constexpr u16string operator""s(const char16_t* str, size_t len);
      constexpr u32string operator""s(const char32_t* str, size_t len);
      constexpr wstring   operator""s(const wchar_t* str, size_t len);
    }
  }
}
\end{codeblock}\rSec2[basic.string]{类模板 \tcode{basic_string}}

\rSec3[basic.string.general]{概述}
\pnum
\indexlibraryglobal{basic_string}%
类模板
\tcode{basic_string}
描述的对象能够存储一个由数量可变的任意类字符对象组成的序列,该序列的第一个元素位于位置零。
若其所持有的类字符对象的类型从上下文中是清晰的,
这样的序列也称为一个“字符串”。
在\ref{basic.string} 的其余部分,
存储在 \tcode{basic_string} 对象中的类字符对象的类型
由 \tcode{charT} 指代。

\pnum
\tcode{basic_string} 的特化是连续容器\iref{container.reqmts}。


\pnum
In all cases,
\crange{data()}{data() + size()} is a valid range,
\tcode{data() + size()} points at an object with value \tcode{charT()}
(a ``null terminator''\indextext{string!null terminator}),
and \tcode{size() <= capacity()} is \tcode{true}.


\indexlibraryglobal{basic_string}%
\indexlibrarymember{traits_type}{basic_string}%
\indexlibrarymember{value_type}{basic_string}%
\indexlibrarymember{allocator_type}{basic_string}%
\indexlibrarymember{size_type}{basic_string}%
\indexlibrarymember{difference_type}{basic_string}%
\indexlibrarymember{pointer}{basic_string}%
\indexlibrarymember{const_pointer}{basic_string}%
\indexlibrarymember{reference}{basic_string}%
\indexlibrarymember{const_reference}{basic_string}%
\indexlibrarymember{iterator}{basic_string}%
\indexlibrarymember{const_iterator}{basic_string}%
\indexlibrarymember{reverse_iterator}{basic_string}%
\indexlibrarymember{const_reverse_iterator}{basic_string}%
\begin{codeblock}
namespace std {
  template<class charT, class traits = char_traits<charT>,
           class Allocator = allocator<charT>>
  class basic_string {
  public:
    // types
    using traits_type            = traits;
    using value_type             = charT;
    using allocator_type         = Allocator;
    using size_type              = allocator_traits<Allocator>::size_type;
    using difference_type        = allocator_traits<Allocator>::difference_type;
    using pointer                = allocator_traits<Allocator>::pointer;
    using const_pointer          = allocator_traits<Allocator>::const_pointer;
    using reference              = value_type&;
    using const_reference        = const value_type&;

    using iterator               = @\impdefx{type of \tcode{basic_string::iterator}}@; // see \ref{container.requirements}
    using const_iterator         = @\impdefx{type of \tcode{basic_string::const_iterator}}@; // see \ref{container.requirements}
    using reverse_iterator       = std::reverse_iterator<iterator>;
    using const_reverse_iterator = std::reverse_iterator<const_iterator>;
    static constexpr size_type npos = size_type(-1);

    // \ref{string.cons}, construct/copy/destroy
    constexpr basic_string() noexcept(noexcept(Allocator())) : basic_string(Allocator()) { }
    constexpr explicit basic_string(const Allocator& a) noexcept;
    constexpr basic_string(const basic_string& str);
    constexpr basic_string(basic_string&& str) noexcept;
    constexpr basic_string(const basic_string& str, size_type pos,
                           const Allocator& a = Allocator());
    constexpr basic_string(const basic_string& str, size_type pos, size_type n,
                           const Allocator& a = Allocator());
    constexpr basic_string(basic_string&& str, size_type pos,
                           const Allocator& a = Allocator());
    constexpr basic_string(basic_string&& str, size_type pos, size_type n,
                           const Allocator& a = Allocator());
    template<class T>
      constexpr basic_string(const T& t, size_type pos, size_type n,
                             const Allocator& a = Allocator());
    template<class T>
      constexpr explicit basic_string(const T& t, const Allocator& a = Allocator());
    constexpr basic_string(const charT* s, size_type n, const Allocator& a = Allocator());
    constexpr basic_string(const charT* s, const Allocator& a = Allocator());
    basic_string(nullptr_t) = delete;
    constexpr basic_string(size_type n, charT c, const Allocator& a = Allocator());
    template<class InputIterator>
      constexpr basic_string(InputIterator begin, InputIterator end,
                             const Allocator& a = Allocator());
    template<@\exposconcept{container-compatible-range}@<charT> R>
      constexpr basic_string(from_range_t, R&& rg, const Allocator& a = Allocator());
    constexpr basic_string(initializer_list<charT>, const Allocator& = Allocator());
    constexpr basic_string(const basic_string&, const Allocator&);
    constexpr basic_string(basic_string&&, const Allocator&);
    constexpr ~basic_string();

    constexpr basic_string& operator=(const basic_string& str);
    constexpr basic_string& operator=(basic_string&& str)
      noexcept(allocator_traits<Allocator>::propagate_on_container_move_assignment::value ||
               allocator_traits<Allocator>::is_always_equal::value);
    template<class T>
      constexpr basic_string& operator=(const T& t);
    constexpr basic_string& operator=(const charT* s);
    basic_string& operator=(nullptr_t) = delete;
    constexpr basic_string& operator=(charT c);
    constexpr basic_string& operator=(initializer_list<charT>);

    // \ref{string.iterators}, iterators
    constexpr iterator       begin() noexcept;
    constexpr const_iterator begin() const noexcept;
    constexpr iterator       end() noexcept;
    constexpr const_iterator end() const noexcept;

    constexpr reverse_iterator       rbegin() noexcept;
    constexpr const_reverse_iterator rbegin() const noexcept;
    constexpr reverse_iterator       rend() noexcept;
    constexpr const_reverse_iterator rend() const noexcept;

    constexpr const_iterator         cbegin() const noexcept;
    constexpr const_iterator         cend() const noexcept;
    constexpr const_reverse_iterator crbegin() const noexcept;
    constexpr const_reverse_iterator crend() const noexcept;

    // \ref{string.capacity}, capacity
    constexpr size_type size() const noexcept;
    constexpr size_type length() const noexcept;
    constexpr size_type max_size() const noexcept;
    constexpr void resize(size_type n, charT c);
    constexpr void resize(size_type n);
    template<class Operation> constexpr void resize_and_overwrite(size_type n, Operation op);
    constexpr size_type capacity() const noexcept;
    constexpr void reserve(size_type res_arg);
    constexpr void shrink_to_fit();
    constexpr void clear() noexcept;
    constexpr bool empty() const noexcept;

    // \ref{string.access}, element access
    constexpr const_reference operator[](size_type pos) const;
    constexpr reference       operator[](size_type pos);
    constexpr const_reference at(size_type n) const;
    constexpr reference       at(size_type n);

    constexpr const charT& front() const;
    constexpr charT&       front();
    constexpr const charT& back() const;
    constexpr charT&       back();

    // \ref{string.modifiers}, modifiers
    constexpr basic_string& operator+=(const basic_string& str);
    template<class T>
      constexpr basic_string& operator+=(const T& t);
    constexpr basic_string& operator+=(const charT* s);
    constexpr basic_string& operator+=(charT c);
    constexpr basic_string& operator+=(initializer_list<charT>);
    constexpr basic_string& append(const basic_string& str);
    constexpr basic_string& append(const basic_string& str, size_type pos, size_type n = npos);
    template<class T>
      constexpr basic_string& append(const T& t);
    template<class T>
      constexpr basic_string& append(const T& t, size_type pos, size_type n = npos);
    constexpr basic_string& append(const charT* s, size_type n);
    constexpr basic_string& append(const charT* s);
    constexpr basic_string& append(size_type n, charT c);
    template<class InputIterator>
      constexpr basic_string& append(InputIterator first, InputIterator last);
    template<@\exposconcept{container-compatible-range}@<charT> R>
      constexpr basic_string& append_range(R&& rg);
    constexpr basic_string& append(initializer_list<charT>);

    constexpr void push_back(charT c);

    constexpr basic_string& assign(const basic_string& str);
    constexpr basic_string& assign(basic_string&& str)
      noexcept(allocator_traits<Allocator>::propagate_on_container_move_assignment::value ||
               allocator_traits<Allocator>::is_always_equal::value);
    constexpr basic_string& assign(const basic_string& str, size_type pos, size_type n = npos);
    template<class T>
      constexpr basic_string& assign(const T& t);
    template<class T>
      constexpr basic_string& assign(const T& t, size_type pos, size_type n = npos);
    constexpr basic_string& assign(const charT* s, size_type n);
    constexpr basic_string& assign(const charT* s);
    constexpr basic_string& assign(size_type n, charT c);
    template<class InputIterator>
      constexpr basic_string& assign(InputIterator first, InputIterator last);
    template<@\exposconcept{container-compatible-range}@<charT> R>
      constexpr basic_string& assign_range(R&& rg);
    constexpr basic_string& assign(initializer_list<charT>);

    constexpr basic_string& insert(size_type pos, const basic_string& str);
    constexpr basic_string& insert(size_type pos1, const basic_string& str,
                                   size_type pos2, size_type n = npos);
    template<class T>
      constexpr basic_string& insert(size_type pos, const T& t);
    template<class T>
      constexpr basic_string& insert(size_type pos1, const T& t,
                                     size_type pos2, size_type n = npos);
    constexpr basic_string& insert(size_type pos, const charT* s, size_type n);
    constexpr basic_string& insert(size_type pos, const charT* s);
    constexpr basic_string& insert(size_type pos, size_type n, charT c);
    constexpr iterator insert(const_iterator p, charT c);
    constexpr iterator insert(const_iterator p, size_type n, charT c);
    template<class InputIterator>
      constexpr iterator insert(const_iterator p, InputIterator first, InputIterator last);
    template<@\exposconcept{container-compatible-range}@<charT> R>
      constexpr iterator insert_range(const_iterator p, R&& rg);
    constexpr iterator insert(const_iterator p, initializer_list<charT>);

    constexpr basic_string& erase(size_type pos = 0, size_type n = npos);
    constexpr iterator erase(const_iterator p);
    constexpr iterator erase(const_iterator first, const_iterator last);

    constexpr void pop_back();

    constexpr basic_string& replace(size_type pos1, size_type n1, const basic_string& str);
    constexpr basic_string& replace(size_type pos1, size_type n1, const basic_string& str,
                                    size_type pos2, size_type n2 = npos);
    template<class T>
      constexpr basic_string& replace(size_type pos1, size_type n1, const T& t);
    template<class T>
      constexpr basic_string& replace(size_type pos1, size_type n1, const T& t,
                                      size_type pos2, size_type n2 = npos);
    constexpr basic_string& replace(size_type pos, size_type n1, const charT* s, size_type n2);
    constexpr basic_string& replace(size_type pos, size_type n1, const charT* s);
    constexpr basic_string& replace(size_type pos, size_type n1, size_type n2, charT c);
    constexpr basic_string& replace(const_iterator i1, const_iterator i2,
                                    const basic_string& str);
    template<class T>
      constexpr basic_string& replace(const_iterator i1, const_iterator i2, const T& t);
    constexpr basic_string& replace(const_iterator i1, const_iterator i2, const charT* s,
                                    size_type n);
    constexpr basic_string& replace(const_iterator i1, const_iterator i2, const charT* s);
    constexpr basic_string& replace(const_iterator i1, const_iterator i2, size_type n, charT c);
    template<class InputIterator>
      constexpr basic_string& replace(const_iterator i1, const_iterator i2,
                                      InputIterator j1, InputIterator j2);
    template<@\exposconcept{container-compatible-range}@<charT> R>
      constexpr basic_string& replace_with_range(const_iterator i1, const_iterator i2, R&& rg);
    constexpr basic_string& replace(const_iterator, const_iterator, initializer_list<charT>);

    constexpr size_type copy(charT* s, size_type n, size_type pos = 0) const;

    constexpr void swap(basic_string& str)
      noexcept(allocator_traits<Allocator>::propagate_on_container_swap::value ||
               allocator_traits<Allocator>::is_always_equal::value);

    // \ref{string.ops}, string operations
    constexpr const charT* c_str() const noexcept;
    constexpr const charT* data() const noexcept;
    constexpr charT* data() noexcept;
    constexpr operator basic_string_view<charT, traits>() const noexcept;
    constexpr allocator_type get_allocator() const noexcept;

    template<class T>
      constexpr size_type find(const T& t, size_type pos = 0) const noexcept(@\seebelow@);
    constexpr size_type find(const basic_string& str, size_type pos = 0) const noexcept;
    constexpr size_type find(const charT* s, size_type pos, size_type n) const;
    constexpr size_type find(const charT* s, size_type pos = 0) const;
    constexpr size_type find(charT c, size_type pos = 0) const noexcept;
    template<class T>
      constexpr size_type rfind(const T& t, size_type pos = npos) const noexcept(@\seebelow@);
    constexpr size_type rfind(const basic_string& str, size_type pos = npos) const noexcept;
    constexpr size_type rfind(const charT* s, size_type pos, size_type n) const;
    constexpr size_type rfind(const charT* s, size_type pos = npos) const;
    constexpr size_type rfind(charT c, size_type pos = npos) const noexcept;

    template<class T>
      constexpr size_type find_first_of(const T& t, size_type pos = 0) const noexcept(@\seebelow@);
    constexpr size_type find_first_of(const basic_string& str, size_type pos = 0) const noexcept;
    constexpr size_type find_first_of(const charT* s, size_type pos, size_type n) const;
    constexpr size_type find_first_of(const charT* s, size_type pos = 0) const;
    constexpr size_type find_first_of(charT c, size_type pos = 0) const noexcept;
    template<class T>
      constexpr size_type find_last_of(const T& t,
                                       size_type pos = npos) const noexcept(@\seebelow@);
    constexpr size_type find_last_of(const basic_string& str,
                                     size_type pos = npos) const noexcept;
    constexpr size_type find_last_of(const charT* s, size_type pos, size_type n) const;
    constexpr size_type find_last_of(const charT* s, size_type pos = npos) const;
    constexpr size_type find_last_of(charT c, size_type pos = npos) const noexcept;

    template<class T>
      constexpr size_type find_first_not_of(const T& t,
                                            size_type pos = 0) const noexcept(@\seebelow@);
    constexpr size_type find_first_not_of(const basic_string& str,
                                          size_type pos = 0) const noexcept;
    constexpr size_type find_first_not_of(const charT* s, size_type pos, size_type n) const;
    constexpr size_type find_first_not_of(const charT* s, size_type pos = 0) const;
    constexpr size_type find_first_not_of(charT c, size_type pos = 0) const noexcept;
    template<class T>
      constexpr size_type find_last_not_of(const T& t,
                                           size_type pos = npos) const noexcept(@\seebelow@);
    constexpr size_type find_last_not_of(const basic_string& str,
                                         size_type pos = npos) const noexcept;
    constexpr size_type find_last_not_of(const charT* s, size_type pos, size_type n) const;
    constexpr size_type find_last_not_of(const charT* s, size_type pos = npos) const;
    constexpr size_type find_last_not_of(charT c, size_type pos = npos) const noexcept;

    constexpr basic_string substr(size_type pos = 0, size_type n = npos) const &;
    constexpr basic_string substr(size_type pos = 0, size_type n = npos) &&;
    constexpr basic_string_view<charT, traits> subview(size_type pos = 0,
                                                       size_type n = npos) const;

    template<class T>
      constexpr int compare(const T& t) const noexcept(@\seebelow@);
    template<class T>
      constexpr int compare(size_type pos1, size_type n1, const T& t) const;
    template<class T>
      constexpr int compare(size_type pos1, size_type n1, const T& t,
                            size_type pos2, size_type n2 = npos) const;
    constexpr int compare(const basic_string& str) const noexcept;
    constexpr int compare(size_type pos1, size_type n1, const basic_string& str) const;
    constexpr int compare(size_type pos1, size_type n1, const basic_string& str,
                          size_type pos2, size_type n2 = npos) const;
    constexpr int compare(const charT* s) const;
    constexpr int compare(size_type pos1, size_type n1, const charT* s) const;
    constexpr int compare(size_type pos1, size_type n1, const charT* s, size_type n2) const;

    constexpr bool starts_with(basic_string_view<charT, traits> x) const noexcept;
    constexpr bool starts_with(charT x) const noexcept;
    constexpr bool starts_with(const charT* x) const;
    constexpr bool ends_with(basic_string_view<charT, traits> x) const noexcept;
    constexpr bool ends_with(charT x) const noexcept;
    constexpr bool ends_with(const charT* x) const;

    constexpr bool contains(basic_string_view<charT, traits> x) const noexcept;
    constexpr bool contains(charT x) const noexcept;
    constexpr bool contains(const charT* x) const;
  };

  template<class InputIterator,
           class Allocator = allocator<typename iterator_traits<InputIterator>::value_type>>
    basic_string(InputIterator, InputIterator, Allocator = Allocator())
      -> basic_string<typename iterator_traits<InputIterator>::value_type,
                      char_traits<typename iterator_traits<InputIterator>::value_type>,
                      Allocator>;

  template<ranges::@\libconcept{input_range}@ R,
           class Allocator = allocator<ranges::range_value_t<R>>>
    basic_string(from_range_t, R&&, Allocator = Allocator())
      -> basic_string<ranges::range_value_t<R>, char_traits<ranges::range_value_t<R>>,
                      Allocator>;

  template<class charT,
           class traits,
           class Allocator = allocator<charT>>
    explicit basic_string(basic_string_view<charT, traits>, const Allocator& = Allocator())
      -> basic_string<charT, traits, Allocator>;

  template<class charT,
           class traits,
           class Allocator = allocator<charT>>
    basic_string(basic_string_view<charT, traits>,
                 typename @\seebelow@::size_type, typename @\seebelow@::size_type,
                 const Allocator& = Allocator())
      -> basic_string<charT, traits, Allocator>;
}
\end{codeblock}\pnum
在 \tcode{basic_string} 的推导指引中,一个 \tcode{size_type} 参数类型指的是该推导指引所推导类型的 \tcode{size_type} 成员类型。
\pnum
类型 \tcode{iterator} 和 \tcode{const_iterator} 满足 constexpr 迭代器要求\iref{iterator.requirements.general}。

\rSec3[string.require]{通用要求}
\pnum
若任何操作会导致 \tcode{size()} 超过 \tcode{max_size()},则该操作抛出一个类型为 \tcode{length_error} 的异常对象。
\pnum
若 \tcode{basic_string} 的任何成员函数或运算符抛出异常,则该函数或运算符对该 \tcode{basic_string} 对象无其他效果。

\pnum
对于类型
\tcode{basic_string<charT, traits, Allocator>}的每个对象,
都使用一个类型为
\tcode{Allocator}的对象来根据需要为所包含的\tcode{charT}
对象分配和释放存储。所用的\tcode{Allocator}对象是
通过\ref{container.reqmts}中描述的方式获得的。
在每个特化\tcode{basic_string<charT, traits, Allocator>}中,
类型\tcode{traits}应满足
字符特性要求\iref{char.traits}。
\begin{note}
每个特化\tcode{basic_string<charT, traits, Allocator>}都是
一个分配器感知容器\iref{container.alloc.reqmts},
但不使用分配器的\tcode{construct}和\tcode{destroy}
成员函数\iref{container.requirements.pre}。
若
\tcode{Allocator::value_type}与\tcode{charT}不是同一类型,
则程序是非良构的。
\end{note}

\begin{note}
若 \tcode{traits::char_type} 与 \tcode{charT} 不是同一类型,则程序为非良构。
\end{note}

\pnum
引用、指向或迭代指向 \tcode{basic_string} 序列元素的引用、指针和迭代器,
可能因以下对该 \tcode{basic_string} 对象的使用而失效:

\begin{itemize}
\item 作为参数传递给任何以非 const \tcode{basic_string} 的引用为参数的标准库函数。
\begin{footnote}
例如,作为非成员函数 \tcode{swap()}\iref{string.special}、
\tcode{operator>{}>()}\iref{string.io} 和 \tcode{getline()}\iref{string.io} 的参数,
或作为 \tcode{basic_string::swap()} 的参数。
\end{footnote}

\item 调用非 const 成员函数,除了
\tcode{operator[]}、
\tcode{at}、
\tcode{data}、
\tcode{front}、
\tcode{back}、
\tcode{begin}、
\tcode{rbegin}、
\tcode{end}、
和
\tcode{rend} 。
\end{itemize}

\rSec3[string.cons]{构造函数与赋值运算符}

\indexlibraryctor{basic_string}%
\begin{itemdecl}
constexpr explicit basic_string(const Allocator& a) noexcept;
\end{itemdecl}

\begin{itemdescr}

\pnum
\ensures
\tcode{size()} 等于 \tcode{0}。
\end{itemdescr}

\indexlibraryctor{basic_string}%
\begin{itemdecl}
constexpr basic_string(const basic_string& str);
constexpr basic_string(basic_string&& str) noexcept;
\end{itemdecl}

\begin{itemdescr}
\pnum
\effects
构造一个对象,其
值为调用前 \tcode{str} 的值。

\pnum
\remarks
第二种形式中,\tcode{str} 处于有效但未指定的状态。
\end{itemdescr}

\indexlibraryctor{basic_string}%
\begin{itemdecl}
constexpr basic_string(const basic_string& str, size_type pos,
                       const Allocator& a = Allocator());
constexpr basic_string(const basic_string& str, size_type pos, size_type n,
                       const Allocator& a = Allocator());
constexpr basic_string(basic_string&& str, size_type pos,
                       const Allocator& a = Allocator());
constexpr basic_string(basic_string&& str, size_type pos, size_type n,
                       const Allocator& a = Allocator());
\end{itemdecl}

\begin{itemdescr}
\pnum
令
\begin{itemize}
\item
\tcode{s} 为此次调用前的 \tcode{str} 的值,且
\item
对于带参数 \tcode{n} 的重载,令 \tcode{rlen} 为 \tcode{pos + min(n, s.size() - pos)},
否则为 \tcode{s.size()}。
\end{itemize}

\pnum
\effects
构造一个对象,其初始值为范围 \range{s.data() + pos}{s.data() + rlen}。

\pnum
\throws
若 \tcode{pos > s.size()} 则抛出 \tcode{out_of_range}。

\pnum
\remarks
对于带 \tcode{basic_string\&\&} 参数的重载,\tcode{str} 将处于有效但未指定的状态。

\pnum
\recommended
对于带 \tcode{basic_string\&\&} 参数的重载,若 \tcode{s.get_allocator() == a} 为 \tcode{true},则实现应避免分配。
\end{itemdescr}

\indexlibraryctor{basic_string}%
\begin{itemdecl}
template<class T>
  constexpr basic_string(const T& t, size_type pos, size_type n, const Allocator& a = Allocator());
\end{itemdecl}

\begin{itemdescr}
\pnum
\constraints
\tcode{is_convertible_v<const T\&, basic_string_view<charT, traits>>}
为 \tcode{true}。

\pnum
\effects
创建一个变量 \tcode{sv},如同通过 \tcode{basic_string_view<charT, traits> sv = t;},然后行为等同于:
\begin{codeblock}
basic_string(sv.substr(pos, n), a);
\end{codeblock}
\end{itemdescr}

\indexlibraryctor{basic_string}%
\begin{itemdecl}
template<class T>
  constexpr explicit basic_string(const T& t, const Allocator& a = Allocator());
\end{itemdecl}

\begin{itemdescr}
\pnum
\constraints
\begin{itemize}
\item
\tcode{is_convertible_v<const T\&, basic_string_view<charT, traits>>} 为
\tcode{true} 且
\item
\tcode{is_convertible_v<const T\&, const charT*>} 为
\tcode{false}。
\end{itemize}

\pnum
\effects
如同通过 \tcode{basic_string_view<charT, traits> sv = t;} 创建变量 \tcode{sv},
然后行为与 \tcode{basic_string(sv.data(), sv.size(), a)} 相同。
\end{itemdescr}

\indexlibraryctor{basic_string}%
\begin{itemdecl}
constexpr basic_string(const charT* s, size_type n, const Allocator& a = Allocator());
\end{itemdecl}

\begin{itemdescr}
\pnum
\expects
\range{s}{s + n} 是有效范围。

\pnum
\effects
构造一个对象,其初始值为范围 \range{s}{s + n}。

\pnum
\ensures
\tcode{size()} 等于 \tcode{n},且
\tcode{traits::compare(data(), s, n)} 等于 \tcode{0}。
\end{itemdescr}

\indexlibraryctor{basic_string}%
\begin{itemdecl}
constexpr basic_string(const charT* s, const Allocator& a = Allocator());
\end{itemdecl}

\begin{itemdescr}
\pnum
\expects
\range{s}{s + traits::length(s)} 是有效范围。

\pnum
\effects
构造一个对象,其初始值为范围 \range{s}{s + traits::length(s)}。

\pnum
\ensures
\tcode{size()} 等于 \tcode{traits::length(s)},且
\tcode{traits::compare(data(), s, traits::length(s))} 等于 \tcode{0}。
\end{itemdescr}

\begin{itemdescr}
\pnum
\constraints
\tcode{Allocator} 是一种
符合分配器资格的类型\iref{container.reqmts}。
\begin{note}
这会影响类模板实参推导。
\end{note}

\pnum
\effects
等价于:\tcode{basic_string(s, traits::length(s), a)}。
\end{itemdescr}

\indexlibraryctor{basic_string}%
\begin{itemdecl}
constexpr basic_string(size_type n, charT c, const Allocator& a = Allocator());
\end{itemdecl}

\begin{itemdescr}
\pnum
\constraints
\tcode{Allocator} 是一种
符合分配器资格的类型\iref{container.reqmts}。
\begin{note}
这会影响类模板实参推导。
\end{note}

\pnum
\effects
构造一个对象,其值由 \tcode{n} 个 \tcode{c} 的副本组成。
\end{itemdescr}

\indexlibraryctor{basic_string}%
\begin{itemdecl}
template<class InputIterator>
  constexpr basic_string(InputIterator begin, Iterator end, const Allocator& a = Allocator());
\end{itemdecl}

\begin{itemdescr}
\pnum
\constraints
\tcode{InputIterator} 是符合输入迭代器\iref{container.reqmts}要求的类型。

\pnum
\effects
按照 \ref{sequence.reqmts} 中的规定,以范围 \range{begin}{end} 中的值构造字符串。
\end{itemdescr}

\indexlibraryctor{basic_string}%
\begin{itemdecl}
template<@\exposconcept{container-compatible-range}@<charT> R>
  constexpr basic_string(from_range_t, R&& rg, const Allocator& = Allocator());
\end{itemdecl}

\begin{itemdescr}
\pnum
\effects
以 \tcode{rg} 范围内的值构造一个字符串,如 \ref{sequence.reqmts} 中所规定。
\end{itemdescr}

\indexlibraryctor{basic_string}%
\begin{itemdecl}
constexpr basic_string(initializer_list<charT> il, const Allocator& a = Allocator());
\end{itemdecl}

\begin{itemdescr}
\pnum
\effects
等价于 \tcode{basic_string(il.begin(), il.end(), a)}。
\end{itemdescr}

\indexlibraryctor{basic_string}%
\begin{itemdecl}
constexpr basic_string(const basic_string& str, const Allocator& alloc);
constexpr basic_string(basic_string&& str, const Allocator& alloc);
\end{itemdecl}

\begin{itemdescr}
\pnum
\effects
构造一个对象,其值为此调用前 \tcode{str} 的值。
存储的分配器由 \tcode{alloc} 构造。
在第二种形式中,\tcode{str} 被置于有效但未指定的状态。

\pnum
\throws
若 \tcode{alloc == str.get_allocator()},第二种形式不抛出异常。
\end{itemdescr}

\begin{itemdecl}
template<class InputIterator,
         class Allocator = allocator<typename iterator_traits<InputIterator>::value_type>>
  basic_string(InputIterator, InputIterator, Allocator = Allocator())
    -> basic_string<typename iterator_traits<InputIterator>::value_type,
                    char_traits<typename iterator_traits<InputIterator>::value_type>,
                    Allocator>;
\end{itemdecl}

\begin{itemdescr}
\pnum
\constraints
\tcode{InputIterator} 是满足输入迭代器要求的类型,
且 \tcode{Allocator} 是满足分配器要求的类型\iref{container.reqmts}。
\end{itemdescr}

\begin{itemdecl}
template<class charT,
         class traits,
         class Allocator = allocator<charT>>
  explicit basic_string(basic_string_view<charT, traits>, const Allocator& = Allocator())
    -> basic_string<charT, traits, Allocator>;

template<class charT,
         class traits,
         class Allocator = allocator<charT>>
  basic_string(basic_string_view<charT, traits>,
               typename @\seebelow@::size_type, typename @\seebelow@::size_type,
               const Allocator& = Allocator())
    -> basic_string<charT, traits, Allocator>;
\end{itemdecl}

\begin{itemdescr}
\pnum
\constraints
\tcode{Allocator} 是一种满足分配器要求的类型\iref{container.reqmts}。
\end{itemdescr}

\indexlibrarymember{operator=}{basic_string}%
\begin{itemdecl}
constexpr basic_string& operator=(const basic_string& str);
\end{itemdecl}

\begin{itemdescr}
\pnum
\effects
如果 \tcode{*this} 与 \tcode{str} 是同一对象,则无效果。
否则,以 \tcode{str} 的副本替换 \tcode{*this} 的值。

\pnum
\returns
\tcode{*this}。
\end{itemdescr}

\indexlibrarymember{operator=}{basic_string}%
\begin{itemdecl}
constexpr basic_string& operator=(basic_string&& str)
  noexcept(allocator_traits<Allocator>::propagate_on_container_move_assignment::value ||
           allocator_traits<Allocator>::is_always_equal::value);
\end{itemdecl}

\begin{itemdescr}
\pnum
\effects
若 \tcode{*this} 和 \tcode{str} 为同一对象,则无效果。
否则,以 \tcode{str} 的值替换 \tcode{*this} 的内容。
\begin{note}
使用移动赋值意味着将 \tcode{str} 的有效状态留在满足其所有不变的某一状态中。
\end{note}

\pnum
\returns
\tcode{*this}。
\end{itemdescr}

\begin{itemdescr}
\pnum
\effects
如同序列容器般移动赋值\iref{sequence.reqmts},
但迭代器、指针和引用可能失效。

\pnum
\returns
\tcode{*this}。
\end{itemdescr}

\indexlibrarymember{operator=}{basic_string}%
\begin{itemdecl}
template<class T>
  constexpr basic_string& operator=(const T& t);
\end{itemdecl}

\begin{itemdescr}
\pnum
\constraints
\begin{itemize}
\item \tcode{is_convertible_v<const T\&, basic_string_view<charT, traits>>}
为 \tcode{true} 且
\item \tcode{is_convertible_v<const T\&, const charT*>}
为 \tcode{false}。
\end{itemize}

\pnum
\effects
等价于:
\begin{codeblock}
basic_string_view<charT, traits> sv = t;
return assign(sv);
\end{codeblock}
\end{itemdescr}

\indexlibrarymember{operator=}{basic_string}%
\begin{itemdecl}
constexpr basic_string& operator=(const charT* s);
\end{itemdecl}

\begin{itemdescr}
\pnum
\effects
等价于:
\tcode{return *this = basic_string_view<charT, traits>(s);}
\end{itemdescr}

\indexlibrarymember{operator=}{basic_string}%
\begin{itemdecl}
constexpr basic_string& operator=(charT c);
\end{itemdecl}

\begin{itemdescr}
\pnum
\effects
等价于:
\begin{codeblock}
return *this = basic_string_view<charT, traits>(addressof(c), 1);
\end{codeblock}
\end{itemdescr}

\indexlibrarymember{operator=}{basic_string}%
\begin{itemdecl}
constexpr basic_string& operator=(initializer_list<charT> il);
\end{itemdecl}

\begin{itemdescr}
\pnum
\effects
等价于:
\begin{codeblock}
return *this = basic_string_view<charT, traits>(il.begin(), il.size());
\end{codeblock}
\end{itemdescr}

\rSec3[string.iterators]{迭代器支持}

\indexlibrarymember{begin}{basic_string}%
\indexlibrarymember{cbegin}{basic_string}%
\begin{itemdecl}
constexpr iterator       begin() noexcept;
constexpr const_iterator begin() const noexcept;
constexpr const_iterator cbegin() const noexcept;
\end{itemdecl}

\begin{itemdescr}
\pnum
\returns
指向字符串首字符的迭代器。
\end{itemdescr}

\indexlibrarymember{end}{basic_string}%
\indexlibrarymember{cend}{basic_string}%
\begin{itemdecl}
constexpr iterator       end() noexcept;
constexpr const_iterator end() const noexcept;
constexpr const_iterator cend() const noexcept;
\end{itemdecl}

\begin{itemdescr}
\pnum
\returns
尾后迭代器。
\end{itemdescr}

\indexlibrarymember{rbegin}{basic_string}%
\indexlibrarymember{crbegin}{basic_string}%
\begin{itemdecl}
constexpr reverse_iterator       rbegin() noexcept;
constexpr const_reverse_iterator rbegin() const noexcept;
constexpr const_reverse_iterator crbegin() const noexcept;
\end{itemdecl}

\begin{itemdescr}
\pnum
\returns
语义上等价于
\tcode{reverse_iterator(end())}
的迭代器。
\end{itemdescr}

\indexlibrarymember{rend}{basic_string}%
\indexlibrarymember{crend}{basic_string}%
\begin{itemdecl}
constexpr reverse_iterator       rend() noexcept;
constexpr const_reverse_iterator rend() const noexcept;
constexpr const_reverse_iterator crend() const noexcept;
\end{itemdecl}

\begin{itemdescr}
\pnum
\returns
语义上等价于
\tcode{reverse_iterator(begin())}
的迭代器。
\end{itemdescr}

\rSec3[string.capacity]{容量}

\indexlibrarymember{size}{basic_string}%  
\indexlibrarymember{length}{basic_string}%  
\begin{itemdecl}
constexpr size_type size() const noexcept;
constexpr size_type length() const noexcept;
\end{itemdecl}

\begin{itemdescr}
\pnum
\returns 返回当前字符串中类字符对象的数量。

\pnum
\complexity 常数时间。
\end{itemdescr}

\indexlibrarymember{max_size}{basic_string}%  
\begin{itemdecl}
constexpr size_type max_size() const noexcept;
\end{itemdecl}

\begin{itemdescr}
\pnum
\returns
可以存储在 \tcode{basic_string} 中的字符类型对象的最大可能数量。

\pnum
\complexity
常数时间。
\end{itemdescr}

\indexlibrarymember{resize}{basic_string}%
\begin{itemdecl}
constexpr void resize(size_type n, charT c);
\end{itemdecl}

\begin{itemdescr}
\pnum
\effects
按如下方式改变 \tcode{*this} 的值:
\begin{itemize}
\item
若 \tcode{n <= size()},则擦除末尾的 \tcode{size() - n} 个元素。
\item
若 \tcode{n > size()},则追加 \tcode{n - size()} 个 \tcode{c} 的副本。
\end{itemize}
\end{itemdescr}

\indexlibrarymember{resize}{basic_string}%
\begin{itemdecl}
constexpr void resize(size_type n);
\end{itemdecl}

\begin{itemdescr}
\pnum
\effects
等价于 \tcode{resize(n, charT())}。
\end{itemdescr}

\indexlibrarymember{resize_and_overwrite}{basic_string}%
\begin{itemdecl}
template<class Operation> constexpr void resize_and_overwrite(size_type n, Operation op);
\end{itemdecl}

\begin{itemdescr}
\pnum
令
\begin{itemize}
\item
\tcode{o} 为调用 \tcode{resize_and_overwrite} 前的 \tcode{size()}。
\item
\tcode{k} 为 \tcode{min(o, n)}。
\item
\tcode{p} 为 \tcode{charT*} 或 \tcode{charT* const} 类型的值,使得在调用前区间 \crange{p}{p + n} 有效,且 \tcode{this->compare(0, k, p, k) == 0} 为 \tcode{true}。区间 \crange{p + k}{p + n} 中的值可能为未明确的\iref{basic.indet}。
\item
\tcode{m} 为 \tcode{size_type} 或 \tcode{const size_type} 类型的值,等于 \tcode{n}。
\item
\tcode{\placeholder{OP}} 为表达式 \tcode{std::move(op)(p, m)}。
\item
\tcode{r} = \tcode{\placeholder{OP}}。
\end{itemize}

\pnum
\mandates
\tcode{\placeholder{OP}} 具有整数类型\iref{iterator.concept.winc}。

\pnum
\expects
\begin{itemize}
\item
\tcode{\placeholder{OP}} 不抛出异常,也不修改 \tcode{p} 或 \tcode{m}。
\item
$\tcode{r} \geq 0$。
\item
$\tcode{r} \leq \tcode{m}$。
\item
在求值 \tcode{\placeholder{OP}} 后,区间 \range{p}{p + r} 中没有未明确的值。
\end{itemize}

\pnum
\effects
求值 \tcode{\placeholder{OP}},将 \tcode{*this} 的内容替换为 \range{p}{p + r},并使所有指向区间 \crange{p}{p + n} 的指针和引用失效。

\pnum
\recommended
实现应避免不必要的复制和分配,例如,可以将 \tcode{p} 设为指向内部存储的指针,并在求值 \tcode{\placeholder{OP}} 后将 \tcode{*(p + r)} 恢复为 \tcode{charT()}。
\end{itemdescr}

\indexlibrarymember{capacity}{basic_string}%
\begin{itemdecl}
constexpr size_type capacity() const noexcept;
\end{itemdecl}

\begin{itemdescr}
\pnum
\returns
字符串中已分配存储的大小。

\pnum
\complexity
常数时间。
\end{itemdescr}

\indexlibrarymember{reserve}{basic_string}%
\begin{itemdecl}
constexpr void reserve(size_type res_arg);
\end{itemdecl}

\begin{itemdescr}
\pnum
\effects
一个指令,用于告知 \tcode{basic_string} 计划的大小变更,以便相应管理存储分配。
在调用 \tcode{reserve} 后,
如果发生重分配,则 \tcode{capacity()} 大于或等于 \tcode{reserve} 的实参;
否则等于 \tcode{capacity()} 的先前值。
当且仅当当前容量小于 \tcode{reserve} 的实参时,此时会发生重分配。

\pnum
\throws
若 \tcode{res_arg > max_size()} 则抛出 \tcode{length_error} 或
由 \tcode{allocator_traits<Allocator>::allocate} 抛出的任何异常。
\end{itemdescr}

\indexlibrarymember{shrink_to_fit}{basic_string}%
\begin{itemdecl}
constexpr void shrink_to_fit();
\end{itemdecl}

\begin{itemdescr}
\pnum
\effects
\tcode{shrink_to_fit} 是一个非强制请求,要求将 \tcode{capacity()} 减少至 \tcode{size()}。
\begin{note}
该请求是非强制的,以便为实现特定的优化留出余地。
\end{note}
它不会增加 \tcode{capacity()},但可能通过导致重新分配来减少 \tcode{capacity()}。

\pnum
\complexity
如果当前大小不等于旧的容量,则复杂度与序列的大小成线性关系;
否则为常数。

\pnum
\remarks
重新分配会使所有指向序列中元素的引用、指针和迭代器以及尾后迭代器失效。
\begin{note}
如果没有发生重新分配,它们保持有效。
\end{note}
\end{itemdescr}

\indexlibrarymember{clear}{basic_string}%
\begin{itemdecl}
constexpr void clear() noexcept;
\end{itemdecl}

\begin{itemdescr}
\pnum
\effects
等价于:\tcode{erase(begin(), end());}
\end{itemdescr}

\indexlibrarymember{empty}{basic_string}%
\begin{itemdecl}
constexpr bool empty() const noexcept;
\end{itemdecl}

\begin{itemdescr}
\pnum
\effects
等价于:
\tcode{return size() == 0;}
\end{itemdescr}

\rSec3[string.access]{元素访问}

\indexlibrarymember{operator[]}{basic_string}%
\begin{itemdecl}
constexpr const_reference operator[](size_type pos) const;
constexpr reference       operator[](size_type pos);
\end{itemdecl}

\begin{itemdescr}
\pnum
\hardexpects
\tcode{pos <= size()} 为 \tcode{true}。

\pnum
\returns
若 \tcode{pos < size()} 则返回 \tcode{*(begin() + pos)}。否则,返回一个对值为 \tcode{charT()} 的 \tcode{charT} 类型对象的引用,将该对象修改为任何不等于 \tcode{charT()} 的值会导致未定义行为。

\pnum
\throws
无。

\pnum
\complexity
常数时间。
\end{itemdescr}

\indexlibrarymember{at}{basic_string}%
\begin{itemdecl}
constexpr const_reference at(size_type pos) const;
constexpr reference       at(size_type pos);
\end{itemdecl}

\begin{itemdescr}
\pnum
\returns
\tcode{operator[](pos)}。

\pnum
\throws
若 \tcode{pos >= size()} 则抛出 \tcode{out_of_range}。
\end{itemdescr}

\indexlibrarymember{front}{basic_string}%
\begin{itemdecl}
constexpr const charT& front() const;
constexpr charT& front();
\end{itemdecl}

\begin{itemdescr}
\pnum
\hardexpects
\tcode{empty()} 为 \tcode{false}。

\pnum
\effects
等价于:\tcode{return operator[](0);}
\end{itemdescr}

\indexlibrarymember{back}{basic_string}%
\begin{itemdecl}
constexpr const charT& back() const;
constexpr charT& back();
\end{itemdecl}

\begin{itemdescr}
\pnum
\hardexpects
\tcode{empty()} 为 \tcode{false}。

\pnum
\effects
等价于:\tcode{return operator[](size() - 1);}
\end{itemdescr}

\rSec3[string.modifiers]{修饰符}

\rSec4[string.op.append]{\tcode{basic_string::operator+=}}

\indexlibrarymember{operator+=}{basic_string}%
\begin{itemdecl}
constexpr basic_string& operator+=(const basic_string& str);
\end{itemdecl}

\begin{itemdescr}
\pnum
\effects
等价于:\tcode{return append(str);}
\end{itemdescr}

\indexlibrarymember{operator+=}{basic_string}%
\begin{itemdecl}
template<class T>
  constexpr basic_string& operator+=(const T& t);
\end{itemdecl}

\begin{itemdescr}
\pnum
\constraints
\begin{itemize}
\item
\tcode{is_convertible_v<const T\&, basic_string_view<charT, traits>>} 为
\tcode{true} 且
\item
\tcode{is_convertible_v<const T\&, const charT*>} 为
\tcode{false}。
\end{itemize}

\pnum
\effects
等价于:
\begin{codeblock}
basic_string_view<charT, traits> sv = t;
return append(sv);
\end{codeblock}
\end{itemdescr}

\indexlibrarymember{operator+=}{basic_string}%
\begin{itemdecl}
constexpr basic_string& operator+=(const charT* s);
\end{itemdecl}

\begin{itemdescr}
\pnum
\effects
等价于:\tcode{return append(s);}
\end{itemdescr}

\indexlibrarymember{operator+=}{basic_string}%
\begin{itemdecl}
constexpr basic_string& operator+=(charT c);
\end{itemdecl}

\begin{itemdescr}
\pnum
\effects
等价于:\tcode{return append(size_type\{1\}, c);}
\end{itemdescr}

\indexlibrarymember{operator+=}{basic_string}%
\begin{itemdecl}
constexpr basic_string& operator+=(initializer_list<charT> il);
\end{itemdecl}

\begin{itemdescr}
\pnum
\effects
等价于:\tcode{return append(il);}
\end{itemdescr}

\rSec4[string.append]{\tcode{basic_string::append}}

\indexlibrarymember{append}{basic_string}%
\begin{itemdecl}
constexpr basic_string& append(const basic_string& str);
\end{itemdecl}

\begin{itemdescr}
\pnum
\effects
等价于:\tcode{return append(str.data(), str.size());}
\end{itemdescr}

\indexlibrarymember{append}{basic_string}%
\begin{itemdecl}
constexpr basic_string& append(const basic_string& str, size_type pos, size_type n = npos);
\end{itemdecl}

\begin{itemdescr}
\pnum
\effects
等价于:
\begin{codeblock}
return append(basic_string_view<charT, traits>(str).substr(pos, n));
\end{codeblock}
\end{itemdescr}

\indexlibrarymember{append}{basic_string}%
\begin{itemdecl}
template<class T>
  constexpr basic_string& append(const T& t);
\end{itemdecl}

\begin{itemdescr}
\pnum
\constraints
\begin{itemize}
\item
\tcode{is_convertible_v<const T\&, basic_string_view<charT, traits>>} 为
\tcode{true} 且
\item
\tcode{is_convertible_v<const T\&, const charT*>} 为
\tcode{false}。
\end{itemize}

\pnum
\effects
等价于:
\begin{codeblock}
basic_string_view<charT, traits> sv = t;
return append(sv.data(), sv.size());
\end{codeblock}
\end{itemdescr}

\indexlibrarymember{append}{basic_string}%
\begin{itemdecl}
template<class T>
  constexpr basic_string& append(const T& t, size_type pos, size_type n = npos);
\end{itemdecl}

\begin{itemdescr}
\pnum
\constraints
\begin{itemize}
\item
\tcode{is_convertible_v<const T\&, basic_string_view<charT, traits>>} 为
\tcode{true} 且
\item
\tcode{is_convertible_v<const T\&, const charT*>} 为
\tcode{false}。
\end{itemize}

\pnum
\effects
等价于:
\begin{codeblock}
basic_string_view<charT, traits> sv = t;
return append(sv.substr(pos, n));
\end{codeblock}
\end{itemdescr}

\indexlibrarymember{append}{basic_string}%
\begin{itemdecl}
constexpr basic_string& append(const charT* s, size_type n);
\end{itemdecl}

\begin{itemdescr}
\pnum
\expects
\range{s}{s + n} 是一个有效范围。

\pnum
\effects
将范围 \range{s}{s + n} 的副本追加到字符串。

\pnum
\returns
\tcode{*this}。
\end{itemdescr}

\indexlibrarymember{append}{basic_string}%
\begin{itemdecl}
constexpr basic_string& append(const charT* s);
\end{itemdecl}

\begin{itemdescr}
\pnum
\effects
等效于:\tcode{return append(s, traits::length(s));}
\end{itemdescr}

\indexlibrarymember{append}{basic_string}%
\begin{itemdecl}
constexpr basic_string& append(size_type n, charT c);
\end{itemdecl}

\begin{itemdescr}
\pnum
\effects
将 \tcode{c} 的 \tcode{n} 个副本追加到字符串。

\pnum
\returns
\tcode{*this}。
\end{itemdescr}

\indexlibrarymember{append}{basic_string}%
\begin{itemdecl}
template<class InputIterator>
  constexpr basic_string& append(InputIterator first, InputIterator last);
\end{itemdecl}

\begin{itemdescr}
\pnum
\constraints
\tcode{InputIterator} 是符合输入迭代器要求的类型\iref{container.reqmts}。

\pnum
\effects
等价于:\tcode{return append(basic_string(first, last, get_allocator()));}
\end{itemdescr}

\indexlibrarymember{append_range}{basic_string}%
\begin{itemdecl}
template<@\exposconcept{container-compatible-range}@<charT> R>
  constexpr basic_string& append_range(R&& rg);
\end{itemdecl}

\begin{itemdescr}
\pnum
\effects
等效于: \tcode{return append(basic_string(from_range, std::forward<R>(rg), get_allocator()));}
\end{itemdescr}

\indexlibrarymember{append}{basic_string}%
\begin{itemdecl}
constexpr basic_string& append(initializer_list<charT> il);
\end{itemdecl}

\begin{itemdescr}
\pnum
\effects
等效于: \tcode{return append(il.begin(), il.size());}
\end{itemdescr}

\indexlibrarymember{push_back}{basic_string}%
\begin{itemdecl}
constexpr void push_back(charT c);
\end{itemdecl}

\begin{itemdescr}
\pnum
\effects
等价于
\tcode{append(size\_type\{1\}, c)}。
\end{itemdescr}

\rSec4[string.assign]{\tcode{basic\_string::assign}}

\indexlibrarymember{assign}{basic_string}%
\begin{itemdecl}
constexpr basic_string& assign(const basic_string& str);
\end{itemdecl}

\begin{itemdescr}
\pnum
\effects
等价于: \tcode{return *this = str;}
\end{itemdescr}

\indexlibrarymember{assign}{basic_string}%
\begin{itemdecl}
constexpr basic_string& assign(basic_string&& str)
  noexcept(allocator_traits<Allocator>::propagate_on_container_move_assignment::value ||
           allocator_traits<Allocator>::is_always_equal::value);
\end{itemdecl}

\begin{itemdescr}
\pnum
\effects
等价于:\tcode{return *this = std::move(str);}
\end{itemdescr}


\indexlibrarymember{assign}{basic_string}%
\begin{itemdecl}
constexpr basic_string& assign(const basic_string& str, size_type pos, size_type n = npos);
\end{itemdecl}

\begin{itemdescr}
\pnum
\effects
等价于:
\begin{codeblock}
return assign(basic_string_view<charT, traits>(str).substr(pos, n));
\end{codeblock}
\end{itemdescr}

\indexlibrarymember{assign}{basic_string}%
\begin{itemdecl}
template<class T>
  constexpr basic_string& assign(const T& t);
\end{itemdecl}

\begin{itemdescr}
\pnum
\constraints
\begin{itemize}
\item
\tcode{is_convertible_v<const T\&, basic_string_view<charT, traits>>} 为
\tcode{true},且
\item
\tcode{is_convertible_v<const T\&, const charT*>} 为
\tcode{false}。
\end{itemize}

\pnum
\effects
等价于:
\begin{codeblock}
basic_string_view<charT, traits> sv = t;
return assign(sv.data(), sv.size());
\end{codeblock}
\end{itemdescr}

\indexlibrarymember{assign}{basic_string}%
\begin{itemdecl}
template<class T>
  constexpr basic_string& assign(const T& t, size_type pos, size_type n = npos);
\end{itemdecl}

\begin{itemdescr}
\pnum
\constraints
\begin{itemize}
\item
\tcode{is_convertible_v<const T\&, basic_string_view<charT, traits>>} 为
\tcode{true},且
\item
\tcode{is_convertible_v<const T\&, const charT*>} 为
\tcode{false}。
\end{itemize}

\pnum
\effects
等价于:
\begin{codeblock}
basic_string_view<charT, traits> sv = t;
return assign(sv.substr(pos, n));
\end{codeblock}
\end{itemdescr}

\indexlibrarymember{assign}{basic_string}%
\begin{itemdecl}
constexpr basic_string& assign(const charT* s, size_type n);
\end{itemdecl}

\begin{itemdescr}
\pnum
\expects
范围 \range{s}{s + n} 是有效范围。

\pnum
\effects
将 \tcode{*this} 控制的字符串替换为范围 \range{s}{s + n} 的副本。

\pnum
\returns
\tcode{*this}。
\end{itemdescr}

\indexlibrarymember{assign}{basic_string}%
\begin{itemdecl}
constexpr basic_string& assign(const charT* s);
\end{itemdecl}

\begin{itemdescr}
\pnum
\effects
等价于:\tcode{return assign(s, traits::length(s));}
\end{itemdescr}

\indexlibrarymember{assign}{basic_string}%
\begin{itemdecl}
constexpr basic_string& assign(initializer_list<charT> il);
\end{itemdecl}

\begin{itemdescr}
\pnum
\effects
等价于:\tcode{return assign(il.begin(), il.size());}
\end{itemdescr}

\begin{itemdescr}
\pnum
\effects
等价于:\tcode{return assign(il.begin(), il.size());}
\end{itemdescr}

\indexlibrarymember{assign}{basic_string}%
\begin{itemdecl}
constexpr basic_string& assign(size_type n, charT c);
\end{itemdecl}

\begin{itemdescr}
\pnum
\effects
等价于:
\begin{codeblock}
clear();
resize(n, c);
return *this;
\end{codeblock}
\end{itemdescr}

\indexlibrarymember{assign}{basic_string}%
\begin{itemdecl}
template<class InputIterator>
  constexpr basic_string& assign(InputIterator first, InputIterator last);
\end{itemdecl}

\begin{itemdescr}
\pnum
\constraints
\tcode{InputIterator} 是符合输入迭代器要求的类型\iref{container.reqmts}。

\pnum
\effects
效果等价于:\tcode{return assign(basic_string(first, last, get_allocator()));}
\end{itemdescr}

\indexlibrarymember{assign_range}{basic_string}%
\begin{itemdecl}
template<@\exposconcept{container-compatible-range}@<charT> R>
  constexpr basic_string& assign_range(R&& rg);
\end{itemdecl}

\begin{itemdescr}
\pnum
\effects
效果等价于:\tcode{return assign(basic_string(from_range, std::forward<R>(rg), get_allocator()));}
\end{itemdescr}

\rSec4[string.insert]{\tcode{basic_string::insert}}

\indexlibrarymember{insert}{basic_string}%
\begin{itemdecl}
constexpr basic_string& insert(size_type pos, const basic_string& str);
\end{itemdecl}

\begin{itemdescr}
\pnum
\effects
等价于: \tcode{return insert(pos, str.data(), str.size());}
\end{itemdescr}

\indexlibrarymember{insert}{basic_string}%
\begin{itemdecl}
constexpr basic_string& insert(size_type pos1, const basic_string& str,
                               size_type pos2, size_type n = npos);
\end{itemdecl}

\begin{itemdescr}
\pnum
\effects
等价于:
\begin{codeblock}
return insert(pos1, basic_string_view<charT, traits>(str), pos2, n);
\end{codeblock}
\end{itemdescr}

\indexlibrarymember{insert}{basic_string}%
\begin{itemdecl}
template<class T>
  constexpr basic_string& insert(size_type pos, const T& t);
\end{itemdecl}

\begin{itemdescr}
\pnum
\constraints
\begin{itemize}
\item
\tcode{is_convertible_v<const T\&, basic_string_view<charT, traits>>} 为
\tcode{true} 且
\item
\tcode{is_convertible_v<const T\&, const charT*>} 为
\tcode{false}。
\end{itemize}

\pnum
\effects
等价于:
\begin{codeblock}
basic_string_view<charT, traits> sv = t;
return insert(pos, sv.data(), sv.size());
\end{codeblock}
\end{itemdescr}

\indexlibrarymember{insert}{basic_string}%
\begin{itemdecl}
template<class T>
  constexpr basic_string& insert(size_type pos1, const T& t,
                                 size_type pos2, size_type n = npos);
\end{itemdecl}

\begin{itemdescr}
\pnum
\constraints
\begin{itemize}
\item
\tcode{is_convertible_v<const T\&, basic_string_view<charT, traits>>} 为
\tcode{true} 并且
\item
\tcode{is_convertible_v<const T\&, const charT*>} 为
\tcode{false}。
\end{itemize}

\pnum
\effects
等价于:
\begin{codeblock}
basic_string_view<charT, traits> sv = t;
return insert(pos1, sv.substr(pos2, n));
\end{codeblock}
\end{itemdescr}

\indexlibrary{insert}{basic_string}%
\begin{itemdecl}
constexpr basic_string& insert(size_type pos, const charT* s, size_type n);
\end{itemdecl}

\begin{itemdescr}
\pnum
\expects
\range{s}{s + n} 是一个有效范围。

\pnum
\effects
在 \tcode{pos} 位置的字符之前插入范围 \range{s}{s + n} 的一份副本,若 \tcode{pos < size()};否则插入到字符串末尾。

\pnum
\returns
\tcode{*this}。

\pnum
\throws
\begin{itemize}
\item 如果 \tcode{pos > size()},抛出 \tcode{out_of_range},
\item 如果 \tcode{n > max_size() - size()},抛出 \tcode{length_error},或
\item 抛出由 \tcode{allocator_traits<Allocator>::allocate} 抛出的任何异常。
\end{itemize}
\end{itemdescr}

\indexlibrarymember{insert}{basic_string}%
\begin{itemdecl}
constexpr basic_string& insert(size_type pos, const charT* s);
\end{itemdecl}

\begin{itemdescr}
\pnum
\effects
等价于:\tcode{return insert(pos, s, traits::length(s));}
\end{itemdescr}

\indexlibrarymember{insert}{basic_string}%
\begin{itemdecl}
constexpr basic_string& insert(size_type pos, size_type n, charT c);
\end{itemdecl}

\begin{itemdescr}
\pnum
\effects
若 \tcode{pos < size()},则在位置 \tcode{pos} 的字符前插入 \tcode{n} 个 \tcode{c} 的副本;
否则在字符串的末尾插入。

\pnum
\returns
\tcode{*this}。

\pnum
\throws
\begin{itemize}
\item 若 \tcode{pos > size()},则为 \tcode{out_of_range},
\item 若 \tcode{n > max_size() - size()},则为 \tcode{length_error},或
\item 任何 \tcode{allocator_traits<Allocator>::allocate} 抛出的异常。
\end{itemize}
\end{itemdescr}

\indexlibrarymember{insert}{basic_string}%
\begin{itemdecl}
constexpr iterator insert(const_iterator p, charT c);
\end{itemdecl}

\begin{itemdescr}
\pnum
\expects
\tcode{p} 是 \tcode{*this} 上的有效迭代器。

\pnum
\effects
在位置 \tcode{p} 处插入 \tcode{c} 的副本。

\pnum
\returns
一个指向被插入字符的迭代器。
\end{itemdescr}

\indexlibrarymember{insert}{basic_string}%
\begin{itemdecl}
constexpr iterator insert(const_iterator p, size_type n, charT c);
\end{itemdecl}

\begin{itemdescr}
\pnum
\expects
\tcode{p} 是 \tcode{*this} 上的有效迭代器。

\pnum
\effects
在位置 \tcode{p} 处插入 \tcode{n} 个 \tcode{c} 的副本。

\pnum
\returns
一个指向首个被插入字符的迭代器,或若 \tcode{n == 0} 则为 \tcode{p}。
\end{itemdescr}

\indexlibrarymember{insert}{basic_string}%
\begin{itemdecl}
template<class InputIterator>
  constexpr iterator insert(const_iterator p, InputIterator first, InputIterator last);
\end{itemdecl}

\begin{itemdescr}
\pnum
\constraints
\tcode{InputIterator} 是满足输入
迭代器\iref{container.reqmts} 要求的类型。

\pnum
\expects
\tcode{p} 是 \tcode{*this} 上的有效迭代器。

\pnum
\effects
效果等同于
\tcode{insert(p - begin(), basic_string(first, last, get_allocator()))}。

\pnum
\returns
返回指向首个被插入字符的迭代器,或者若 \tcode{first == last} 则返回 \tcode{p}。
\end{itemdescr}

\indexlibrarymember{insert_range}{basic_string}%
\begin{itemdecl}
template<@\exposconcept{container-compatible-range}@<charT> R>
  constexpr iterator insert_range(const_iterator p, R&& rg);
\end{itemdecl}

\begin{itemdescr}
\pnum
\expects
\tcode{p} 是 \tcode{*this} 上的一个有效迭代器。

\pnum
\effects
等价于
\tcode{insert(p - begin(), basic_string(from_range, std::forward<R>(rg), get_allocator()))}。

\pnum
\returns
一个引用到第一个被插入字符的迭代器,或者
\tcode{p} 如果 \tcode{rg} 为空。
\end{itemdescr}

\indexlibrarymember{insert}{basic_string}%
\begin{itemdecl}
constexpr iterator insert(const_iterator p, initializer_list<charT> il);
\end{itemdecl}

\begin{itemdescr}
\pnum
\effects
等价于: \tcode{return insert(p, il.begin(), il.end());}
\end{itemdescr}

\rSec4[string.erase]{\tcode{basic_string::erase}}

\indexlibrarymember{erase}{basic_string}%
\begin{itemdecl}
constexpr basic_string& erase(size_type pos = 0, size_type n = npos);
\end{itemdecl}

\begin{itemdescr}
\pnum
\effects
确定要被移除的字符串的有效长度 \tcode{xlen}
为 \tcode{n} 与 \tcode{size() - pos} 中较小者。
移除范围 \range{begin() + pos}{begin() + pos + xlen} 中的字符。

\pnum
\returns
\tcode{*this}。

\pnum
\throws
若 \tcode{pos > size()} 则抛出 \tcode{out_of_range}。
\end{itemdescr}

\indexlibrarymember{erase}{basic_string}%
\begin{itemdecl}
constexpr iterator erase(const_iterator p);
\end{itemdecl}

\begin{itemdescr}
\pnum
\expects
\tcode{p} 是 \tcode{*this} 上的有效可解引用迭代器。

\pnum
\effects
移除 \tcode{p} 所指向的字符。

\pnum
\returns
一个迭代器,指向被擦除元素之前紧接 \tcode{p} 之后的元素。
若不存在这样的元素,则返回 \tcode{end()}。

\pnum
\throws
无。
\end{itemdescr}

\indexlibrarymember{erase}{basic_string}%
\begin{itemdecl}
constexpr iterator erase(const_iterator first, const_iterator last);
\end{itemdecl}

\begin{itemdescr}
\pnum
\expects
\tcode{first} 与 \tcode{last} 是 \tcode{*this} 上的有效迭代器。
\range{first}{last} 是一个有效范围。

\pnum
\effects
移除范围 \range{first}{last} 中的字符。

\pnum
\returns
一个迭代器,指向其他元素被擦除之前由 \tcode{last} 指向的元素。
若不存在这样的元素,则返回 \tcode{end()}。

\pnum
\throws
无。
\end{itemdescr}

\indexlibrarymember{pop_back}{basic_string}%
\begin{itemdecl}
constexpr void pop_back();
\end{itemdecl}

\begin{itemdescr}
\pnum
\hardexpects
\tcode{empty()} 为 \tcode{false}。

\pnum
\effects
等价于 \tcode{erase(end() - 1)}。

\pnum
\throws
无。
\end{itemdescr}

\rSec4[string.replace]{\tcode{basic_string::replace}}

\indexlibrarymember{replace}{basic_string}%
\begin{itemdecl}
constexpr basic_string& replace(size_type pos1, size_type n1, const basic_string& str);
\end{itemdecl}

\begin{itemdescr}
\pnum
\effects
等价于:\tcode{return replace(pos1, n1, str.data(), str.size());}
\end{itemdescr}

\indexlibrarymember{replace}{basic_string}%
\begin{itemdecl}
constexpr basic_string& replace(size_type pos1, size_type n1, const basic_string& str,
                                size_type pos2, size_type n2 = npos);
\end{itemdecl}

\begin{itemdescr}
\pnum
\effects
等价于:
\begin{codeblock}
return replace(pos1, n1, basic_string_view<charT, traits>(str).substr(pos2, n2));
\end{codeblock}
\end{itemdescr}

\indexlibrarymember{replace}{basic_string}%
\begin{itemdecl}
template<class T>
  constexpr basic_string& replace(size_type pos1, size_type n1, const T& t);
\end{itemdecl}

\begin{itemdescr}
\pnum
\constraints
\begin{itemize}
\item
\tcode{is_convertible_v<const T\&, basic_string_view<charT, traits>>} 为
\tcode{true},且
\item
\tcode{is_convertible_v<const T\&, const charT*>} 为
\tcode{false}。
\end{itemize}

\pnum
\effects
等价于:
\begin{codeblock}
basic_string_view<charT, traits> sv = t;
return replace(pos1, n1, sv.data(), sv.size());
\end{codeblock}
\end{itemdescr}

\indexlibrarymember{replace}{basic_string}%
\begin{itemdecl}
template<class T>
  constexpr basic_string& replace(size_type pos1, size_type n1, const T& t,
                                  size_type pos2, size_type n2 = npos);
\end{itemdecl}

\begin{itemdescr}
\pnum
\constraints
\begin{itemize}
\item
\tcode{is_convertible_v<const T\&, basic_string_view<charT, traits>>} 为
\tcode{true} 且
\item
\tcode{is_convertible_v<const T\&, const charT*>} 为
\tcode{false}。
\end{itemize}

\pnum
\effects
等价于:
\begin{codeblock}
basic_string_view<charT, traits> sv = t;
return replace(pos1, n1, sv.substr(pos2, n2));
\end{codeblock}
\end{itemdescr}

\indexlibrarymember{replace}{basic_string}%
\begin{itemdecl}
constexpr basic_string& replace(size_type pos1, size_type n1, const charT* s, size_type n2);
\end{itemdecl}

\begin{itemdescr}
\pnum
\expects
\range{s}{s + n2} 是有效范围。

\pnum
\effects
确定待移除字符串的有效长度 \tcode{xlen} 为 \tcode{n1} 与 \tcode{size() - pos1} 的较小者。若 \tcode{size() - xlen >= max_size() - n2} 则抛出 \tcode{length_error}。否则,此函数将范围 \range{begin() + pos1}{begin() + pos1 + xlen} 中的字符替换为范围 \range{s}{s + n2} 的副本。

\pnum
\returns
\tcode{*this}。

\pnum
\throws
\begin{itemize}
\item 若 \tcode{pos1 > size()} 则为 \tcode{out_of_range},
\item 若结果字符串的长度将超过 \tcode{max_size()} 则为 \tcode{length_error},或
\item \tcode{allocator_traits<Allocator>::allocate} 抛出的任何异常。
\end{itemize}
\end{itemdescr}

\indexlibrarymember{replace}{basic_string}%
\begin{itemdecl}
constexpr basic_string& replace(size_type pos, size_type n, const charT* s);
\end{itemdecl}

\begin{itemdescr}
\pnum
\effects
等价于:\tcode{return replace(pos, n, s, traits::length(s));}
\end{itemdescr}

\indexlibrarymember{replace}{basic_string}%
\begin{itemdecl}
constexpr basic_string& replace(size_type pos1, size_type n1, size_type n2, charT c);
\end{itemdecl}

\begin{itemdescr}
\pnum
\effects
确定待移除字符串的有效长度 \tcode{xlen} 为 \tcode{n1} 与 \tcode{size() - pos1} 中的较小者。
若 \tcode{size() - xlen >=} \tcode{max_size() - n2} 则抛出 \tcode{length_error}。
否则,该函数以 \tcode{n2} 个 \tcode{c} 的副本替换范围
\range{begin() + pos1}{begin() + pos1 + xlen} 中的字符。

\pnum
\returns
\tcode{*this}。

\pnum
\throws
\begin{itemize}
\item 若 \tcode{pos1 > size()} 则抛出 \tcode{out_of_range},
\item 若结果字符串的长度将超过 \tcode{max_size()} 则抛出 \tcode{length_error},或
\item \tcode{allocator_traits<Allocator>::allocate} 所抛出的任何异常。
\end{itemize}
\end{itemdescr}

\indexlibrarymember{replace}{basic_string}%
\begin{itemdecl}
constexpr basic_string& replace(const_iterator i1, const_iterator i2, const basic_string& str);
\end{itemdecl}

\begin{itemdescr}
\pnum
\effects
等效于:\tcode{return replace(i1, i2, basic_string_view<charT, traits>(str));}
\end{itemdescr}

\indexlibrarymember{replace}{basic_string}%
\begin{itemdecl}
template<class T>
  constexpr basic_string& replace(const_iterator i1, const_iterator i2, const T& t);
\end{itemdecl}

\begin{itemdescr}
\pnum
\constraints
\begin{itemize}
\item
\tcode{is_convertible_v<const T\&, basic_string_view<charT, traits>>} 为
\tcode{true} 且
\item
\tcode{is_convertible_v<const T\&, const charT*>} 为
\tcode{false}。
\end{itemize}

\pnum
\expects
\range{begin()}{i1} 与 \range{i1}{i2} 为有效范围。

\pnum
\effects
等效于:
\begin{codeblock}
basic_string_view<charT, traits> sv = t;
return replace(i1 - begin(), i2 - i1, sv.data(), sv.size());
\end{codeblock}
\end{itemdescr}

\indexlibrarymember{replace}{basic_string}%
\begin{itemdecl}
constexpr basic_string& replace(const_iterator i1, const_iterator i2, const charT* s, size_type n);
\end{itemdecl}

\begin{itemdescr}
\pnum
\effects
等效于:\tcode{return replace(i1, i2, basic_string_view<charT, traits>(s, n));}
\end{itemdescr}

\indexlibrarymember{replace}{basic_string}%
\begin{itemdecl}
constexpr basic_string& replace(const_iterator i1, const_iterator i2, const charT* s);
\end{itemdecl}

\begin{itemdescr}
\pnum
\effects
等效于:\tcode{return replace(i1, i2, basic_string_view<charT, traits>(s));}
\end{itemdescr}

\indexlibrarymember{replace}{basic_string}%
\begin{itemdecl}
constexpr basic_string& replace(const_iterator i1, const_iterator i2, size_type n, charT c);
\end{itemdecl}

\begin{itemdescr}
\pnum
\expects
\range{begin()}{i1} 和 \range{i1}{i2} 是有效范围。

\pnum
\effects
等价于: \tcode{return replace(i1 - begin(), i2 - i1, n, c);}
\end{itemdescr}

\indexlibrarymember{replace}{basic_string}%
\begin{itemdecl}
template<class InputIterator>
  constexpr basic_string& replace(const_iterator i1, const_iterator i2,
                                  InputIterator j1, InputIterator j2);
\end{itemdecl}

\begin{itemdescr}
\pnum
\expects
\range{begin()}{i1} 和 \range{i1}{i2} 是有效范围。

\pnum
\effects
等价于: \tcode{return replace(i1 - begin(), i2 - i1, basic_string(j1, j2));}
\end{itemdescr}

\begin{itemdescr}
\pnum
\constraints
\tcode{InputIterator} 是一种满足输入迭代器要求的类型\iref{container.reqmts}。

\pnum
\effects
等价于:\tcode{return replace(i1, i2, basic_string(j1, j2, get_allocator()));}
\end{itemdescr}

\indexlibrarymember{replace_with_range}{basic_string}%
\begin{itemdecl}
template<@\exposconcept{container-compatible-range}@<charT> R>
  constexpr basic_string& replace_with_range(const_iterator i1, const_iterator i2, R&& rg);
\end{itemdecl}

\begin{itemdescr}
\pnum
\effects
等价于:
\begin{codeblock}
return replace(i1, i2, basic_string(from_range, std::forward<R>(rg), get_allocator()));
\end{codeblock}
\end{itemdescr}

\indexlibrarymember{replace}{basic_string}%
\begin{itemdecl}
constexpr basic_string& replace(const_iterator i1, const_iterator i2, initializer_list<charT> il);
\end{itemdecl}

\begin{itemdescr}
\pnum
\effects
等效于:\tcode{return replace(i1, i2, il.begin(), il.size());}
\end{itemdescr}

\rSec4[string.copy]{\tcode{basic_string::copy}}

\indexlibrarymember{copy}{basic_string}%
\begin{itemdecl}
constexpr size_type copy(charT* s, size_type n, size_type pos = 0) const;
\end{itemdecl}

\begin{itemdescr}
\pnum
\effects
等效于:
\tcode{return basic_string_view<charT, traits>(*this).copy(s, n, pos);}
\begin{note}
此操作不会用空字符终止 \tcode{s}。
\end{note}
\end{itemdescr}

\rSec4[string.swap]{\tcode{basic_string::swap}}

\indexlibrarymember{swap}{basic_string}%
\begin{itemdecl}
constexpr void swap(basic_string& s)
  noexcept(allocator_traits<Allocator>::propagate_on_container_swap::value ||
           allocator_traits<Allocator>::is_always_equal::value);
\end{itemdecl}

\begin{itemdescr}
\pnum
\expects
\tcode{allocator_traits<Allocator>::propagate_on_container_swap::value} 为 \tcode{true}
或
\tcode{get_allocator() == s.get_allocator()}。

\pnum
\ensures
\tcode{*this}
含有原本在 \tcode{s} 中的同一字符序列,
\tcode{s} 含有原本在
\tcode{*this} 中的同一字符序列。

\pnum
\throws
无。

\pnum
\complexity
常数时间。
\end{itemdescr}

\rSec3[string.ops]{字符串操作}

\rSec4[string.accessors]{访问器}

\indexlibrarymember{c_str}{basic_string}%
\indexlibrarymember{data}{basic_string}%
\begin{itemdecl}
constexpr const charT* c_str() const noexcept;
constexpr const charT* data() const noexcept;
\end{itemdecl}

\begin{itemdescr}
\pnum
\returns
返回指针 \tcode{p},使得对于 \crange{0}{size()} 中的每个 \tcode{i},均有 \tcode{p + i == addressof(operator[](i))}。

\pnum
\complexity
常数时间。

\pnum
\remarks
程序不得修改存储于字符数组中的任何值;否则行为未定义。
\end{itemdescr}

\indexlibrarymember{data}{basic_string}%
\begin{itemdecl}
constexpr charT* data() noexcept;
\end{itemdecl}

\begin{itemdescr}
\pnum
\returns
返回指针 \tcode{p},使得对于 \crange{0}{size()} 中的每个 \tcode{i},均有 \tcode{p + i == addressof(operator[](i))}。

\pnum
\complexity
常数时间。

\pnum
\remarks
程序不得将存储在 \tcode{p + size()} 处的值修改为除 \tcode{charT()} 以外的任何值;否则行为未定义。
\end{itemdescr}

\indexlibrarymember{operator basic_string_view}{basic_string}%
\begin{itemdecl}
constexpr operator basic_string_view<charT, traits>() const noexcept;
\end{itemdecl}

\begin{itemdescr}
\pnum
\effects
等价于:
\tcode{return basic_string_view<charT, traits>(data(), size());}
\end{itemdescr}

\indexlibrarymember{get_allocator}{basic_string}%
\begin{itemdecl}
constexpr allocator_type get_allocator() const noexcept;
\end{itemdecl}

\begin{itemdescr}
\pnum
\returns
返回用于构造字符串的 \tcode{Allocator} 对象副本,或者如果该分配器已被替换,则返回最新替换的副本。
\end{itemdescr}

\rSec4[string.find]{搜索}
\pnum
\indexlibrarymember{find}{basic_string}%
\indexlibrarymember{rfind}{basic_string}%
\indexlibrarymember{find_first_of}{basic_string}%
\indexlibrarymember{find_last_of}{basic_string}%
\indexlibrarymember{find_first_not_of}{basic_string}%
\indexlibrarymember{find_last_not_of}{basic_string}%
令 \placeholder{F} 为 \tcode{find}、\tcode{rfind}、\tcode{find_first_of}、\tcode{find_last_of}、\tcode{find_first_not_of} 和 \tcode{find_last_not_of} 其中之一。

\begin{itemize}
\item
形式如
\begin{codeblock}
constexpr size_type @\placeholder{F}@(const basic_string& str, size_type pos) const noexcept;
\end{codeblock}
的每个成员函数具有等价于:
\tcode{return \placeholder{F}(basic_string_view<charT, traits>(str), pos);}
的效果。

\item
形式如
\begin{codeblock}
constexpr size_type @\placeholder{F}@(const charT* s, size_type pos) const;
\end{codeblock}
的每个成员函数具有等价于:
\tcode{return \placeholder{F}(basic_string_view<charT, traits>(s), pos);}
的效果。

\item
形式如
\begin{codeblock}
constexpr size_type @\placeholder{F}@(const charT* s, size_type pos, size_type n) const;
\end{codeblock}
的每个成员函数具有等价于:
\tcode{return \placeholder{F}(basic_string_view<charT, traits>(s, n), pos);}
的效果。

\item
形式如
\begin{codeblock}
constexpr size_type @\placeholder{F}@(charT c, size_type pos) const noexcept;
\end{codeblock}
的每个成员函数具有等价于:
\begin{codeblock}
return @\placeholder{F}@(basic_string_view<charT, traits>(addressof(c), 1), pos);
\end{codeblock}
的效果。
\end{itemize}

\indexlibrarymember{find}{basic_string}%
\indexlibrarymember{rfind}{basic_string}%
\indexlibrarymember{find_first_of}{basic_string}%
\indexlibrarymember{find_last_of}{basic_string}%
\indexlibrarymember{find_first_not_of}{basic_string}%
\indexlibrarymember{find_last_not_of}{basic_string}%
\begin{itemdecl}
template<class T>
  constexpr size_type find(const T& t, size_type pos = 0) const noexcept(@\seebelow@);
template<class T>
  constexpr size_type rfind(const T& t, size_type pos = npos) const noexcept(@\seebelow@);
template<class T>
  constexpr size_type find_first_of(const T& t, size_type pos = 0) const noexcept(@\seebelow@);
template<class T>
  constexpr size_type find_last_of(const T& t, size_type pos = npos) const noexcept(@\seebelow@);
template<class T>
  constexpr size_type find_first_not_of(const T& t, size_type pos = 0) const noexcept(@\seebelow@);
template<class T>
  constexpr size_type find_last_not_of(const T& t, size_type pos = npos) const noexcept(@\seebelow@);
\end{itemdecl}

\begin{itemdescr}
\pnum
\constraints
\begin{itemize}
\item
\tcode{is_convertible_v<const T\&, basic_string_view<charT, traits>>} 为
\tcode{true} 且
\item
\tcode{is_convertible_v<const T\&, const charT*>} 为
\tcode{false}。
\end{itemize}

\pnum
\effects
令 \placeholder{G} 为该函数名。
等价于:
\begin{codeblock}
basic_string_view<charT, traits> s = *this, sv = t;
return s.@\placeholder{G}@(sv, pos);
\end{codeblock}

\pnum
\remarks
该异常规范等价于
\tcode{is_nothrow_convertible_v<const T\&, basic_string_view<charT, traits>>}。
\end{itemdescr}

\rSec4[string.substr]{\tcode{basic_string::substr}}

\indexlibrarymember{substr}{basic_string}%
\begin{itemdecl}
constexpr basic_string substr(size_type pos = 0, size_type n = npos) const &;
\end{itemdecl}

\begin{itemdescr}
\pnum
\effects
等价于:\tcode{return basic_string(*this, pos, n);}
\end{itemdescr}

\indexlibrarymember{substr}{basic_string}%
\begin{itemdecl}
constexpr basic_string substr(size_type pos = 0, size_type n = npos) &&;
\end{itemdecl}

\begin{itemdescr}
\pnum
\effects
等价于:\tcode{return basic_string(std::move(*this), pos, n);}
\end{itemdescr}

\indexlibrarymember{subview}{basic_string}%
\begin{itemdecl}
constexpr basic_string_view<charT, traits> subview(size_type pos = 0, size_type n = npos) const;
\end{itemdecl}

\begin{itemdescr}
\pnum
\effects
等价于:\tcode{return basic_string_view<charT, traits>(*this).subview(pos, n);}
\end{itemdescr}

\rSec4[string.compare]{\tcode{basic_string::compare}}

\indexlibrarymember{compare}{basic_string}%
\begin{itemdecl}
template<class T>
  constexpr int compare(const T& t) const noexcept(@\seebelow@);
\end{itemdecl}

\begin{itemdescr}
\pnum
\constraints
\begin{itemize}
\item
\tcode{is_convertible_v<const T\&, basic_string_view<charT, traits>>} 为
\tcode{true} 且
\item
\tcode{is_convertible_v<const T\&, const charT*>} 为
\tcode{false}。
\end{itemize}

\pnum
\effects
等价于:\tcode{return basic_string_view<charT, traits>(*this).compare(t);}

\pnum
\remarks
异常规范等价于
\tcode{is_nothrow_convertible_v<const T\&, basic_string_view<charT, traits>>}。
\end{itemdescr}

\indexlibrarymember{compare}{basic_string}%
\begin{itemdecl}
template<class T>
  constexpr int compare(size_type pos1, size_type n1, const T& t) const;
\end{itemdecl}

\begin{itemdescr}
\pnum
\constraints
\begin{itemize}
\item
\tcode{is_convertible_v<const T\&, basic_string_view<charT, traits>>} 为
\tcode{true} 且
\item
\tcode{is_convertible_v<const T\&, const charT*>} 为
\tcode{false}。
\end{itemize}

\pnum
\effects
等价于:
\begin{codeblock}
return basic_string_view<charT, traits>(*this).substr(pos1, n1).compare(t);
\end{codeblock}
\end{itemdescr}

\indexlibrarymember{compare}{basic_string}%
\begin{itemdecl}
template<class T>
  constexpr int compare(size_type pos1, size_type n1, const T& t,
                        size_type pos2, size_type n2 = npos) const;
\end{itemdecl}

\begin{itemdescr}
\pnum
\constraints
\begin{itemize}
\item
\tcode{is_convertible_v<const T\&, basic_string_view<charT, traits>>} 为
\tcode{true},且
\item
\tcode{is_convertible_v<const T\&, const charT*>} 为
\tcode{false}。
\end{itemize}

\pnum
\effects
等价于:
\begin{codeblock}
basic_string_view<charT, traits> s = *this, sv = t;
return s.substr(pos1, n1).compare(sv.substr(pos2, n2));
\end{codeblock}
\end{itemdescr}

\indexlibrarymember{compare}{basic_string}%
\begin{itemdecl}
constexpr int compare(const basic_string& str) const noexcept;
\end{itemdecl}

\begin{itemdescr}
\pnum
\effects
等价于:
\tcode{return compare(basic_string_view<charT, traits>(str));}
\end{itemdescr}

\indexlibrarymember{compare}{basic_string}%
\begin{itemdecl}
constexpr int compare(size_type pos1, size_type n1, const basic_string& str) const;
\end{itemdecl}

\begin{itemdescr}
\pnum
\effects
等价于:
\tcode{return compare(pos1, n1, basic_string_view<charT, traits>(str));}
\end{itemdescr}

\indexlibrarymember{compare}{basic_string}%
\begin{itemdecl}
constexpr int compare(size_type pos1, size_type n1, const basic_string& str,
                      size_type pos2, size_type n2 = npos) const;
\end{itemdecl}

\begin{itemdescr}
\pnum
\effects
等价于:
\begin{codeblock}
return compare(pos1, n1, basic_string_view<charT, traits>(str), pos2, n2);
\end{codeblock}
\end{itemdescr}

\indexlibrarymember{compare}{basic_string}%
\begin{itemdecl}
constexpr int compare(const charT* s) const;
\end{itemdecl}

\begin{itemdescr}
\pnum
\effects
等价于:
\begin{codeblock}
return compare(basic_string_view<charT, traits>(s));
\end{codeblock}
\end{itemdescr}

\begin{itemdescr}
\pnum
\effects
等价于:
\tcode{return compare(basic_string_view<charT, traits>(s));}
\end{itemdescr}

\indexlibrarymember{compare}{basic_string}%
\begin{itemdecl}
constexpr int compare(size_type pos, size_type n1, const charT* s) const;
\end{itemdecl}

\begin{itemdescr}
\pnum
\effects
等价于:\tcode{return compare(pos, n1, basic_string_view<charT, traits>(s));}
\end{itemdescr}

\indexlibrarymember{compare}{basic_string}%
\begin{itemdecl}
constexpr int compare(size_type pos, size_type n1, const charT* s, size_type n2) const;
\end{itemdecl}

\begin{itemdescr}
\pnum
\effects
等价于:\tcode{return compare(pos, n1, basic_string_view<charT, traits>(s, n2));}
\end{itemdescr}

\begin{itemdescr}
\pnum
\effects
等价于: \tcode{return compare(pos, n1, basic_string_view<charT, traits>(s, n2));}
\end{itemdescr}

\rSec4[string.starts.with]{\tcode{basic_string::starts_with}}

\indexlibrarymember{starts_with}{basic_string}%
\begin{itemdecl}
constexpr bool starts_with(basic_string_view<charT, traits> x) const noexcept;
constexpr bool starts_with(charT x) const noexcept;
constexpr bool starts_with(const charT* x) const;
\end{itemdecl}

\begin{itemdescr}
\pnum
\effects
等价于:
\begin{codeblock}
return basic_string_view<charT, traits>(data(), size()).starts_with(x);
\end{codeblock}
\end{itemdescr}

\rSec4[string.ends.with]{\tcode{basic_string::ends_with}}

\indexlibrarymember{ends_with}{basic_string}%
\begin{itemdecl}
constexpr bool ends_with(basic_string_view<charT, traits> x) const noexcept;
constexpr bool ends_with(charT x) const noexcept;
constexpr bool ends_with(const charT* x) const;
\end{itemdecl}

\begin{itemdescr}
\pnum
\effects
等价于:
\begin{codeblock}
return basic_string_view<charT, traits>(data(), size()).ends_with(x);
\end{codeblock}
\end{itemdescr}

\rSec4[string.contains]{\tcode{basic_string::contains}}

\indexlibrarymember{contains}{basic_string}%
\begin{itemdecl}
constexpr bool contains(basic_string_view<charT, traits> x) const noexcept;
constexpr bool contains(charT x) const noexcept;
constexpr bool contains(const charT* x) const;
\end{itemdecl}

\begin{itemdescr}
\pnum
\effects
等价于:
\begin{codeblock}
return basic_string_view<charT, traits>(data(), size()).contains(x);
\end{codeblock}
\end{itemdescr}

\rSec2[string.nonmembers]{非成员函数}

\indexlibraryglobal{basic_string}

\rSec3[string.op.plus]{\tcode{operator+}}

\indexlibrarymember{operator+}{basic_string}%
\begin{itemdecl}
template<class charT, class traits, class Allocator>
  constexpr basic_string<charT, traits, Allocator>
    operator+(const basic_string<charT, traits, Allocator>& lhs,
              const basic_string<charT, traits, Allocator>& rhs);
template<class charT, class traits, class Allocator>
  constexpr basic_string<charT, traits, Allocator>
    operator+(const basic_string<charT, traits, Allocator>& lhs, const charT* rhs);
\end{itemdecl}

\begin{itemdescr}
\pnum
\effects
等效于:
\begin{codeblock}
basic_string<charT, traits, Allocator> r = lhs;
r.append(rhs);
return r;
\end{codeblock}
\end{itemdescr}

\indexlibrarymember{operator+}{basic_string}%
\begin{itemdecl}
template<class charT, class traits, class Allocator>
  constexpr basic_string<charT, traits, Allocator>
    operator+(basic_string<charT, traits, Allocator>&& lhs,
              const basic_string<charT, traits, Allocator>& rhs);
template<class charT, class traits, class Allocator>
  constexpr basic_string<charT, traits, Allocator>
    operator+(basic_string<charT, traits, Allocator>&& lhs, const charT* rhs);
\end{itemdecl}

\begin{itemdescr}
\pnum
\effects
等价于:
\begin{codeblock}
lhs.append(rhs);
return std::move(lhs);
\end{codeblock}
\end{itemdescr}

\indexlibrarymember{operator+}{basic_string}%
\begin{itemdecl}
template<class charT, class traits, class Allocator>
  constexpr basic_string<charT, traits, Allocator>
    operator+(basic_string<charT, traits, Allocator>&& lhs,
              basic_string<charT, traits, Allocator>&& rhs);
\end{itemdecl}

\begin{itemdescr}
\pnum
\effects
等价于:
\begin{codeblock}
lhs.append(rhs);
return std::move(lhs);
\end{codeblock}
不同之处在于 \tcode{lhs} 与 \tcode{rhs}
均被置于合法但未指定的状态。
\begin{note}
若 \tcode{lhs} 与 \tcode{rhs} 拥有相等的分配器,
实现可从任意一方移动。
\end{note}
\end{itemdescr}

\indexlibrarymember{operator+}{basic_string}%
\begin{itemdecl}
template<class charT, class traits, class Allocator>
  constexpr basic_string<charT, traits, Allocator>
    operator+(const basic_string<charT, traits, Allocator>& lhs,
              basic_string<charT, traits, Allocator>&& rhs);
template<class charT, class traits, class Allocator>
  constexpr basic_string<charT, traits, Allocator>
    operator+(const charT* lhs, basic_string<charT, traits, Allocator>&& rhs);
\end{itemdecl}

\begin{itemdescr}
\pnum
\effects
效果等同于:
\begin{codeblock}
rhs.insert(0, lhs);
return std::move(rhs);
\end{codeblock}
\end{itemdescr}

\indexlibrarymember{operator+}{basic_string}%
\begin{itemdecl}
template<class charT, class traits, class Allocator>
  constexpr basic_string<charT, traits, Allocator>
    operator+(const charT* lhs, const basic_string<charT, traits, Allocator>& rhs);
\end{itemdecl}

\begin{itemdescr}
\pnum
\effects
等效于:
\begin{codeblock}
basic_string<charT, traits, Allocator> r = rhs;
r.insert(0, lhs);
return r;
\end{codeblock}
\end{itemdescr}

\indexlibrarymember{operator+}{basic_string}%
\begin{itemdecl}
template<class charT, class traits, class Allocator>
  constexpr basic_string<charT, traits, Allocator>
    operator+(charT lhs, const basic_string<charT, traits, Allocator>& rhs);
\end{itemdecl}

\begin{itemdescr}
\pnum
\effects
等效于:
\begin{codeblock}
basic_string<charT, traits, Allocator> r = rhs;
r.insert(0, lhs);
return r;
\end{codeblock}
\end{itemdescr}

\begin{itemdescr}
\pnum
\effects
等价于:
\begin{codeblock}
basic_string<charT, traits, Allocator> r = rhs;
r.insert(r.begin(), lhs);
return r;
\end{codeblock}
\end{itemdescr}

\indexlibrarymember{operator+}{basic_string}%
\begin{itemdecl}
template<class charT, class traits, class Allocator>
  constexpr basic_string<charT, traits, Allocator>
    operator+(charT lhs, basic_string<charT, traits, Allocator>&& rhs);
\end{itemdecl}

\begin{itemdescr}
\pnum
\effects
等价于:
\begin{codeblock}
rhs.insert(rhs.begin(), lhs);
return std::move(rhs);
\end{codeblock}
\end{itemdescr}

\indexlibrarymember{operator+}{basic_string}%
\begin{itemdecl}
template<class charT, class traits, class Allocator>
  constexpr basic_string<charT, traits, Allocator>
    operator+(const basic_string<charT, traits, Allocator>& lhs, charT rhs);
\end{itemdecl}

\begin{itemdescr}
\pnum
\effects
等价于:
\begin{codeblock}
basic_string<charT, traits, Allocator> r = lhs;
r.push_back(rhs);
return r;
\end{codeblock}
\end{itemdescr}

\indexlibrarymember{operator+}{basic_string}%
\begin{itemdecl}
template<class charT, class traits, class Allocator>
  constexpr basic_string<charT, traits, Allocator>
    operator+(basic_string<charT, traits, Allocator>&& lhs, charT rhs);
\end{itemdecl}

\begin{itemdescr}
\pnum
\effects
等价于:
\begin{codeblock}
lhs.push_back(rhs);
return std::move(lhs);
\end{codeblock}
\end{itemdescr}

\indexlibrarymember{operator+}{basic_string}%
\begin{itemdecl}
template<class charT, class traits, class Allocator>
  constexpr basic_string<charT, traits, Allocator>
    operator+(const basic_string<charT, traits, Allocator>& lhs,
              type_identity_t<basic_string_view<charT, traits>> rhs);
\end{itemdecl}

\begin{itemdescr}
\pnum
\effects
等价于:
\begin{codeblock}
basic_string<charT, traits, Allocator> r = lhs;
r.append(rhs);
return r;
\end{codeblock}
\end{itemdescr}

\indexlibrarymember{operator+}{basic_string}%
\begin{itemdecl}
template<class charT, class traits, class Allocator>
  constexpr basic_string<charT, traits, Allocator>
    operator+(basic_string<charT, traits, Allocator>&& lhs,
              type_identity_t<basic_string_view<charT, traits>> rhs);
\end{itemdecl}

\begin{itemdescr}
\pnum
等价于:
\begin{codeblock}
lhs.append(rhs);
return std::move(lhs);
\end{codeblock}
\end{itemdescr}

\indexlibrarymember{operator+}{basic_string}%
\begin{itemdecl}
template<class charT, class traits, class Allocator>
  constexpr basic_string<charT, traits, Allocator>
    operator+(type_identity_t<basic_string_view<charT, traits>> lhs,
              const basic_string<charT, traits, Allocator>& rhs);
\end{itemdecl}

\begin{itemdescr}
\pnum
等价于:
\begin{codeblock}
return basic_string<charT, traits, Allocator>(lhs) += rhs;
\end{codeblock}
\end{itemdescr}

\begin{itemdescr}
\pnum
等价于:
\begin{codeblock}
basic_string<charT, traits, Allocator> r = rhs;
r.insert(0, lhs);
return r;
\end{codeblock}
\end{itemdescr}

\indexlibrarymember{operator+}{basic_string}%
\begin{itemdecl}
template<class charT, class traits, class Allocator>
  constexpr basic_string<charT, traits, Allocator>
    operator+(type_identity_t<basic_string_view<charT, traits>> lhs,
              basic_string<charT, traits, Allocator>&& rhs);
\end{itemdecl}

\begin{itemdescr}
\pnum
等价于:
\begin{codeblock}
rhs.insert(0, lhs);
return std::move(rhs);
\end{codeblock}
\end{itemdescr}

\pnum
\begin{note}
将 \tcode{type_identity_t} 的特化用作参数类型,确保类型为 \tcode{basic_string<charT, traits, Allocator>}
的对象可与具有隐式转换到 \tcode{basic_string_view<charT, traits>}\iref{over.match.oper} 的类型 \tcode{T} 的对象连接。
\end{note}

\rSec3[string.cmp]{非成员比较运算符函数}
\begin{itemdecl}
template<class charT, class traits, class Allocator>
  constexpr bool
    operator==(const basic_string<charT, traits, Allocator>& lhs,
               const basic_string<charT, traits, Allocator>& rhs) noexcept;
template<class charT, class traits, class Allocator>
  constexpr bool operator==(const basic_string<charT, traits, Allocator>& lhs,
                            const charT* rhs);

template<class charT, class traits, class Allocator>
  constexpr @\seebelow@ operator<=>(const basic_string<charT, traits, Allocator>& lhs,
            @\itcorr@                      const basic_string<charT, traits, Allocator>& rhs) noexcept;
template<class charT, class traits, class Allocator>
  constexpr @\seebelow@ operator<=>(const basic_string<charT, traits, Allocator>& lhs,
            @\itcorr@                      const charT* rhs);
\end{itemdecl}
\begin{itemdescr}
\pnum
\effects
令 \tcode{\placeholder{op}} 为该运算符。
等价于:
\begin{codeblock}
return basic_string_view<charT, traits>(lhs) @\placeholder{op}@ basic_string_view<charT, traits>(rhs);
\end{codeblock}
\end{itemdescr}

\rSec3[string.special]{\tcode{swap}}

\indexlibrarymember{swap}{basic_string}%
\begin{itemdecl}
template<class charT, class traits, class Allocator>
  constexpr void
    swap(basic_string<charT, traits, Allocator>& lhs,
         basic_string<charT, traits, Allocator>& rhs)
      noexcept(noexcept(lhs.swap(rhs)));
\end{itemdecl}

\begin{itemdescr}
\pnum
\effects
等价于 \tcode{lhs.swap(rhs)}。
\end{itemdescr}

\rSec3[string.io]{插入器与提取器}

\indexlibrarymember{operator>>}{basic_string}%
\begin{itemdecl}
template<class charT, class traits, class Allocator>
  basic_istream<charT, traits>&
    operator>>(basic_istream<charT, traits>& is, basic_string<charT, traits, Allocator>& str);
\end{itemdecl}

\begin{itemdescr}
\pnum
\effects
行为如格式化输入函数\iref{istream.formatted.reqmts}所述。
构造 \tcode{sentry} 对象后,
若 \tcode{sentry} 对象转换为 \tcode{bool} 类型值得到 \tcode{true},
则调用 \tcode{str.erase()},
随后从 \tcode{is} 提取字符并将其追加到 \tcode{str},如同调用 \tcode{str.append(1, c)}。
若 \tcode{is.width()} 大于零,则追加的字符最大数量 \tcode{n} 为 \tcode{is.width()};
否则 \tcode{n} 为 \tcode{str.max_size()}。
字符将被提取并追加直至发生以下任一情况:
\begin{itemize}
\item
存储了
\textit{n}
个字符;
\item
输入序列出现文件结束;
\item
对下一个可用输入字符
\textit{c}
有
\tcode{isspace(c, is.getloc())}
为
\tcode{true}。
\end{itemize}

\pnum
最后一个字符(若存在)被提取后,
调用 \tcode{is.width(0)} 并销毁
\tcode{sentry}
对象。

\pnum
若函数未提取任何字符,则在调用 \tcode{setstate} 前,
在输入函数的局部错误状态中设置 \tcode{ios_base::failbit}。

\pnum
\returns
\tcode{is}。
\end{itemdescr}

\indexlibrarymember{operator<<}{basic_string}%
\begin{itemdecl}
template<class charT, class traits, class Allocator>
  basic_ostream<charT, traits>&
    operator<<(basic_ostream<charT, traits>& os,
               const basic_string<charT, traits, Allocator>& str);
\end{itemdecl}

\begin{itemdescr}
\pnum
\effects
等价于:\tcode{return os << basic_string_view<charT, traits>(str);}
\end{itemdescr}

\indexlibrarymember{getline}{basic_string}%
\begin{itemdecl}
template<class charT, class traits, class Allocator>
  basic_istream<charT, traits>&
    getline(basic_istream<charT, traits>& is,
            basic_string<charT, traits, Allocator>& str,
            charT delim);
template<class charT, class traits, class Allocator>
  basic_istream<charT, traits>&
    getline(basic_istream<charT, traits>&& is,
            basic_string<charT, traits, Allocator>& str,
            charT delim);
\end{itemdecl}

\begin{itemdescr}
\pnum
\effects
从输入流 \tcode{is} 读取字符存入 \tcode{str} 直至遇到分隔符 \tcode{delim}、到达流尾或发生错误。若 \tcode{delim} 被读取,将其提取但不存入 \tcode{str}。

\pnum
\ensures
若未读取任何字符(立即遇到文件尾或错误),则调用 \tcode{str.erase()};否则 \tcode{str} 为读取的字符序列。

\pnum
\returns
\tcode{is}。
\end{itemdescr}

\begin{itemdescr}
\pnum
\effects
表现为一个未格式化输入函数\iref{istream.unformatted},除了它不影响后续对
\tcode{basic_istream<>::gcount()} 调用所返回的值。
在构造一个 \tcode{sentry} 对象之后,
如果该 \tcode{sentry} 对象在转换为类型 \tcode{bool} 的值时返回 \tcode{true},
则调用 \tcode{str.erase()},
然后从 \tcode{is} 中提取字符并附加它们到 \tcode{str},
如同通过调用
\tcode{str.append(1, c)}
直至以下任意情况发生:
\begin{itemize}
\item
输入序列中出现文件结束;
\item
对于下一个可用的输入字符
\textit{c},
满足
\tcode{traits::eq(c, delim)}
(在此情况下,
\textit{c}
被提取但不被附加);
\item
存储了
\tcode{str.max_size()}
个字符
(在此情况下,
\tcode{ios_base::failbit} 被设置到输入函数的局部错误状态中)。
\end{itemize}

\pnum
条件按照所示顺序测试。
在任何情况下,
在最后一个字符被提取后,
\tcode{sentry}
对象被销毁。

\pnum
如果函数未提取任何字符,
则在调用 \tcode{setstate} 前,
\tcode{ios_base::failbit} 被设置到输入函数的局部错误状态中。

\pnum
\returns
\tcode{is}.
\end{itemdescr}

\indexlibrarymember{getline}{basic_string}%
\begin{itemdecl}
template<class charT, class traits, class Allocator>
  basic_istream<charT, traits>&
    getline(basic_istream<charT, traits>& is,
            basic_string<charT, traits, Allocator>& str);
template<class charT, class traits, class Allocator>
  basic_istream<charT, traits>&
    getline(basic_istream<charT, traits>&& is,
            basic_string<charT, traits, Allocator>& str);
\end{itemdecl}

\begin{itemdescr}
\pnum
\returns
\tcode{getline(is, str, is.widen('\textbackslash n'))}.
\end{itemdescr}

\rSec3[string.erasure]{擦除}

\indexlibrarymember{erase}{basic_string}%
\begin{itemdecl}
template<class charT, class traits, class Allocator, class U = charT>
  constexpr typename basic_string<charT, traits, Allocator>::size_type
    erase(basic_string<charT, traits, Allocator>& c, const U& value);
\end{itemdecl}

\begin{itemdescr}
\pnum
\effects
等效于:
\begin{codeblock}
auto it = remove(c.begin(), c.end(), value);
auto r = distance(it, c.end());
c.erase(it, c.end());
return r;
\end{codeblock}
\end{itemdescr}

\indexlibrarymember{erase_if}{basic_string}%
\begin{itemdecl}
template<class charT, class traits, class Allocator, class Predicate>
  constexpr typename basic_string<charT, traits, Allocator>::size_type
    erase_if(basic_string<charT, traits, Allocator>& c, Predicate pred);
\end{itemdecl}

\begin{itemdescr}
\pnum
\effects
等效于:
\begin{codeblock}
auto it = remove_if(c.begin(), c.end(), pred);
auto r = distance(it, c.end());
c.erase(it, c.end());
return r;
\end{codeblock}
\end{itemdescr}

\rSec2[string.conversions]{数值转换}

\indexlibraryglobal{stoi}%
\indexlibraryglobal{stol}%
\indexlibraryglobal{stoul}%
\indexlibraryglobal{stoll}%
\indexlibraryglobal{stoull}%
\begin{itemdecl}
int stoi(const string& str, size_t* idx = nullptr, int base = 10);
long stol(const string& str, size_t* idx = nullptr, int base = 10);
unsigned long stoul(const string& str, size_t* idx = nullptr, int base = 10);
long long stoll(const string& str, size_t* idx = nullptr, int base = 10);
unsigned long long stoull(const string& str, size_t* idx = nullptr, int base = 10);
\end{itemdecl}

\begin{itemdescr}
\pnum
\effects
前两个函数调用 \tcode{strtol(str.c_str(), ptr, base)},
而最后三个函数分别调用 \tcode{strtoul(str.c_str(), ptr, base)}、
\tcode{strtoll(str.c_str(), ptr, base)} 与 \tcode{strtoull(\brk{}str.c_str(), ptr,
base)}。每个函数都返回转换后的结果(如果存在)。参数 \tcode{ptr} 指定指向函数内部对象的指针,
该指针用于确定在 \tcode{*idx} 处存储何值。若函数不抛出异常且 \tcode{idx != nullptr},
则该函数在 \tcode{*idx} 中存储 \tcode{str} 中第一个未被转换的元素的索引。

\pnum
\returns
转换后的结果。

\pnum
\throws
若 \tcode{strtol}、\tcode{strtoul}、\tcode{strtoll} 或 \tcode{strtoull} 报告无法执行转换,
则抛出 \tcode{invalid_argument}。若 \tcode{strtol}、\tcode{strtoul}、\tcode{strtoll} 或 \tcode{strtoull}
将 \tcode{errno} 设为 \tcode{ERANGE},或转换后的值超出返回类型可表示值的范围,则抛出 \tcode{out_of_range}。
\end{itemdescr}

\indexlibraryglobal{stof}%
\indexlibraryglobal{stod}%
\indexlibraryglobal{stold}%
\begin{itemdecl}
float stof(const string& str, size_t* idx = nullptr);
double stod(const string& str, size_t* idx = nullptr);
long double stold(const string& str, size_t* idx = nullptr);
\end{itemdecl}

\begin{itemdescr}
\pnum
\effects
这些函数分别调用
\tcode{strtof(str.c_str(), ptr)}、\tcode{strtod(str.c_str(), ptr)} 以及
\tcode{strtold(\brk{}str.c_str(), ptr)}。每个函数返回转换结果(若存在)。参数 \tcode{ptr} 指定一个指向函数内部对象的指针,该指针用于确定在 \tcode{*idx} 处存储什么。若函数未抛出异常且 \tcode{idx != nullptr},则该函数在 \tcode{*idx} 中存储 \tcode{str} 的第一个未转换元素的索引。

\pnum
\returns
转换后的结果。

\pnum
\throws
若 \tcode{strtof}、\tcode{strtod} 或 \tcode{strtold} 报告无法执行转换,则抛出 \tcode{invalid_argument}。若 \tcode{strtof}、\tcode{strtod} 或 \tcode{strtold} 将 \tcode{errno} 设置为 \tcode{ERANGE},或者转换后的值超出返回类型可表示值的范围,则抛出 \tcode{out_of_range}。
\end{itemdescr}

\indexlibraryglobal{to_string}%
\begin{itemdecl}
string to_string(int val);
string to_string(unsigned val);
string to_string(long val);
string to_string(unsigned long val);
string to_string(long long val);
string to_string(unsigned long long val);
string to_string(float val);
string to_string(double val);
string to_string(long double val);
\end{itemdecl}

\begin{itemdescr}
\pnum
\returns
\tcode{format("\{\}", val)}.
\end{itemdescr}

\indexlibraryglobal{stoi}%
\indexlibraryglobal{stol}%
\indexlibraryglobal{stoul}%
\indexlibraryglobal{stoll}%
\indexlibraryglobal{stoull}%
\begin{itemdecl}
int stoi(const wstring& str, size_t* idx = nullptr, int base = 10);
long stol(const wstring& str, size_t* idx = nullptr, int base = 10);
unsigned long stoul(const wstring& str, size_t* idx = nullptr, int base = 10);
long long stoll(const wstring& str, size_t* idx = nullptr, int base = 10);
unsigned long long stoull(const wstring& str, size_t* idx = nullptr, int base = 10);
\end{itemdecl}

\begin{itemdescr}
\pnum
\effects
前两个函数调用 \tcode{wcstol(str.c_str(), ptr, base)},
后三个函数分别调用 \tcode{wcstoul(str.c_str(), ptr, base)}、\tcode{wcstoll(str.c_str(), ptr, base)} 和 \tcode{wcstoull(\brk{}str.c_str(), ptr, base)}。每个函数返回转换后的结果(若存在)。参数 \tcode{ptr} 指定一个指向函数内部对象的指针,该对象用于确定在 \tcode{*idx} 处存储的内容。如果函数未抛出异常且 \tcode{idx != nullptr},则函数将 \tcode{str} 中第一个未转换元素的下标存储到 \tcode{*idx} 中。

\pnum
\returns
转换后的结果。

\pnum
\throws
如果 \tcode{wcstol}、\tcode{wcstoul}、\tcode{wcstoll} 或 \tcode{wcstoull} 报告无法执行转换,则抛出 \tcode{invalid_argument}。如果转换后的值超出返回类型的可表示值范围,则抛出 \tcode{out_of_range}。
\end{itemdescr}

\indexlibraryglobal{stof}%
\indexlibraryglobal{stod}%
\indexlibraryglobal{stold}%
\begin{itemdecl}
float stof(const wstring& str, size_t* idx = nullptr);
double stod(const wstring& str, size_t* idx = nullptr);
long double stold(const wstring& str, size_t* idx = nullptr);
\end{itemdecl}

\begin{itemdescr}
\pnum
\effects
这些函数分别调用 \tcode{wcstof(str.c\_str(), ptr)}、\tcode{wcstod(str.c\_str(), ptr)} 和 \tcode{wcstold(\brk{}str.c\_str(), ptr)}。每个函数返回转换结果,如果有的话。实参 \tcode{ptr} 指向函数内部用于确定在 \tcode{*idx} 处存储何物的对象的指针。如果该函数未抛出异常且 \tcode{idx != nullptr},则该函数将 \tcode{str} 的首个未转换元素的下标存储于 \tcode{*idx}。

\pnum
\returns
转换结果。

\pnum
\throws
如果 \tcode{wcstof}、\tcode{wcstod} 或 \tcode{wcstold} 报告无法执行转换,则抛出 \tcode{invalid\_argument}。如果 \tcode{wcstof}、\tcode{wcstod} 或 \tcode{wcstold} 将 \tcode{errno} 设为 \tcode{ERANGE},则抛出 \tcode{out\_of\_range}。
\end{itemdescr}

\indexlibraryglobal{to_wstring}%
\begin{itemdecl}
wstring to_wstring(int val);
wstring to_wstring(unsigned val);
wstring to_wstring(long val);
wstring to_wstring(unsigned long val);
wstring to_wstring(long long val);
wstring to_wstring(unsigned long long val);
wstring to_wstring(float val);
wstring to_wstring(double val);
wstring to_wstring(long double val);
\end{itemdecl}

\begin{itemdescr}
\pnum
\returns
\tcode{format(L"\{\}", val)}.
\end{itemdescr}

\rSec2[basic.string.hash]{哈希支持}

\indexlibrarymember{hash}{basic_string}%
\indexlibrarymember{hash}{string}%
\indexlibrarymember{hash}{u8string}%
\indexlibrarymember{hash}{u16string}%
\indexlibrarymember{hash}{u32string}%
\indexlibrarymember{hash}{wstring}%
\begin{itemdecl}
template<class A> struct hash<basic_string<char, char_traits<char>, A>>;
template<class A> struct hash<basic_string<char8_t, char_traits<char8_t>, A>>;
template<class A> struct hash<basic_string<char16_t, char_traits<char16_t>, A>>;
template<class A> struct hash<basic_string<char32_t, char_traits<char32_t>, A>>;
template<class A> struct hash<basic_string<wchar_t, char_traits<wchar_t>, A>>;
\end{itemdecl}

\begin{itemdescr}
\pnum
若 \tcode{S} 属于这些字符串类型之一,
\tcode{SV} 属于对应的字符串视图类型,且
\tcode{s} 属于类型为 \tcode{S} 的对象,
则 \tcode{hash<S>()(s) == hash<SV>()(SV(s))}。
\end{itemdescr}

\rSec2[basic.string.literals]{\tcode{basic_string} 字面量的后缀}

\indexlibrarymember{operator""""s}{string}%
\begin{itemdecl}
constexpr string operator""s(const char* str, size_t len);
\end{itemdecl}

\begin{itemdescr}
\pnum
\returns
\tcode{string\{str, len\}}.
\end{itemdescr}

\indexlibrarymember{operator""""s}{u8string}%
\begin{itemdecl}
constexpr u8string operator""s(const char8_t* str, size_t len);
\end{itemdecl}
\begin{itemdescr}
\pnum
\returns
\tcode{u8string\{str, len\}}.
\end{itemdescr}

\indexlibrarymember{operator""""s}{u16string}%
\begin{itemdecl}
constexpr u16string operator""s(const char16_t* str, size_t len);
\end{itemdecl}
\begin{itemdescr}
\pnum
\returns
\tcode{u16string\{str, len\}}.
\end{itemdescr}

\indexlibrarymember{operator""""s}{u32string}%
\begin{itemdecl}
constexpr u32string operator""s(const char32_t* str, size_t len);
\end{itemdecl}
\begin{itemdescr}
\pnum
\returns
\tcode{u32string\{str, len\}}.
\end{itemdescr}

\indexlibrarymember{operator""""s}{wstring}%
\begin{itemdecl}
constexpr wstring operator""s(const wchar_t* str, size_t len);
\end{itemdecl}
\begin{itemdescr}
\pnum
\returns
\tcode{wstring\{str, len\}}.
\end{itemdescr}

\pnum
\begin{note}
相同的后缀 \tcode{s} 用于表示秒的 \tcode{chrono::duration} 字面量,但不存在冲突,因为持续时间后缀应用于数字,而字符串字面量后缀应用于字符数组字面量。
\end{note}

\rSec1[c.strings]{空终止序列工具}

\rSec2[cstring.syn]{头文件 \tcode{<cstring>} 概要}

\indexlibraryglobal{memchr}%
\indexlibraryglobal{memcmp}%
\indexlibraryglobal{memcpy}%
\indexlibraryglobal{memmove}%
\indexlibraryglobal{memset}%
\indexlibraryglobal{size_t}%
\indexlibraryglobal{strcat}%
\indexlibraryglobal{strchr}%
\indexlibraryglobal{strcmp}%
\indexlibraryglobal{strcoll}%
\indexlibraryglobal{strcpy}%
\indexlibraryglobal{strcspn}%
\indexlibraryglobal{strerror}%
\indexlibraryglobal{strlen}%
\indexlibraryglobal{strncat}%
\indexlibraryglobal{strncmp}%
\indexlibraryglobal{strncpy}%
\indexlibraryglobal{strpbrk}%
\indexlibraryglobal{strrchr}%
\indexlibraryglobal{strspn}%
\indexlibraryglobal{strstr}%
\indexlibraryglobal{strtok}%
\indexlibraryglobal{strxfrm}%
\begin{codeblock}
#define __STDC_VERSION_STRING_H__ 202311L

namespace std {
  using size_t = @\textit{见 \ref{support.types.layout}}@;                                            // 独立实现

  void* memcpy(void* s1, const void* s2, size_t n);                     // 独立实现
  void* memccpy(void* s1, const void* s2, int c, size_t n);             // 独立实现
  void* memmove(void* s1, const void* s2, size_t n);                    // 独立实现
  char* strcpy(char* s1, const char* s2);                               // 独立实现
  char* strncpy(char* s1, const char* s2, size_t n);                    // 独立实现
  char* strdup(const char* s);
  char* strndup(const char* s, size_t size);
  char* strcat(char* s1, const char* s2);                               // 独立实现
  char* strncat(char* s1, const char* s2, size_t n);                    // 独立实现
  int memcmp(const void* s1, const void* s2, size_t n);                 // 独立实现
  int strcmp(const char* s1, const char* s2);                           // 独立实现
  int strcoll(const char* s1, const char* s2);
  int strncmp(const char* s1, const char* s2, size_t n);                // 独立实现
  size_t strxfrm(char* s1, const char* s2, size_t n);
  const void* memchr(const void* s, int c, size_t n);                   // 独立实现;见 \ref{library.c}
  void* memchr(void* s, int c, size_t n);                               // 独立实现;见 \ref{library.c}
  const char* strchr(const char* s, int c);                             // 独立实现;见 \ref{library.c}
  char* strchr(char* s, int c);                                         // 独立实现;见 \ref{library.c}
  size_t strcspn(const char* s1, const char* s2);                       // 独立实现
  const char* strpbrk(const char* s1, const char* s2);                  // 独立实现;见 \ref{library.c}
  char* strpbrk(char* s1, const char* s2);                              // 独立实现;见 \ref{library.c}
  const char* strrchr(const char* s, int c);                            // 独立实现;见 \ref{library.c}
  char* strrchr(char* s, int c);                                        // 独立实现;见 \ref{library.c}
  size_t strspn(const char* s1, const char* s2);                        // 独立实现
  const char* strstr(const char* s1, const char* s2);                   // 独立实现;见 \ref{library.c}
  char* strstr(char* s1, const char* s2);                               // 独立实现;见 \ref{library.c}
  char* strtok(char* s1, const char* s2);
  void* memset(void* s, int c, size_t n);                               // 独立实现
  void* memset_explicit(void* s, int c, size_t n);                      // 独立实现
  char* strerror(int errnum);
  size_t strlen(const char* s);                                         // 独立实现
}

#define NULL @\textit{见 \ref{support.types.nullptr}}@                                                // 独立实现
\end{codeblock}

\pnum
头文件 \libheaderdef{cstring} 的内容和含义与 C 标准库头文件 \libheader{string.h} 相同。
\pnum
不要求函数 \tcode{strerror} 和 \tcode{strtok} 避免数据竞争\iref{res.on.data.races}。
\pnum
\indextext{signal-safe!\idxcode{memcpy}}%
\indextext{signal-safe!\idxcode{memmove}}%
函数 \tcode{memcpy} 和 \tcode{memmove} 是信号安全的\iref{support.signal}。
这些函数中的每一个在复制字符序列到目标区域之前,都在目标存储区域中隐式创建对象\iref{intro.object}。
这些函数中的每一个都返回指向适当创建的对象的指针(若有),否则返回第一个参数的值。

\pnum
\begin{note}
函数
\tcode{strchr}、\tcode{strpbrk}、\tcode{strrchr}、\tcode{strstr} 与 \tcode{memchr}
在本文件中有不同的签名,
但它们的行为与 C 标准库中的相同\iref{library.c}。
\end{note}

\xrefc{7.26}

