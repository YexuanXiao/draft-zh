%!TEX root = std.tex
\rSec0[stmt]{语句}%
\indextext{语句|(}

\gramSec[gram.stmt]{语句}

\indextext{块(语句)|参见{语句,复合}}

\rSec1[stmt.pre]{引言}

\pnum
除非另有说明,否则语句按顺序执行\iref{intro.execution}。

\begin{bnf}
\nontermdef{statement}\br
    labeled-statement\br
    \opt{attribute-specifier-seq} expression-statement\br
    \opt{attribute-specifier-seq} compound-statement\br
    \opt{attribute-specifier-seq} selection-statement\br
    \opt{attribute-specifier-seq} iteration-statement\br
    \opt{attribute-specifier-seq} expansion-statement\br
    \opt{attribute-specifier-seq} jump-statement\br
    \opt{attribute-specifier-seq} assertion-statement\br
    declaration-statement\br
    \opt{attribute-specifier-seq} try-block
\end{bnf}

\begin{bnf}
\nontermdef{init-statement}\br
    expression-statement\br
    simple-declaration\br
    alias-declaration
\end{bnf}

\begin{bnf}
\nontermdef{condition}\br
    expression\br
    \opt{attribute-specifier-seq} decl-specifier-seq declarator brace-or-equal-initializer\br
    structured-binding-declaration initializer
\end{bnf}

\begin{bnf}
\nontermdef{for-range-declaration}\br
    \opt{attribute-specifier-seq} decl-specifier-seq declarator\br
    structured-binding-declaration
\end{bnf}

\begin{bnf}
\nontermdef{for-range-initializer}\br
    expr-or-braced-init-list
\end{bnf}

可选的 \grammarterm{attribute-specifier-seq} 属于相应的语句。
关于 \grammarterm{for-range-declaration} 中可选的 \grammarterm{attribute-specifier-seq},参见~\ref{dcl.meaning}。

\pnum
一个 \grammarterm{statement} 的\defn{子语句} 是以下之一:
\begin{itemize}
\item
  对于一个 \grammarterm{labeled-statement},是其 \grammarterm{statement},
\item
  对于一个 \grammarterm{compound-statement},是其 \grammarterm{statement-seq} 中的任何 \grammarterm{statement},
\item
  对于一个 \grammarterm{selection-statement},是其任何 \grammarterm{statement} 或 \grammarterm{compound-statement}(但不是其 \grammarterm{init-statement}),
\item
  对于一个 \grammarterm{iteration-statement},是其 \grammarterm{statement}(但不是 \grammarterm{init-statement}),或
\item
  对于一个 \grammarterm{expansion-statement},是其 \grammarterm{compound-statement}
  (但不是其 \grammarterm{init-statement})。
\end{itemize}
\begin{note}
一个 \grammarterm{lambda-expression} 的 \grammarterm{compound-statement}
不是该 \grammarterm{lambda-expression} 在词法上出现的(如果有)\grammarterm{statement} 的子语句。
\end{note}

\pnum
\indextext{语句!包围}%
一个 \grammarterm{statement} \tcode{S1} \defnx{包围}{包围语句}
一个 \grammarterm{statement} \tcode{S2},如果
\begin{itemize}
\item
  \tcode{S2} 是 \tcode{S1} 的一个子语句,
\item
  \tcode{S1} 是一个 \grammarterm{selection-statement}、
  \grammarterm{iteration-statement} 或 \grammarterm{expansion-statement},
  并且 \tcode{S2} 是 \tcode{S1} 的 \grammarterm{init-statement},
\item
  \tcode{S1} 是一个 \grammarterm{try-block} 并且 \tcode{S2}
  是其 \grammarterm{compound-statement} 或其 \grammarterm{handler} 的任何 \grammarterm{compound-statement},或者
\item
  \tcode{S1} 包围一个语句 \tcode{S3} 并且 \tcode{S3} 包围 \tcode{S2}。
\end{itemize}
\indextext{语句!被包围}%
一个语句 \tcode{S1} 是
\defnx{被包围于}{被包围于语句}
一个语句 \tcode{S2},如果
\tcode{S2} 包围 \tcode{S1}。

\pnum
\indextext{\idxgram{condition}{s}!规则}%
\grammarterm{condition} 的规则既适用于
\grammarterm{selection-statement}\iref{stmt.select} 也适用于
\keyword{for} 和 \keyword{while} 语句\iref{stmt.iter}。
如果一个 \grammarterm{structured-binding-declaration}
出现在一个 \grammarterm{condition} 中,
则该 \grammarterm{condition} 是一个结构化绑定声明\iref{dcl.pre}。
一个既不是 \grammarterm{expression} 也不是结构化绑定声明的
\grammarterm{condition} 是一个声明\iref{dcl}。
该 \grammarterm{declarator} 不得
指定一个函数或数组。该 \grammarterm{decl-specifier-seq} 不得
定义类或枚举。如果 \keyword{auto} \grammarterm{type-specifier} 出现在
\grammarterm{decl-specifier-seq} 中,
则被声明标识符的类型从初始化器中推导,如~\ref{dcl.spec.auto}所述。

\pnum
一个既不是 \grammarterm{expression} 也不是结构化绑定声明的
\grammarterm{condition} 的\defnadj{决策}{变量} 是所声明的变量。
一个作为结构化绑定声明的 \grammarterm{condition} 的决策变量在\ref{dcl.struct.bind}中规定。

\pnum
在除 \keyword{switch} 语句外的语句中,一个不是 \grammarterm{expression} 的
\grammarterm{condition} 的值是决策变量
按语境转换为 \tcode{bool} 的值\iref{conv}。
如果该
转换是非良构的,那么程序是非良构的。
一个
是表达式的 \grammarterm{condition} 的值是该表达式的值,
对于除 \keyword{switch} 外的语句,按语境转换为 \tcode{bool};
如果该转换是非良构的,那么程序是
非良构的。在用法明确的情况下,
条件的值将简称为「条件」。

\pnum
如果一个 \grammarterm{condition} 可以在句法上解析为
表达式或声明中的任意一种,
则它被解释为后者。

\pnum
在一个 \grammarterm{condition} 或一个 \grammarterm{for-range-declaration} 的
\grammarterm{decl-specifier-seq} 中,
包括该 \grammarterm{condition} 的任何 \grammarterm{structured-binding-declaration} 的声明说明符序列,
每个
\grammarterm{decl-specifier} 应是 \grammarterm{type-specifier}
或 \keyword{constexpr}。
一个 \grammarterm{for-range-declaration} 的 \grammarterm{decl-specifier-seq}
不得定义类或枚举。

\rSec1[stmt.label]{标签}%
\indextext{语句!带标签的}

\pnum
\indextext{语句!带标签的}%
\indextext{\idxcode{:}!标签说明符}%
标签可以被添加到一条语句上,或
用在\grammarterm{复合语句}中的任何位置。

\begin{bnf}
\nontermdef{label}\br
    \opt{attribute-specifier-seq} identifier \terminal{:}\br
    \opt{attribute-specifier-seq} \keyword{case} constant-expression \terminal{:}\br
    \opt{attribute-specifier-seq} \keyword{default} \terminal{:}
\end{bnf}

\begin{bnf}
\nontermdef{labeled-statement}\br
    label statement
\end{bnf}

可选的 \grammarterm{attribute-specifier-seq} 属于该标签。
\indextext{语句!\idxcode{goto}}%
带有 \grammarterm{identifier} 的标签的唯一用途是
作为 \tcode{goto} 的目标。
\indextext{标签!作用域}%
同一个函数中的两个标签不能有相同的 \grammarterm{identifier}。
一个标签可以在其声明之前就用于一条 \tcode{goto} 语句。

\pnum
\indextext{\idxgram{labeled-statement}}%
\indextext{标签!\idxcode{case}}%
\indextext{标签!\idxcode{default}}%
其 \grammarterm{label} 为 \keyword{case} 或 \keyword{default} 标签的 \grammarterm{labeled-statement}
应当被一个 \keyword{switch} 语句所包围\iref{stmt.pre},即\iref{stmt.switch}。

\pnum
一个\defnadj{control-flow-limited}{statement} 是指满足以下条件的语句 \tcode{S}:
\begin{itemize}
\item
  出现在 \tcode{S} 内部的 \keyword{case} 或 \keyword{default} 标签
  必须与位于 \tcode{S} 内部的一个 \keyword{switch} 语句相关联\iref{stmt.switch},并且
\item
  在 \tcode{S} 内声明的标签只能由位于 \tcode{S} 内部的语句所引用\iref{stmt.goto}。
\end{itemize}

\pnum
一个 \grammarterm{identifier} 标签不能被
一个 \grammarterm{expansion-statement}\iref{stmt.expand} 所包围。

\rSec1[stmt.expr]{表达式语句}%
\indextext{statement!expression}

\pnum
表达式语句具有如下形式
\begin{bnf}
\nontermdef{expression-statement}\br
    \opt{expression} \terminal{;}
\end{bnf}

该表达式是
一个被弃值表达式\iref{expr.context}。
表达式语句的所有
\indextext{side effects}%
副作用均在下一条语句执行前完成。
\indextext{statement!empty}%
当表达式语句中的\grammarterm{expression}缺失时,它被称为
一条\defnadj{null}{statement}。
\begin{note}
大多数语句都是表达式语句——通常是赋值或
函数调用。空语句可用于为
诸如\keyword{while}
语句\iref{stmt.while}之类的迭代语句提供空体。
\end{note}

\rSec1[stmt.block]{复合语句或块}%
\indextext{\idxcode{\{\}}!块语句}%

\pnum
一个\defnadj{复合}{语句}(亦称块)将一系列语句分组为一个单一语句。

\begin{bnf}
\nontermdef{compound-statement}\br
    \terminal{\{} \opt{statement-seq} \opt{label-seq} \terminal{\}}
\end{bnf}

\begin{bnf}
\nontermdef{statement-seq}\br
    statement \opt{statement-seq}
\end{bnf}

\begin{bnf}
\nontermdef{label-seq}\br
    label \opt{label-seq}
\end{bnf}

位于\grammarterm{复合语句}结尾的标签,其处理方式如同其后跟随了一个空语句。

\pnum
\begin{note}
复合语句定义了一个块作用域\iref{基本作用域}。
声明是一种\grammarterm{语句}\iref{stmt.dcl}。
\end{note}

\rSec1[stmt.select]{选择语句}%
\indextext{语句!选择|(}

\rSec2[stmt.select.general]{一般规定}%

\pnum
选择语句从若干条控制流中选取其一。

\indextext{语句!\idxcode{if}}%
\indextext{语句!\idxcode{switch}}%
%
\begin{bnf}
\nontermdef{selection-statement}\br
    \keyword{if} \opt{\keyword{constexpr}} \terminal{(} \opt{init-statement} condition \terminal{)} statement\br
    \keyword{if} \opt{\keyword{constexpr}} \terminal{(} \opt{init-statement} condition \terminal{)} statement \keyword{else} statement\br
    \keyword{if} \opt{\terminal{!}} \keyword{consteval} compound-statement\br
    \keyword{if} \opt{\terminal{!}} \keyword{consteval} compound-statement \keyword{else} statement\br
    \keyword{switch} \terminal{(} \opt{init-statement} condition \terminal{)} statement
\end{bnf}

关于条件中可选的\grammarterm{attribute-specifier-seq},见~\ref{dcl.meaning}。
\begin{note}
一条\grammarterm{init-statement}以分号结尾。
\end{note}

\pnum
\indextext{作用域!\idxgram{selection-statement}}%
\begin{note}
任何一条\grammarterm{selection-statement}及其每个子语句均具有块作用域\iref{basic.scope.block}。
\end{note}

\rSec2[stmt.if]{\keyword{if} 语句}%
\indextext{语句!\idxcode{if}}

\pnum
若条件\iref{stmt.pre}产生 \tcode{true},则执行第一条子语句。若选择语句含有 \keyword{else} 部分且条件产生 \tcode{false},则执行第二条子语句。若通过标签到达第一条子语句,则不计算条件且不执行第二条子语句。在 \keyword{if} 语句的第二种形式(即包含 \keyword{else} 的)中,若第一条子语句也是 \keyword{if} 语句,则该内部 \tcode{if} 语句应包含 \keyword{else} 部分。
\begin{footnote}
换言之,\keyword{else} 与最近的无 else \keyword{if} 相关联。
\end{footnote}

\pnum
若 \keyword{if} 语句的形式为 \tcode{if constexpr},则条件值在语境中转换到 \keyword{bool} 且转换后的表达式应是常量表达式\iref{expr.const};此形式被称为\defn{constexpr if 语句}。若转换后条件的值为 \tcode{false},则第一条子语句为\defn{弃置语句},否则第二条子语句(若存在)为弃置语句。在实例化外围模板实体\iref{temp.pre}期间,若条件在其实例化后不依赖值,则弃置的子语句(若有)不被实例化。constexpr if 语句的每条子语句是控制流受限语句\iref{stmt.label}。
\begin{example}
\begin{codeblock}
if constexpr (sizeof(int[2])) {}        // 允许窄化
\end{codeblock}
\end{example}
\begin{note}
弃置语句中的 odr 使用\iref{term.odr.use} 不要求实体被定义。
\end{note}
\begin{example}
\begin{codeblock}
template<typename T, typename ... Rest> void g(T&& p, Rest&& ...rs) {
  // ... 处理 \tcode{p}

  if constexpr (sizeof...(rs) > 0)
    g(rs...);       // 从未以空实参列表实例化
}

extern int x;       // 不要求 \tcode{x} 的定义

int f() {
  if constexpr (true)
    return 0;
  else if (x)
    return x;
  else
    return -x;
}
\end{codeblock}
\end{example}

\pnum
形式为
\begin{ncsimplebnf}
\keyword{if} \opt{\keyword{constexpr}} \terminal{(} init-statement 条件 \terminal{)} 语句
\end{ncsimplebnf}
的 \keyword{if} 语句等价于
\begin{ncsimplebnf}
\terminal{\{}\br
\bnfindent init-statement\br
\bnfindent \keyword{if} \opt{\keyword{constexpr}} \terminal{(} 条件 \terminal{)} 语句\br
\terminal{\}}
\end{ncsimplebnf}
且形式为
\begin{ncsimplebnf}
\keyword{if} \opt{\keyword{constexpr}} \terminal{(} init-statement 条件 \terminal{)} 语句 \keyword{else} 语句
\end{ncsimplebnf}
的 \keyword{if} 语句等价于
\begin{ncsimplebnf}
\terminal{\{}\br
\bnfindent init-statement\br
\bnfindent \keyword{if} \opt{\keyword{constexpr}} \terminal{(} 条件 \terminal{)} 语句 \keyword{else} 语句\br
\terminal{\}}
\end{ncsimplebnf}
除了 \grammarterm{init-statement} 与 \grammarterm{condition} 在同一作用域中。

\pnum
形式为 \tcode{\keyword{if} \keyword{consteval}} 的 \keyword{if} 语句被称为\defnadj{consteval if}{语句}。consteval if 语句中的 \grammarterm{statement}(若有)应是 \grammarterm{compound-statement}。
\begin{example}
\begin{codeblock}
constexpr void f(bool b) {
  if (true)
    if consteval { }
    else ;              // 错误:不是 \grammarterm{compound-statement};\keyword{else} 不与外层 \keyword{if} 相关联
}
\end{codeblock}
\end{example}

\pnum
若 consteval if 语句在明显常量求值\iref{expr.const}的语境中被求值,则执行第一条子语句。
\begin{note}
第一条子语句是立即函数语境。
\end{note}
否则,若选择语句有 \keyword{else} 部分,则执行第二条子语句。consteval if 语句的每条子语句是控制流受限语句\iref{stmt.label}。

\pnum
形式为
\begin{ncsimplebnf}
\keyword{if} \terminal{!} \keyword{consteval} compound-statement
\end{ncsimplebnf}
的 \keyword{if} 语句本身不是 consteval if 语句,但等价于 consteval if 语句
\begin{ncsimplebnf}
\keyword{if} \keyword{consteval} \terminal{\{} \terminal{\}} \keyword{else} compound-statement
\end{ncsimplebnf}
形式为
\begin{ncsimplebnf}
\keyword{if} \terminal{!} \keyword{consteval} compound-statement$_1$ \keyword{else} statement$_2$
\end{ncsimplebnf}
的 \keyword{if} 语句本身不是 consteval if 语句,但等价于 consteval if 语句
\begin{ncsimplebnf}
\keyword{if} \keyword{consteval} statement$_2$ \keyword{else} compound-statement$_1$
\end{ncsimplebnf}

\rSec2[stmt.switch]{The \keyword{switch} 语句}%
\indextext{语句!\idxcode{switch}}

\pnum
\keyword{switch} 语句根据条件的值将控制转移到多个语句之一。

\pnum
若 \grammarterm{条件} 是 \grammarterm{表达式},则该条件的值即为 \grammarterm{表达式} 的值;
否则,它为决策变量的值。
该条件的值应为整数类型、枚举类型或类类型。
若它属于类类型,则该条件在上下文中隐式转换(见\iref{conv})到某个整数或枚举类型。
若(可能转换后的)类型受整数提升(见\iref{conv.prom})影响,则该条件转换到提升后的类型。
\keyword{switch} 语句内的任何语句可以带有如下形式的一个或多个 \grammarterm{case 标号}:
\begin{ncbnf}
\indextext{标号!\idxcode{case}}%
\keyword{case} 常量表达式 \terminal{:}
\end{ncbnf}
其中,该 \grammarterm{常量表达式} 应为转换后的 \keyword{switch} 条件调整类型的常量表达式(见\iref{expr.const})。
在同一 \keyword{switch} 中的各 \grammarterm{case 常量} 经转换后的值不得有两个相同。

\pnum
\indextext{标号!\idxcode{default}}%
在 \keyword{switch} 语句内,最多只能有一个如下形式的标号:
\begin{codeblock}
default :
\end{codeblock}

\pnum
\keyword{switch} 语句可以嵌套;一个 \keyword{case} 或 \keyword{default} 标号关联于其所在的最近外层 \keyword{switch} 语句。

\pnum
执行 \keyword{switch} 语句时,先求值其条件。
\indextext{标号!\idxcode{case}}%
若某 \grammarterm{case 常量} 的值与条件值相同,则控制转到匹配的 \grammarterm{case 标号} 后的语句。
若无一 \grammarterm{case 常量} 与条件值匹配,且有
\indextext{标号!\idxcode{default}}%
\keyword{default} 标号,则控制转到该 \keyword{default} 标号后的语句。
若无匹配且无 \keyword{default},则 \keyword{switch} 内的语句均不执行。

\pnum
\keyword{case} 和 \keyword{default} 标号自身不改变控制流;控制流将不受阻碍地穿过这些标号。欲退出 \keyword{switch},见 \keyword{break}(\ref{stmt.break})。
\begin{note}
通常,作为 \keyword{switch} 主体的子语句是复合语句,且 \keyword{case} 和 \keyword{default} 标号出现在该(复合)子语句包含的顶层语句上,但这并非必须。
\indextext{语句!声明于 \tcode{switch}}%
声明可以出现在 \keyword{switch} 语句的子语句中。
\end{note}

\pnum
形式如下的 \keyword{switch} 语句:
\begin{ncsimplebnf}
\keyword{switch} \terminal{(} 初始化语句 条件 \terminal{)} 语句
\end{ncsimplebnf}
等价于
\begin{ncsimplebnf}
\terminal{\{}\br
\bnfindent 初始化语句\br
\bnfindent \keyword{switch} \terminal{(} 条件 \terminal{)} 语句\br
\terminal{\}}
\end{ncsimplebnf}
但此时 \grammarterm{初始化语句} 与 \grammarterm{条件} 处于同一作用域。

\indextext{语句!选择|)}

\rSec1[stmt.iter]{迭代语句}%
\indextext{statement!iteration|(}

\rSec2[stmt.iter.general]{概述}%

\pnum
迭代语句指定循环。

\indextext{statement!\idxcode{while}}%
\indextext{statement!\idxcode{do}}%
\indextext{statement!\idxcode{for}}%
%
\begin{bnf}
\nontermdef{iteration-statement}\br
    \keyword{while} \terminal{(} condition \terminal{)} statement\br
    \keyword{do} statement \keyword{while} \terminal{(} expression \terminal{)} \terminal{;}\br
    \keyword{for} \terminal{(} init-statement \opt{condition} \terminal{;} \opt{expression} \terminal{)} statement\br
    \keyword{for} \terminal{(} \opt{init-statement} for-range-declaration \terminal{:} for-range-initializer \terminal{)} statement
\end{bnf}

\begin{note}
一个 \grammarterm{init-statement} 以分号结尾。
\end{note}

\pnum
\indextext{scope!\idxgram{iteration-statement}}%
\grammarterm{iteration-statement} 中的子语句隐式地定义了一个块作用域\iref{basic.scope},每次循环都会进入和退出该作用域。
如果 \grammarterm{iteration-statement} 中的子语句是单个语句且非 \grammarterm{compound-statement},则如同被重写为包含原语句的 \grammarterm{compound-statement}。
\begin{example}
\begin{codeblock}
while (--x >= 0)
  int i;
\end{codeblock}
可以等价地重写为
\begin{codeblock}
while (--x >= 0) {
  int i;
}
\end{codeblock}
因此在 \keyword{while} 语句之后,\tcode{i} 不再处于作用域内。
\end{example}

\pnum
一个 \defnadj{trivially empty}{迭代语句} 是匹配以下形式之一的迭代语句:
\begin{itemize}
\item \tcode{while (} \grammarterm{expression} \tcode{) ;}
\item \tcode{while (} \grammarterm{expression} \tcode{) \{ \}}
\item \tcode{do ; while (} \grammarterm{expression} \tcode{) ;}
\item \tcode{do \{ \} while (} \grammarterm{expression} \tcode{) ;}
\item \tcode{for (} \grammarterm{init-statement} \opt{\grammarterm{expression}} \tcode{; ) ;}
\item \tcode{for (} \grammarterm{init-statement} \opt{\grammarterm{expression}} \tcode{; ) \{ \}}
\end{itemize}
平凡空迭代语句的 \defnadj{controlling}{表达式} 是 \tcode{while}、\tcode{do} 或 \tcode{for} 语句中的 \grammarterm{expression}(或者对于没有 \grammarterm{expression} 的 \tcode{for} 语句,则为 \tcode{true})。
一个 \defnadj{trivial infinite}{循环} 是一个平凡空迭代语句,其转换后的控制表达式,当解释为 \grammarterm{constant-expression}\iref{expr.const} 时,是一个常量表达式并且求值为 \tcode{true}。
平凡无限循环的 \grammarterm{statement} 被替换为对函数 \tcode{std::this_thread::yield}\iref{thread.thread.this} 的调用;
在独立式实现中是否发生此替换是 \impldef{是否将平凡无限循环的 \grammarterm{statement} 替换为对函数 \tcode{std::this_thread::yield} 的调用}。
\begin{note}
在独立式环境中,不保证并发的前进性;
因此此类系统需要显式的协作。
调用 yield 可以在原本没有协作意图的地方添加隐式协作。
\end{note}

\rSec2[stmt.while]{The \keyword{while} statement}%
\indextext{statement!\idxcode{while}}

\pnum
在 \keyword{while} 语句中,子语句被重复执行,直到条件\iref{stmt.pre}的值变为
\tcode{false}。每次执行子语句之前都会进行测试。

\pnum
\indextext{statement!declaration in \keyword{while}}%
\keyword{while} 语句等价于
\begin{ncsimplebnf}
\exposid{label} \terminal{:}\br
\terminal{\{}\br
\bnfindent \keyword{if} \terminal{(} condition \terminal{)} \terminal{\{}\br
\bnfindent \bnfindent statement\br
\bnfindent \bnfindent \keyword{goto} \exposid{label} \terminal{;}\br
\bnfindent \terminal{\}}\br
\terminal{\}}
\end{ncsimplebnf}
\begin{note}
条件中创建的变量会在每次循环迭代时被销毁并重新创建。
\begin{example}
\begin{codeblock}
struct A {
  int val;
  A(int i) : val(i) { }
  ~A() { }
  operator bool() { return val != 0; }
};
int i = 1;
while (A a = i) {
  // ...
  i = 0;
}
\end{codeblock}
在 while 循环中,构造函数和析构函数各被调用了两次,一次针对条件为 true 的情况,一次针对条件为 false 的情况。
\end{example}
\end{note}

\rSec2[stmt.do]{\keyword{do} 语句}%
\indextext{语句!\idxcode{do}}

\pnum
表达式被语境转换为 \tcode{bool}\iref{conv};
若该转换非良构,则程序非良构。

\pnum
在 \keyword{do} 语句中,子语句被反复执行,
直至表达式的值变为 \tcode{false}。
检测在每次执行语句之后进行。

\rSec2[stmt.for]{The \tcode{for} 语句}%
\indextext{语句!\idxcode{for}}

\pnum
\keyword{for} 语句
\begin{ncsimplebnf}
\keyword{for} \terminal{(} init-statement \opt{条件} \terminal{;} \opt{表达式} \terminal{)} 语句
\end{ncsimplebnf}
等价于
\begin{ncsimplebnf}
\terminal{\{}\br
\bnfindent init-statement\br
\bnfindent \keyword{while} \terminal{(} 条件 \terminal{)} \terminal{\{}\br
\bnfindent\bnfindent 语句\br
\bnfindent\bnfindent 表达式 \terminal{;}\br
\bnfindent \terminal{\}}\br
\terminal{\}}
\end{ncsimplebnf}
但 \grammarterm{init-statement} 与 \grammarterm{条件} 处于同一作用域,并且 \grammarterm{语句} 中的
\indextext{语句!\tcode{continue} 在 \tcode{for} 中}%
\keyword{continue}(不包含在另一迭代语句内)将在重新求值 \grammarterm{条件} 之前执行 \grammarterm{表达式}。
\begin{note}
因此第一条语句指定循环的初始化;条件\iref{stmt.pre} 指定每次迭代之前进行的测试,使得条件变为 \tcode{false} 时退出循环;表达式常指定每次迭代之后的递增操作。
\end{note}

\pnum
\grammarterm{条件} 和 \grammarterm{表达式} 可以省略任意之一或全部省略。省略 \grammarterm{条件} 会使隐含的 \keyword{while} 子句等价于 \tcode{while(true)}。

\rSec2[stmt.ranged]{基于范围的 \keyword{for} 语句}%
\indextext{statement!range based for@range based \tcode{for}}

\pnum
基于范围的 \keyword{for} 语句
\begin{ncsimplebnf}
\keyword{for} \terminal{(} \opt{init-statement} for-range-declaration \terminal{:} for-range-initializer \terminal{)} statement
\end{ncsimplebnf}
等价于
\begin{ncsimplebnf}
\terminal{\{}\br
\bnfindent \opt{init-statement}\br
\bnfindent \keyword{auto} \terminal{\&\&}\exposid{range} \terminal{=} for-range-initializer \terminal{;}\br
\bnfindent \keyword{auto} \exposid{begin} \terminal{=} \exposid{begin-expr} \terminal{;}\br
\bnfindent \keyword{auto} \exposid{end} \terminal{=} \exposid{end-expr} \terminal{;}\br
\bnfindent \keyword{for} \terminal{(} \terminal{;} \exposid{begin} \terminal{!=} \exposid{end}\terminal{;} \terminal{++}\exposid{begin} \terminal{)} \terminal{\{}\br
\bnfindent\bnfindent for-range-declaration \terminal{=} \terminal{*} \exposid{begin} \terminal{;}\br
\bnfindent\bnfindent statement\br
\bnfindent \terminal{\}}\br
\terminal{\}}
\end{ncsimplebnf}
其中
\begin{itemize}
\item
若 \grammarterm{for-range-initializer} 是一个 \grammarterm{expression},
则其被视为如同被括号所围绕(故逗号运算符不能被重新解释为分隔两个 \grammarterm{init-declarator}{s});

\item \exposid{range}、\exposid{begin} 及 \exposid{end} 是为说明而定义的变量;且

\item
\exposid{begin-expr} 与 \exposid{end-expr} 按以下方式确定:

\begin{itemize}
\item 若 \exposid{range} 的类型是到数组类型 \tcode{R} 的引用,则 \exposid{begin-expr} 与 \exposid{end-expr} 分别是
\exposid{range} 及 \exposid{range} \tcode{+} \tcode{N},
其中 \tcode{N} 是数组界限。若 \tcode{R} 是未知界限数组或非完整类型的数组,则程序非良构;

\item 若 \exposid{range} 的类型是到类类型 \tcode{C} 的引用,且
于 \tcode{C} 的作用域中查找名称 \tcode{begin} 与 \tcode{end}
各自至少找到一个声明,
则 \exposid{begin-expr} 与 \exposid{end-expr} 分别是
\tcode{\exposid{range}.begin()} 与 \tcode{\exposid{range}.end()};

\item 否则,\exposid{begin-expr} 与 \exposid{end-expr} 分别是
\tcode{begin(\exposid{range})} 与 \tcode{end(\exposid{range})},
其中 \tcode{begin} 与 \tcode{end} 将经历
实参依赖查找\iref{basic.lookup.argdep}。
\begin{note}
不进行通常的无限定查找\iref{basic.lookup.unqual}。
\end{note}
\end{itemize}
\end{itemize}
\begin{example}
\begin{codeblock}
int array[5] = { 1, 2, 3, 4, 5 };
for (int& x : array)
  x *= 2;
\end{codeblock}
\end{example}
\begin{note}
某些 \grammarterm{for-range-initializer} 中临时变量的生存期被延伸以包含整个循环\iref{class.temporary}。
\end{note}
\begin{example}
\begin{codeblock}
using T = std::list<int>;
const T& f1(const T& t) { return t; }
const T& f2(T t)        { return t; }
T g();

void foo() {
  for (auto e : f1(g())) {}     // OK,\tcode{g()} 返回值的生存期被延长
  for (auto e : f2(g())) {}     // 未定义行为
}
\end{codeblock}
\end{example}

\indextext{statement!iteration|)}

\rSec1[stmt.expand]{展开语句}
\indextext{语句!展开|(}

\pnum
展开语句指定其子语句的重复实例化\iref{temp.decls.general}。

\begin{bnf}
\nontermdef{expansion-statement}\br
    \keyword{template} \keyword{for} \terminal{(}
        \opt{初始语句} for-range-declaration \terminal{:}
        expansion-initializer \terminal{)} compound-statement
\end{bnf}

\begin{bnf}
\nontermdef{expansion-initializer}\br
    表达式\br
    expansion-init-list
\end{bnf}

\begin{bnf}
\nontermdef{expansion-init-list}\br
    \terminal{\{} \opt{表达式列表} \terminal{\}}
\end{bnf}

\pnum
\grammarterm{expansion-statement} 的 \grammarterm{compound-statement}
是控制流受限语句\iref{stmt.label}。

\pnum
对于表达式 \tcode{E},令表达式
\exposid{begin-expr} 和 \exposid{end-expr} 按~\ref{stmt.ranged} 的规则确定。
若表达式不是数组类型且满足以下任一条件,则它是 \defn{可展开迭代的}:
\begin{itemize}
\item
\exposid{begin-expr} 和 \exposid{end-expr} 的形式为
\tcode{E.begin()} 和 \tcode{E.end()},或
\item
对 \tcode{begin(E)} 和 \tcode{end(E)} 的实参依赖查找
各找到至少一个函数或函数模板。
\end{itemize}

\pnum
展开语句是
\begin{itemize}
\item
若其 \grammarterm{expansion-initializer}
形式为 \grammarterm{expansion-init-list},则为 \defnadj{枚举型}{展开语句};
\item
否则,若其 \grammarterm{expansion-initializer}
是可展开迭代的表达式,则为 \defnadj{迭代型}{展开语句};
\item
否则,为 \defnadj{解构型}{展开语句}。
\end{itemize}

\pnum
展开语句 $S$ 等价于包含
\grammarterm{for-range-declaration}(含其隐含初始化)
及 $S$ 的 compound-statement 的实例化的 \grammarterm{compound-statement},具体如下:
\begin{itemize}
\item
若 $S$ 为枚举型展开语句,则 $S$ 等价于:
\begin{codeblock}
{
  @\grammarterm{初始语句}@
  @$S_{0}$@
  @\vdots@
  @$S_{N-1}$@
}
\end{codeblock}
其中 $N$ 为 \grammarterm{表达式列表} 的元素个数,
$S_{i}$ 为
\begin{codeblock}
{
  @\grammarterm{for-range-declaration}@ = @$E_{i}$@;
  @\grammarterm{compound-statement}@
}
\end{codeblock}
且 $E_{i}$ 为 \grammarterm{表达式列表} 的第 $i$ 个元素。

\item
否则,若 $S$ 为迭代型展开语句,则 $S$ 等价于:
\begin{codeblock}
{
  @\grammarterm{初始语句}@
  static constexpr auto&& @\exposidnc{range}@ = @\grammarterm{expansion-initializer}@;
  static constexpr auto @\exposidnc{begin}@ = @\exposidnc{begin-expr}@;     // 见 \ref{stmt.ranged}
  static constexpr auto @\exposidnc{end}@ = @\exposidnc{end-expr}@;         // 见 \ref{stmt.ranged}

  @$S_{0}$@
  @\vdots@
  @$S_{N-1}$@
}
\end{codeblock}
其中 $N$ 为计算表达式
\begin{codeblock}
[] consteval {
  std::ptrdiff_t result = 0;
  for (auto i = @\exposid{begin}@; i != @\exposid{end}@; ++i, ++result);
  return result;                                // \exposid{begin} 到 \exposid{end} 的距离
}()
\end{codeblock}
的结果,且 $S_{i}$ 为
\begin{codeblock}
{
  static constexpr auto @\exposid{iter}@ = @\exposid{begin}@ + i;
  @\grammarterm{for-range-declaration}@ = *@\exposid{iter}@;
  @\grammarterm{compound-statement}@
}
\end{codeblock}
变量 \exposid{range}、\exposid{begin}、\exposid{end}、\exposid{iter}
仅用于说明。
\begin{note}
若 \exposid{range}
不是常量表达式\iref{expr.const},则实例化非良构。
\end{note}

\item
否则,$S$ 为解构型展开语句,且 $S$ 等价于:
\begin{codeblock}
{
  @\grammarterm{初始语句}@
  @\opt{\keyword{constexpr}}@ auto&& [@$u_{0}$@, @$u_{1}$@, @$\dotsc$@, @$u_{N-1}$@] = @\grammarterm{expansion-initializer}@;
  @$S_{0}$@
  @\vdots@
  @$S_{N-1}$@
}
\end{codeblock}
其中 $N$ 为 \grammarterm{expansion-initializer} 类型的结构化绑定大小,且 $S_{i}$ 为
\begin{codeblock}
{
  @\grammarterm{for-range-declaration}@ = @$u_{i}$@;
  @\grammarterm{compound-statement}@
}
\end{codeblock}
关键字 \keyword{constexpr} 出现在 $u_{0}, u_{1}, \dotsc, u_{N-1}$ 的声明中
当且仅当 \keyword{constexpr} 是
\grammarterm{for-range-declaration} 的
\grammarterm{decl-specifier-seq} 中的
一个 \grammarterm{decl-specifier}。
\end{itemize}

\pnum
\begin{example}
\begin{codeblock}
consteval int f(auto const&... Containers) {
  int result = 0;
  template for (auto const& c : {Containers...}) {      // OK,枚举型展开语句
    result += c[0];
  }
  return result;
}
constexpr int c1[] = {1, 2, 3};
constexpr int c2[] = {4, 3, 2, 1};
static_assert(f(c1, c2) == 5);
\end{codeblock}
\end{example}

\pnum
\begin{example}
\begin{codeblock}
consteval int f() {
  constexpr std::array<int, 3> arr {1, 2, 3};
  int result = 0;
  template for (constexpr int s : arr) {                // OK,迭代型展开语句
    result += sizeof(char[s]);
  }
  return result;
}
static_assert(f() == 6);
\end{codeblock}
\end{example}

\pnum
\begin{example}
\begin{codeblock}
struct S {
  int i;
  short s;
};

consteval long f(S s) {
  long result = 0;
  template for (auto x : s) {                           // OK,解构型展开语句
    result += sizeof(x);
  }
  return result;
}
static_assert(f(S{}) == sizeof(int) + sizeof(short));
\end{codeblock}
\end{example}

\indextext{语句!展开|)}

\rSec1[stmt.jump]{跳转语句}%

\rSec2[stmt.jump.general]{概述}%
\indextext{statement!跳转}

\pnum
跳转语句无条件转移控制流。
\indextext{statement!跳转}%

\indextext{statement!\idxcode{break}}%
\indextext{statement!\idxcode{continue}}%
\indextext{return statement@\tcode{return} 语句|见{\tcode{return}}}%
\indextext{\idxcode{return}}%
\indextext{statement!\idxcode{goto}}%
%
\begin{bnf}
\nontermdef{jump-statement}\br
    \keyword{break} \terminal{;}\br
    \keyword{continue} \terminal{;}\br
    \keyword{return} \opt{expr-or-braced-init-list} \terminal{;}\br
    coroutine-return-statement\br
    \keyword{goto} identifier \terminal{;}
\end{bnf}

\pnum
\indextext{local variable!析构}%
\indextext{scope!析构函数与退出}%
\begin{note}
无论以何种方式离开作用域,在该作用域中构造的具有自动存储期的对象\iref{basic.stc.auto}均按其构造顺序的逆序销毁\iref{stmt.dcl}。
关于临时对象,见~\ref{class.temporary}。
然而,程序可通过调用
\indextext{\idxcode{exit}}%
\indexlibraryglobal{exit}%
\tcode{std::exit()} 或
\indextext{\idxcode{abort}}%
\indexlibraryglobal{abort}%
\tcode{std::abort()}\iref{support.start.term} 等方式终止,而不销毁具有自动存储期的对象。
\end{note}
\begin{note}
协程挂起\iref{expr.await}不被视为离开作用域。
\end{note}

\rSec2[stmt.break]{\keyword{break} 语句}%
\indextext{语句!\idxcode{break}}

\pnum
一条 \keyword{break} 语句应被\iref{stmt.pre}%
\indextext{\idxgram{iteration-statement}}%
\indextext{语句!\idxcode{switch}}%
一条 \grammarterm{iteration-statement}\iref{stmt.iter}、
一条 \grammarterm{expansion-statement}\iref{stmt.expand} 或
一条 \keyword{switch} 语句\iref{stmt.switch} 所包围。
\keyword{break} 语句导致终止最内层此类包围语句;
控制传递到被终止语句之后的语句(如果有)。

\rSec2[stmt.cont]{\keyword{continue} 语句}%
\indextext{statement!\idxcode{continue}}

\pnum
\keyword{continue}
语句应被\iref{stmt.pre}
\indextext{\idxgram{iteration-statement}}%
某个 \grammarterm{iteration-statement} 或 \grammarterm{expansion-statement} 所包围。
如果最内层包围该语句的 $X$
是一个 \grammarterm{iteration-statement}\iref{stmt.iter},
则 \keyword{continue} 语句
使得控制流转到 $X$ 的 \grammarterm{statement}
或 \grammarterm{compound-statement} 的结尾。
否则,控制流转到当前 $S_{i}$\iref{stmt.expand} 的
\grammarterm{compound-statement} 的结尾。

\rSec2[stmt.return]{\keyword{return} 语句}%
\indextext{\idxcode{return}}%
\indextext{function return|see{\tcode{return}}}%

\pnum
函数通过 \tcode{return} 语句将控制权返回给调用者。

\pnum
\tcode{return} 语句的 \grammarterm{expr-or-braced-init-list} 称为其操作数。不带操作数的 \tcode{return} 语句只能用于返回类型为 \cv{}~\keyword{void} 的函数、构造函数\iref{class.ctor}或析构函数\iref{class.dtor}。
\indextext{\idxcode{return}!构造函数和}%
\indextext{\idxcode{return}!析构函数和}%
操作数类型为 \keyword{void} 的 \tcode{return} 语句只能用于返回类型为 \cv{}~\keyword{void} 的函数。
带有任何其他操作数的 \tcode{return} 语句只能用于返回类型不是 \cv{}~\keyword{void} 的函数;
\indextext{conversion!return type}%
\tcode{return} 语句通过从操作数复制初始化\iref{dcl.init}来初始化函数调用(显式或隐式)返回的引用或纯右值结果对象。
\begin{note}
构造函数或析构函数没有返回类型。
\end{note}
\begin{note}
如果操作数不是纯右值,或者其类型与函数的返回类型不同,则 \tcode{return} 语句可能涉及调用构造函数以执行操作数的复制或移动。
如果返回的是具有自动存储期的变量,则与 \tcode{return} 语句关联的复制操作可以被省略或转换为移动操作\iref{class.copy.elision}。
\end{note}

\pnum
结果对象的析构函数可能被调用\iref{class.dtor,except.ctor}。
\begin{example}
\begin{codeblock}
class A {
  ~A() {}
};
A f() { return A(); }   // 错误:\tcode{A} 的析构函数是私有的(即使它从未被调用)
\end{codeblock}
\end{example}

\pnum
从构造函数末尾流出、从析构函数末尾流出或从返回类型为 \cv{}~\keyword{void} 的非协程函数末尾流出,等价于不带操作数的 \tcode{return} 语句。
否则,从既不是 \tcode{main}\iref{basic.start.main} 也不是协程\iref{dcl.fct.def.coroutine} 的函数末尾流出会导致未定义行为。

\pnum
调用结果的复制初始化在由 \tcode{return} 语句操作数确立的完整表达式末尾的临时对象销毁之前顺序发生,而该临时对象的销毁又在包围 \tcode{return} 语句的块的局部变量销毁\iref{stmt.jump}之前顺序发生。
\begin{note}
这些操作在函数每个剩余包围块的局部变量销毁\iref{stmt.dcl}之前顺序发生,
而该局部变量的销毁又在函数后置条件断言求值\iref{dcl.contract.func}之前顺序发生,
而后置条件断言求值又在函数形参销毁\iref{expr.call}之前顺序发生。
\end{note}

\pnum
在返回类型为引用的函数中(为 \tcode{std::is_convertible}\iref{meta.rel} 发明的函数除外),将返回的引用绑定到临时表达式\iref{class.temporary}的 \tcode{return} 语句是病态的。
\begin{example}
\begin{codeblock}
auto&& f1() {
  return 42;            // 病态
}
const double& f2() {
  static int x = 42;
  return x;             // 病态
}
auto&& id(auto&& r) {
  return static_cast<decltype(r)&&>(r);
}
auto&& f3() {
  return id(42);        // OK,但可能是一个 bug
}
\end{codeblock}
\end{example}

\rSec2[stmt.return.coroutine]{\keyword{co_return} 语句}%
\indextext{\idxcode{co_return}}%
\indextext{coroutine return|see{\tcode{co_return}}}%

\begin{bnf}
\nontermdef{coroutine-return-statement}\br
    \keyword{co_return} \opt{expr-or-braced-init-list} \terminal{;}
\end{bnf}

\pnum
\keyword{co_return} 语句将控制转移给协程的调用者或恢复者\iref{dcl.fct.def.coroutine}。
协程不得包含 \tcode{return} 语句\iref{stmt.return}。
\begin{note}
对于此判断,无论 \tcode{return} 语句是否被丢弃语句\iref{stmt.if}包围都无关紧要。
\end{note}

\pnum
\keyword{co_return} 语句的 \grammarterm{expr-or-braced-init-list} 称为其操作数。
令 \placeholder{p} 为一个命名协程承诺对象\iref{dcl.fct.def.coroutine}的左值。
\keyword{co_return} 语句等价于:
\begin{ncsimplebnf}
\terminal{\{} S\terminal{;} \keyword{goto} \exposid{final-suspend}\terminal{;} \terminal{\}}
\end{ncsimplebnf}
其中 \exposid{final-suspend} 是 \ref{dcl.fct.def.coroutine} 中定义的仅供说明的标签,
而 \placeholder{S} 定义如下:
\begin{itemize}
\item
若操作数是 \grammarterm{braced-init-list} 或非 \keyword{void} 类型的表达式,
则 \placeholder{S} 是 \placeholder{p}\tcode{.return_value(}\grammarterm{expr-or-braced-init-list}{}\tcode{)}。
表达式 \placeholder{S} 应是 \keyword{void} 类型的纯右值。

\item
否则,
\placeholder{S} 是 \grammarterm{compound-statement} \tcode{\{}{ }\opt{\grammarterm{expression}} \tcode{;} \placeholder{p}\tcode{.return_void()}\tcode{;{ }\}}。
表达式 \placeholder{p}\tcode{.return_void()} 应是 \keyword{void} 类型的纯右值。
\end{itemize}

\pnum
若在承诺类型的范围内查找名称 \tcode{return_void} 找到任何声明,
则流出协程的 \grammarterm{function-body} 末端等价于不带操作数的 \keyword{co_return};
否则流出协程的 \grammarterm{function-body} 末端将导致未定义行为。

\rSec2[stmt.goto]{The \keyword{goto} statement}%
\indextext{statement!\idxcode{goto}}

\pnum
\keyword{goto} 语句无条件地将控制权转移到由标识符标记的语句。该标识符必须是位于当前函数中的一个\indextext{label}%
标签(\iref{stmt.label})。

\rSec1[stmt.contract.assert]{断言语句}

\begin{bnf}
\nontermdef{assertion-statement}\br
    \terminal{contract_assert} \opt{attribute-specifier-seq} \terminal{(} conditional-expression \terminal{)} \terminal{;}
\end{bnf}

\pnum
\indexdefn{contract assertion!statement|see{assertion, statement}}
\indextext{assertion!statement}
\grammarterm{assertion-statement}引入一个契约断言\iref{basic.contract}。
可选的\grammarterm{attribute-specifier-seq}属于所引入的契约断言。

\pnum
\grammarterm{assertion-statement}的谓词\iref{basic.contract.general}
是其\grammarterm{conditional-expression}在上下文中转换为\tcode{bool}的结果。

\pnum
连续\grammarterm{assertion-statement}的求值是按顺序求值\iref{basic.contract.eval}
这些\grammarterm{assertion-statement}引入的契约断言。
\begin{note}
因此,一个\grammarterm{assertion-statement}序列可以作为一个整体被重复求值。
\begin{example}
\begin{codeblock}
int f(int i)
{
  contract_assert(i == 0);  // \#1
  contract_assert(i >= 0);  // \#2
  return 0;
}
int g = f(0);   // 可以求值 \#1, \#2, \#1, \#2
\end{codeblock}
\end{example}
\end{note}

\rSec1[stmt.dcl]{声明语句}%
\indextext{语句!声明}

\pnum
声明语句在块中引入一个或多个新名称;其形式为
\begin{bnf}
\nontermdef{declaration-statement}\br
    block-declaration
\end{bnf}
\begin{note}
如果声明引入的标识符先前已在外层块中声明,
\indextext{声明隐藏|see{名称隐藏}}%
\indextext{名称隐藏}%
\indextext{块(语句)!结构}%
则外层声明在该块的其余部分中被隐藏\iref{basic.lookup.unqual},
之后其作用恢复。
\end{note}

\pnum
\indextext{块(语句)!初始化于其中}%
\indextext{初始化!自动}%
\indextext{活跃|see{变量,活跃}}%
具有自动存储期\iref{basic.stc.auto}的块变量在其\grammarterm{init-declarator}之后的整个所属作用域内均为\defnx{活跃}{变量!活跃}。
\indextext{初始化!跳转跳过}%
\indextext{\idxcode{goto}!初始化与}%
在函数内每次从点 $P$ 到点 $Q$ 的控制转移(包括语句的顺序执行)时,
所有在 $P$ 处活跃而在 $Q$ 处不活跃的具有自动存储期的块变量按其构造的逆序销毁。
随后,所有在 $Q$ 处活跃而在 $P$ 处不活跃的具有自动存储期的块变量按声明顺序初始化;
除非所有这些变量都具有空初始化\iref{basic.life},否则控制转移不得为跳转。
\begin{footnote}
从\keyword{switch}语句的条件到\keyword{case}标签的控制转移在此方面被视为跳转。
\end{footnote}
当执行\grammarterm{declaration-statement}时,$P$ 和 $Q$ 分别是紧接在其之前和之后的点;
当函数返回时,$Q$ 在其函数体之后。
\begin{example}
\begin{codeblock}
void f() {
  // ...
  goto lx;          // 错误:跳入\tcode{a}的作用域
  // ...
ly:
  X a = 1;
  // ...
lx:
  goto ly;          // OK,跳转意味着对\tcode{a}调用析构函数,随后立即在标签\tcode{ly}之后再次构造
}
\end{codeblock}
\end{example}

\pnum
\indextext{初始化!自动}%
\indextext{初始化!动态块作用域}%
\indextext{初始化!局部\tcode{static}}%
\indextext{初始化!局部\tcode{thread_local}}%
具有静态存储期\iref{basic.stc.static}或线程存储期\iref{basic.stc.thread}的块变量的动态初始化,
在控制首次流经其声明时执行;此类变量在其初始化完成后视为已初始化。
如果初始化因抛出异常而退出,则初始化未完成,因此下次控制进入该声明时将再次尝试。
如果控制并发地进入该声明而变量正被初始化,则并发执行应等待初始化完成。
\begin{note}
符合标准的实现不得在初始化器执行周围引入任何死锁。
死锁仍可能由程序逻辑导致;
实现只需避免因其自身同步操作导致的死锁。
\end{note}
如果控制递归地重新进入该声明而变量正被初始化,则行为未定义。
\begin{example}
\begin{codeblock}
int foo(int i) {
  static int s = foo(2*i);      // 未定义行为:递归调用
  return i+1;
}
\end{codeblock}
\end{example}

\pnum
\indextext{\idxcode{static}!局部变量的析构}%
关联于具有静态或线程存储期的块变量的对象,当且仅当它已被构造时才被销毁。
\begin{note}
\ref{basic.start.term}描述了此类对象的销毁顺序。
\end{note}

\rSec1[stmt.ambig]{歧义解析}%
\indextext{歧义!声明与表达式}

\pnum
涉及 \grammarterm{expression-statement} 与 \grammarterm{declaration} 的语法存在歧义:
以函数式显式类型转换\iref{expr.type.conv} 作为最左子表达式的 \grammarterm{expression-statement} 可能与首个 \grammarterm{declarator} 以 \tcode{(} 开头的 \grammarterm{declaration} 难以区分。
在这些情况下,\grammarterm{statement} 被视为 \grammarterm{declaration},除非另行指定。

\pnum
\begin{note}
如果 \grammarterm{statement} 在语法上不能是 \grammarterm{declaration},则不存在歧义,因此本规则不适用。
某些情况下,需要检查整个 \grammarterm{statement} 才能确定是否如此。这明确了诸多示例的含义。
\begin{example}
假定 \tcode{T} 是 \grammarterm{simple-type-specifier}\iref{dcl.type.simple},

\begin{codeblock}
T(a)->m = 7;        // expression-statement
T(a)++;             // expression-statement
T(a,5)<<c;          // expression-statement

T(*d)(int);         // declaration
T(e)[5];            // declaration
T(f) = { 1, 2 };    // declaration
T(*g)(double(3));   // declaration
\end{codeblock}

在上方最后一个示例中,指向 \tcode{T} 的指针 \tcode{g} 被初始化为 \tcode{double(3)}。
这由于语义原因显然是非良构的,但不影响语法分析。
\end{example}

其余情况均是 \grammarterm{declaration}。
\begin{example}
\begin{codeblock}
class T {
  // ...
public:
  T();
  T(int);
  T(int, int);
};
T(a);               // declaration
T(*b)();            // declaration
T(c)=7;             // declaration
T(d),e,f=3;         // declaration
extern int h;
T(g)(h,2);          // declaration
\end{codeblock}
\end{example}
\end{note}

\pnum
歧义解析纯粹是语法上的;即这种语句中出现的名字的含义,在歧义解析中一般不使用或不改变,除了判定它们是否为 \grammarterm{type-name}。
类模板会根据需要被实例化以确定一个限定名是否为 \grammarterm{type-name}。
歧义解析先于语法分析进行,且被解析为声明的语句可能是一个非良构的声明。
如果在语法分析期间,查找发现模板实参中的名字绑定到了正在被解析的(部分)声明,则程序非良构。不要求诊断。
\begin{example}
\begin{codeblock}
struct T1 {
  T1 operator()(int x) { return T1(x); }
  int operator=(int x) { return x; }
  T1(int) { }
};
struct T2 { T2(int) { } };
int a, (*(*b)(T2))(int), c, d;

void f() {
  // 歧义解析要求将此解析为声明:
  T1(a) = 3,
  T2(4),                        // \tcode{T2} 将被声明为 \tcode{T1} 类型的变量,但这将不允许
  (*(*b)(T2(c)))(int(d));       // 声明的最后部分正确解析,因为它依赖于 \tcode{T2} 为类型名
}
\end{codeblock}
\end{example}

\pnum
在语法上为歧义且可作为具有最外层带 \grammarterm{trailing-return-type} 的 \grammarterm{declarator} 的 \grammarterm{declaration} 的语句,只有当它以 \keyword{auto} 开头时才被视为 \grammarterm{declaration}。
\begin{example}
\begin{codeblock}
struct M;
struct S {
  S* operator()();
  int N;
  int M;

  void mem(S s) {
    auto(s)()->M;               // expression,\tcode{S::M} 隐藏了 \tcode{::M}
  }
};

void f(S s) {
  {
    auto(s)()->N;               // expression
    auto(s)()->M;               // function declaration
  }
  {
    S(s)()->N;                  // expression
    S(s)()->M;                  // expression
  }
}
\end{codeblock}
\end{example}
\indextext{statement|)}

